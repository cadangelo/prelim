\chapter{Literature Review} \label{ch:litrev}

The goal of this thesis work is to optimize the neutron transport step of a
coupled, multi-physics process occuring in a system that has moving components.  One
important application of this work is the quatification of the shutdown dose
rate (SDR) during maintenace operations in fusion energy systems (FES).

During the operation of a fusion device, the nuclear reactions (e.g. D-T
fusion) occuring in the plasma result in the production of high energy (~14 MeV) neutrons
that penetrate deeply into the system components.  Some of the neutron reaction
pathways result in the production of radioisotopes that persist long after
device shutdown.  The activated components emit high energy photons as they
reach stability over time.  These high energy photons
can cause grave health effects, therefore it is necessary to     
quantify the dose rate in order to ensure the safety of personnel working in
fusion facilities.  This is not only important for the time during operation
and immediately after shutdown when the device is in a static configuration,
but quantifying the dose rate during a maintenance activity when activated
components are moving around the facility.

Performing computational simulations of the radiation transport in these
devices and calculating quantities of interest, such as flux and dose rate, are
a crucial part of the fusion reactor design phase.  These simulations can
inform decisions about sustainability and safety of the device.  
This chapter will provide background on computational radiation
transport, methods for SDR analysis, methods for optimizing radiation transport
calculations, and finally  how radiation transport calculations are currently
handled in systems with moving geometries and sources.


%\section{Computational Radiation Transport}

%The neutron distribution along with and irradiation and decay scheme of interest 
%can then be input for a nuclear inventory code to generate the photon
%emission density.  This photon emission density is used as a source for the
%photon transport step which ultimately produces the photon flux in every region.
%The photon flux coupled with a flux-to-dose-rate conversion factor will give us
%the dose rate.


%There are two ways to solve the Boltzmann transport equation: deterministically
%and stochastically.
%\cite{haghighat_wagner_2003}.

%\subsection{Deterministic Codes}

% Section about analog MC and the challenges faced when trying to calculate SDR in
% FES with no VR.

\section{Analog Monte Carlo Calculations} \label{sec:analog_mc}

The quantification of the SDR requires a detailed distribution of the
neutron and photon flux throughout all regions of phase space (space, energy,
and direction).  Due to the size and complexity of FES, the most optimal way to 
obtain accurate particle distributions is through Monte Carlo (MC) radiation transport 
rather than deterministic methods.
\todo{do I mention deterministic anywhere else?}

%To quantify the dose rate, we need to know the flux of photons
%produced by the decaying radionuclides in the activated components. The
%activation is caused by neutron irradiation.
%We need to know what the neutron flux looks like in every region of phase
%space.

In general, MC calculations rely on repeated random sampling to solve
mathematical problems.  In radiation transport applications, the MC method is 
used to solve the Boltzmann transport equation \cite{l_m} through the simulation of random particle
walks through phase space.  In analog operation mode (i.e. no variance reduction), 
the source particle's position, energy, direction
and subsequent collisions are sampled from probability
distribution functions (PDFs).  Quantities of interest such as flux can be
scored, or tallied, by averaging particle behavior
in discrete regions of phase space.

One challenge incurred by MC simulations of FES is the presence of heavily
shielded regions.  The particles undergo a high degree of
collisions (absorption and scattering) in the shielding which results in low
particle fluxes in the attenuated regions.  Regions that have low
particle fluxes have higher statistical uncertainty.

The statistical error is a function of the relative error, $\Re$, which is defined as
\begin{equation} \label{eq:1.2}
		\Re = \frac{\sigma_{\overline{x}}}{{\overline{x}}}
\end{equation}
where $\overline{x}$ is the average of the tally scores, and $\sigma_{\overline{x}}$ is the standard
deviation of the tally scores.  For a well behaved tally, $\Re$ is proportional
to $1/\sqrt{N}$ where N is the number of tally scores \cite{mcnp_manual}.

The relative error is inversely proportional to the number of tally scores.
Therefore, to reliably predict results in these regions, many particle
histories need to be simulated which can require large amounts of computer time.

The efficiency of MC calculations is measured by a quantity known as the figure of merit
(FOM).  The FOM is a function of relative error, $\Re$, and computer processing
time, $t_{proc}$, 
as given by
\begin{equation} \label{eq:1.3}
	FOM = \frac{1}{{{{\Re}^2}t_{proc}}}
\end{equation}
A high FOM is desirable because it means that less computation time is needed to achieve
a reasonably low error ( >0.1 \cite{mcnp_manual}). 

\section{Shutdown Dose Rate Analysis}\label{sec:sdr_calcs}
This section will discuss the two primary workflows used to investigate the SDR:
the Direct 1-Step (D1S) \cite{d1s} and the Rigorous 2-Step (R2S) \cite{r2s}
method.  Both methods couple the neutron and photon transport via activation
analysis to calculate the SDR.

\subsection{D1S}
As its name implies, the D1S method performs coupled neutron-photon transport
in the same simulation.  It relies upon a modified version of the MC code,
MCNP5, with special cross-section data and the FISPACT nuclear inventory code
for time correction factors.  
The nuclear data available in the standard MCNP code includes prompt photon 
production cross sections, $\sigma_{n,\gamma}$.
If a prompt photon reaction is sampled,
the photon is stored until the original neutron transport is
completed. Then, the photon is transported as part of the same simulation.
The version of MCNP5 used by D1S includes a set of modified cross sections that
approximate the transmutation process.  This allows the delayed 
photons to be emitted as prompt so they can be transported in the same simulations
as neutrons. A time correction factor is later applied.
Because both neutron and photon transport occur in the same simulation,
therefore on the same geometry, D1S
%neither the D1S or Advanced D1S (an improved version with additional features)
is not currently applicable to geometries that undergo movement after shutdown.
This has been identified as a necessary improvement and the development of a
subroutine to produce portable decay photon sources for pure photon
calculations is underway \cite{adv-d1s}. 
%errmmm.. this sounds like TR2S

\subsection{R2S}\label{sec:r2s}
In constrast to D1S, the R2S method relies upon separate MC neutron and photon
transport simulations.  The transport steps are coupled through activation analysis 
by a dedicated nuclear inventory code.
The goal of the neutron transport step is to determine the neutron flux
as a function of space and energy.  This neutron flux along with a specific irradiation
and decay scenario are used as input into a nuclear inventory
code to determine the photon emmission density as a function of decay time.
The calculated photon emmission density for each decay time
is then used as the source for MC photon transport.  A photon flux tally
fitted with flux-to-dose-rate conversion factors is used to
determine the final SDR \cite{r2s}.


\section{Monte Carlo Variance Reduction Methods}\label{sec:vr_methods}

As mentioned in section \ref{sec:analog_mc}, the presence of heavy shielding in
FES presents a challenge for MC calculations because of the high degree of 
uncertainty caused by low particle fluxes. A set of techniques, known as 
variance reduction (VR),
improve the effiency of these calculations by reducing the compute time 
necessary to achieve a statistically reasonable result.
%Some MC problems are so challenging that without VR, would be impossible to 
%simulate. 

More specifically, VR methods aim to increase the FOM given in Eq. \ref{eq:1.3} 
by increasing N and decreasing $\Re$, a function of $\sigma_{\overline{x}}$,
by preferentially sampling trajectories that are likely to
contribute to the tallies of interest.
%This effectively forces more collisions in regions of phase space that are
%important to the tally of interest. 

One way this is accomplished is by sampling from biased PDFs that govern particle behavior. 
In order to compensate for this biased sampling, the particle statistical weight
is adjusted accordingly \cite{cadis}.
The relationship between the particle statistical weight, w, and the PDF that
governs particle behavior is as follows
\begin{equation} \label{eq:2.1}
		w_{biased} pdf_{biased} = w_{unbiased} pdf_{unbiased}
\end{equation}
If this biased sampling results in an event occuring more frequently than it actually does,
the particle weight is decreased and vice versa.

One of the earliest and still commonly used methods of VR is particle splitting
and rouletting.  This is particularly useful in simulations where particles
penetrate deeply into shielded regions, like in FES.
To increase the number of particle histories that can contribute to a tally of
interest, it is desirable to split particles as they enter more important 
regions and roulette particles as they enter less important regions. 
%When a particle is split into N particles, the weight of each new particle must be reduced 
%according to Eq. \ref{eq:split_weight}
%\begin{equation}\label{eq:split_weight}
%		N*w_{biased} pdf_{biased} = w_{unbiased} pdf_{unbiased}
%\end{equation}
%When a particle instead moves to a region of less importance, it is terminated 
%with a probablity of 1/N.
The decision to split or roulette particles first requires assigning an 
importance, $I$, to every region in the geometry. 
When a particle moves from a region A to a region B,
the ratio of importances is calculated.  If region B is more important than
region A such that 
$I_{B}/I_{A} \geqslant 1$, 
the particle with original weight $w_{0}$ is split into 
$N = I_{B}/I_{A}$
particles, each with weight $w_{0}/N$.  If instead region B is less important
than region A such that
$I_{B}/I_{A} < 1$, 
the particle will undergo roulette. 
The particle will survive with a probability $N$ and weight $w_{0}/N$
\cite{Carter_Cashwell_1975}.

The weight window method in the Monte Carlo N-Particle (MCNP) code utilizes both
splitting and rouletting.  A weight window is a region of phase-space that is 
assigned an upper and lower bound.  The windows can be assigned to cells in the
geometry, on a superimposed mesh, and to energy bins.  When a particle enters
a weight window, its weight is assessed; if its weight is above the upper bound,
it is split and if it is below the lower bound, it is rouletted.  There are various
methods to produce these weight window bounds automatically which drastically reduces the
time, effort, and expertise required by the analyst.  Some of these methods will 
be discussed in the following section.


\section{Automated Variance Reduction}\label{sec:auto_vr}

Historically, VR techniques have required a priori knowledge of the problem physics
in order to assign importance parameters.
Many techniques have been developed over the years to automate the selection and
assignment of these parameters to reduce computational and human effort.

One class of VR techniques, known as hybrid deterministic/MC methods,
is based upon the solution to the adjoint Boltzmann transport equation having
significance as the measure of importance of a particle to some specified
objective function.  
Because deterministic solutions to the transport equation %are less precise but
require much less computation time, they are useful as an estimate of the
adjoint particle flux throughout phase space.  
%This estimate of adjoint flux
%can then be used to determine the importance of specific regions.
To demonstrate the use of the adjoint solution as an importance function,
first start with the operator form of the linear, time-independent Boltzmann transport
equation \cite{l_m}
\begin{equation} \label{eq:3.1a}
	H\Psi(\overrightarrow{r}, E,\widehat{\Omega})  = q(\overrightarrow{r}, E,\widehat{\Omega})
\end{equation}
where $\Psi$ is the angular flux, q is the source of particles, and the operator H
which describes all particle behavior is given by
\begin{equation} \label{eq:3.1b}
		H = \widehat{\Omega} \cdot \nabla +
		    \sigma_{t}(\overrightarrow{r},E) - 
			\int_{0}^{\infty} dE'
			\int_{4\pi} d\Omega'
			\sigma_{s}( \overrightarrow{r}, E' 
			\rightarrow E, \widehat{\Omega}' 
			\rightarrow \widehat{\Omega} )
\end{equation}
where $\sigma_{t}$ is the total cross-section and $\sigma_{s}$ is the
double-differential scattering cross-section.  The source and angular flux are
functions of five independent variables: a three-dimensional position vector ($\overrightarrow{r}$) 
a two-dimensional directional vector ($\widehat{\Omega}$), and energy (E).
The adjoint identity is stated as
\begin{equation} \label{eq:3.2}
		\langle \Psi^{+} , H\Psi \rangle =
		\langle \Psi, H^{+}\Psi^{+} \rangle
\end{equation}
where $ \langle \cdot \rangle$ refers to the integration over space,
energy, and angle and the adjoint operator $H^{+}$ is given by
\begin{equation} \label{eq:3.2b}
		H^{+} = -\widehat{\Omega} \cdot \nabla +
		    \sigma_{t}(\overrightarrow{r},E) - 
			\int_{0}^{\infty} dE'
			\int_{4\pi} d\Omega'
			\sigma_{s}( \overrightarrow{r}, E 
			\rightarrow E', \widehat{\Omega} 
			\rightarrow \widehat{\Omega}' )
\end{equation}
This identity can be used to form the adjoint transport equation.
\begin{equation} \label{eq:3.3}
		H^{+}\Psi^{+} = q^{+}
\end{equation}

Substituting Eq.\ref{eq:3.1b} and \ref{eq:3.3} into Eq. \ref{eq:3.2}, 
the adjoint identity can also be written as
\begin{equation} \label{eq:3.2b}
		\langle \Psi^{+} , q \rangle =
		\langle \Psi, q^{+} \rangle
\end{equation}

As mentioned, the solution to the adjoint transport
equation will be used as an importance function
therefore the thoughtful selection of an adjoint source $q^{+}$ is needed.

Consider the equation for detector response, R
\begin{equation} \label{eq:3.4}
		R = \langle \Psi, \sigma_{d}\rangle 
\end{equation}
where $\sigma_{d}$ is a detector response function.
If the adjoint source is chosen to be equivalent to the detector response
function,
\begin{equation} \label{eq:3.5}
		q^{+} = \sigma_{d}
\end{equation}
and substituted into Eq. \ref{eq:3.4}  
\begin{equation}\label{eq:resp_adjq}
		R = \langle \Psi, q^{+} \rangle %= \langle \Psi^{+}, q \rangle
\end{equation}
the response has the same form as the right side of Eq. \ref{eq:3.2b}, therefore
the response can also be written as 
\begin{equation}\label{eq:resp_adjf}
		R = \langle \Psi^{+}, q \rangle
\end{equation}

%the adjoint flux $\Psi^{+}$ represents the importance of a region to R.
%$\sigma_{d}$. /// I think this is wrong and should be R
This final relation allows us to know the response $R$ for any source $q$ once the
adjoint flux $\Psi^{+}$ for a detector of interest is known.

\subsection{CADIS}
The Consistent Adjoint Driven Importance Sampling (CADIS) method is one of the
hybrid deterministic/MC VR
techniques that uses the adjoint solution as an importance function to 
formulate source and transport biasing parameters for MC transport \cite{cadis}.
More specifically, CADIS provides a method for generating
a biased source and the weight window lower bounds in a consistent manner.
Recall that the response, or tally, of interest in a transport calculation can be 
represented in terms of the adjoint flux by Eq. \ref{eq:resp_adjf}.
%the following equation
%\begin{equation} \label{eq:3.6}
%	R = \int_{V}d\overrightarrow{r} \int_{E} dE
%	\int_{4\pi} d\widehat{\Omega}
%	\sigma_{d}(\overrightarrow{r}, E, \widehat{\Omega})
%	\Psi(\overrightarrow{r}, E, \widehat{\Omega})
%\end{equation}
%and in terms of the adjoint flux, 
%\begin{equation} \label{eq:3.7}
%	R = \int_{V}d\overrightarrow{r} \int_{E} dE
%	\int_{4\pi} d\widehat{\Omega}
%	q(\overrightarrow{r}, E, \widehat{\Omega})
%	\Psi^{+}(\overrightarrow{r}, E, \widehat{\Omega})
%\end{equation}
%The MC solution of the response relies upon the sampling of the particle source
%distribution, $q(\overrightarrow{r}, E, \widehat{\Omega})$, represented by a PDF.
To decrease the variance, the CADIS method formulates a biased
source distribution that represents the contribution of particles from phase space
$(\overrightarrow{r}, E, \widehat{\Omega})$ 
to the total detector response, R.
\begin{equation} \label{eq:3.8}
	\widehat{q}(\overrightarrow{r}, E, \widehat{\Omega}) =
	\frac{\Psi^{+}(\overrightarrow{r}, E,\widehat{\Omega})
	q(\overrightarrow{r}, E, \widehat{\Omega})}{R}
\end{equation}
This essentially is a way to bias the sampling of source particles as a function of their 
contribuiton to the total detector response.
As previously mentioned, when sampling from a biased distribution, the particle weight
needs to be adjusted to eliminate systematic bias.  
\begin{equation} \label{eq:3.9}
	w(\overrightarrow{r}, E,
	\widehat{\Omega})\widehat{q}(\overrightarrow{r}, E, \widehat{\Omega})=
	w_{0}q(\overrightarrow{r}, E, \widehat{\Omega})
\end{equation}
Substituting in Eq. \ref{eq:3.8} and setting $w_0$ equal to one, the corrected 
particle weight is given as
\begin{equation} \label{eq:3.10}
	w(\overrightarrow{r}, E, \widehat{\Omega})=
	\frac{R}{\Psi^{+}(\overrightarrow{r}, E, \widehat{\Omega})}
\end{equation}
There is an inverse relation to the adjoint flux, or importance function.
This relationship for the particle statistical weight is used in the source sampling
and transport process so that source particles are created with weights that reside 
within their corresponding weight window.
%%The biased transport operator is derived consistently from the biased source
%%parameter.
%In the weight window technique, particles are either split or
%rouletted as they move from region to region based on the ratio of their
%importances and their weight is updated accordingly.  
%This means regions that have a high adjoint flux will have lower weight window lower bounds,
%meaning particles will be split.
%This can be extended to
%the transport biasing parameter which uses the adjoint flux as an importance
%function as follows
%The weights are used for both the source and transport biasing parameters and
%are derived %from importance sampling 
%in a consistent manner.  
%The transport is biased according to the following relationship
%\begin{equation} \label{eq:3.11}
%	w(\overrightarrow{r}, E, \widehat{\Omega})=
%	w(\overrightarrow{r}', E', \widehat{\Omega}')
%	\left [ \frac{\Psi^{+}(\overrightarrow{r}', E', \widehat{\Omega}')}
%	{\Psi^{+}(\overrightarrow{r}, E, \widehat{\Omega})} \right ]
%\end{equation}
%The consistently generated transport biasing parameter, 
The equation for weight window lower bounds
is given as 
\begin{equation} \label{eq:3.12}
	w_{l}(\overrightarrow{r}, E, \widehat{\Omega}) = 
	\frac{R}{\Psi^{+}(\overrightarrow{r}, E, \widehat{\Omega})
	(\frac{\alpha + 1}{2})}
\end{equation}
The width of the weight windows is determined by a parameter defined to be the
ratio between upper and lower bounds $\alpha =
w_{u}/w_{l}$.  MCNP uses a default value of 5 for $\alpha$.

CADIS is ideally suited to reduce the variance of a detector response in a single
target.  There are other methods, such as FW-CADIS, that are suited for reducing the variance in 
several targets or even globally.

\subsection{FW-CADIS}
The Foward-Weighted CADIS method is another hybrid deterministic/MC VR method.
FW-CADIS aims to increase the efficiency of 
detector responses globally or in multiple localized targets \cite{fwcadis}.
The goal is to create uniform particle density in the tally regions so in turn,
there is uniform statistical uncertainty in the MC solution.
This method relies upon a forward deterministic transport solution to
weight the source for adjoint deterministic transport.  The adjoint solution is
then used with the standard CADIS method to produce source and transport
biasing pararameters for the forward MC transport simulation.

If the objective is a spatially dependent total response rate, the FW-CADIS 
adjoint source is formulated as
\begin{equation}
	q^{+}(\overrightarrow{r}, E) = \frac{\sigma_d(\overrightarrow{r}, E)}
	{\int_E
	 {\phi(\overrightarrow{r}, E)\sigma_d(\overrightarrow{r}, E)} dE}
\end{equation}
where $\sigma_d(\overrightarrow{r}, E) $ is the response function.
This effectively weights the adjoint source by the inverse of %the  deterministic estimate of
the total forward response
which means that in regions with low forward flux, the adjoint flux, and therefore
importance, will be high and vice versa.  This will result in the overall goal of
nearly equal statistical uncertainty in regions of interest.  

\section{Automated Variance Reduction for Multiphysics}\label{sec:auto_vr_sdr}

In its essence, SDR analysis is the analysis of a coupled, multiphysics system;
the inital neutron irradiation is coupled to the decay photon transport
through activation analysis.
As discussed in section \ref{sec:r2s}, the R2S method requires separate MC calculations 
for the neutron and photon transport.
Applying this workflow to full-scale, 3D FES becomes impractical 
due to the computational effort required
to produce accurate space- and energy-dependent fluxes throughout the geometry.

Optimizing the final step, photon transport in the case of SDR analysis,  
can be done through a straightforward application of the CADIS method to solve 
for the response at a single detector or the FW-CADIS method if the response is desired
in multiple detectors or globally. 
%An adjoint transport calculation can be performed where the detector
%response function is set equal to the photon flux tally fitted with
%flux-to-dose-rate conversion factors
%The resulting adjoint photon flux scored in a region of phase space is equivalent
%to the importance of that region to the detector response.
%and the resulting adjoint flux can be used as an importance function
%Therefore, the adjoint flux can be used to determine source and transport
%biasing parameters.

Optimizing the initial step of a multi-step process, neutron transport in the case of SDR, is not as
straightforward.
%Because the response function depends on the subsequent
%computational steps of activation and decay photon transport \cite{mscadis}.
The Multi-Step CADIS method described in the next section 
provides an explaination for this challege and a method for solving it.

\subsection{MS-CADIS}\label{sec:mscadis}

Optimizing the inital step of a coupled, multi-step process relies upon
an importance function that represents the importance of the particles to the
final response of interest, not the response of that individual step \cite{mscadis}.  
This is challenging because the the final response of interest depends on the subsequent 
steps of the multi-step process.

MS-CADIS can be applied to any coupled, multi-step process, but when applied to
SDR calculations, it aims to increase the efficiency of the neutron transport step
using an importance function that captures the potential of regions to become 
activated and the importance of the resulting decay photons to the final SDR \cite{mscadis}.

%As shown before in Eq. *need eqn number*, //don't think this was mentioned
The detector response can be
expressed as the integral of the importance function, $I$, multiplied by the source
distribution, $q$
\begin{equation} \label{eq:4.1}
R = %\int_{V}\int_{E} 
    \langle I(\overrightarrow{r}, E), 
    q(\overrightarrow{r}, E) \rangle
    %dV dE
\end{equation}
MS-CADIS provides a method to calculate an approximation of this importance function
where the response is the final response of the multi-step process.  In the case
of an R2S calculation, the final response is the SDR caused by the decay
photons.  The SDR is defined as 
\begin{equation} \label{eq:4.2}
  SDR =  \langle \sigma_{d}(\overrightarrow{r},E_{\gamma}),
  \phi_{\gamma}(\overrightarrow{r}, E_{\gamma}) \rangle
\end{equation}
where $\sigma_{d}$ is the flux-to-dose-rate conversion factor at the position of
the detector and $\phi_{\gamma}$ is photon flux.
%Consider Eq. \ref{eq:resp_adjf} where it was shown that the response, R, is
%equal to the product of the source and adjoint flux.  
%If the adjoint source function is chosen such that the product of the adjoint
%source and the flux is equal to the response, 
%If equation \ref{eq:4.1}
%is taken to have the same form as Eq. \ref{eq:resp_adjf}, and 
If the adjoint photon
source is chosen to be $\sigma_d$, the importance function, I,
is defined as the adjoint solution, $\phi_{\gamma}^{+}$
leading to the following relationship
\begin{equation} \label{eq:4.3}
	SDR =  \langle q_{\gamma}^{+}(\overrightarrow{r},E_{\gamma}),
  \phi_{\gamma}(\overrightarrow{r}, E_{\gamma}) \rangle 
	= \langle q_{\gamma}(\overrightarrow{r},E_{\gamma}),
  \phi_{\gamma}^{+}(\overrightarrow{r}, E_{\gamma}) \rangle 
\end{equation}

Because the final goal is an importance function that represents the importance
of neutrons the final SDR, the
neutron adjoint identity is also set equal to the photon response.
\begin{equation} \label{eq:4.4}
  SDR =  \langle q_{n}^{+}\overrightarrow{r},E_{n}),
  \phi_{n}(\overrightarrow{r}, E_{n}) \rangle 
  = \langle q_{n}\overrightarrow{r},E_{n}),
  \phi_{n}^{+}(\overrightarrow{r}, E_{n}) \rangle 
\end{equation}
Combining equations \ref{eq:4.3} and \ref{eq:4.4}, 
gives the relationship between the neutron and photon responses.
\begin{equation} \label{eq:4.5}
  %SDR =  
	\langle q_{n}^{+}(\overrightarrow{r},E_{n}),
  \phi_{n}(\overrightarrow{r}, E_{n}) \rangle 
%	= \langle q_{n}(\overrightarrow{r},E_{n}),
%  \phi_{n}^{+}(\overrightarrow{r}, E_{n}) \rangle 
  %SDR 
%	=  \langle q_{p}^{+}(\overrightarrow{r},E_{p}),
%  \phi_{p}(\overrightarrow{r}, E_{p}) \rangle 
	= \langle q_{\gamma}(\overrightarrow{r},E_{\gamma}),
  \phi_{\gamma}^{+}(\overrightarrow{r}, E_{\gamma}) \rangle 
\end{equation}

In order to use the adjoint neutron flux as an importance function to optimize
the neutron transport, a solution for the adjoint neutron source, $q_n^{+}$, is needed.
This requires an equation
relating the photon source, $q_\gamma$, to the neutron flux, $\phi_n$.
%One can solve for the photon source by performing a full activation analysis,
%but as an approximation,

\subsection{GT-CADIS}
The Groupwise Transmutation (GT)-CADIS method, an implementation of MS-CADIS
solely for SDR analysis, provides a solution to the adjoint neutron source 
by calculating a coupling term that relates the neutron flux to the photon 
source \cite{gtcadis}.

The decay photon source at a single point
%is a result of neutron irradiation and 
%the two quantities can be related by the following
%definition:
%The photon source at a single point, Eq. \ref{eq:4.6} 
can be
expressed as this non-linear function of $\phi_n$ 
\begin{equation} \label{eq:4.8}
	q_{\gamma}(E_{\gamma}) = \int_{E_n} f(\phi_{n}) dE_{n}
\end{equation}
This function can not be linearized for arbitrary transmutation networks
and irradiation scenarios, but a linear relationship can
be formulated when a set of criteria, known as the Single Neutron Interaction
Low Burnup (SNILB) criteria, are met.
The SNILB criteria can be used to obtain solutions for the coupling term, 
$T(\overrightarrow{r}, E_{n}, E_{\gamma})$, which is defined by the following equation.
\begin{equation} \label{eq:Tdef}
	q_{\gamma}(\overrightarrow{r}, E_{\gamma}) = 
	\int_{E_n}T(\overrightarrow{r}, E_{n}, E_{\gamma})
	\phi_{n}(\overrightarrow{r}, E_{n}) dE_{n}
\end{equation}
If a T can be found that satisfies this relationship, Eq. \ref{eq:Tdef} can be substituted into
the photon source in Eq. \ref{eq:4.5} which can
be used to solve for the adjoint neutron source given below % in Eq. \ref{eq:4.5}.
\begin{equation} \label{eq:gt_adj_nsrc}
	q_{n}^{+}(\overrightarrow{r},E_{n})
        = \int_{E_p}T(\overrightarrow{r}, E_{n}, E_{p})
	\phi_{p}^{+}(\overrightarrow{r}, E_{p}) dE_{p}
\end{equation}



To calculate T, 
A series of single energy group neutron irradiations is performed on each material in
the geometry.  
%The total number of irradiations performed is the product of the
%number of materials and the number of neutron energy groups.  
This results in
a photon source that is a function of each neutron energy group.  According to 
Eq. \ref{eq:Tdef}, dividing this
photon source by the neutron flux results in the coupling term T.
%\todo{Move this from proposal section to here}

%GT-CADIS provides an analytical solution to T when the SNILB criteria are met.
It has been shown that for typical FES spectra, materials, and irradiation
scenarios, the SNILB criteria are met; therefore, GT-CADIS provides a solution
for T, and therefore the adjoint neutron source needed to optimize the neutron
transport for SDR analysis of FES.

%When met, the photon emission density resulting from the irradiation of a
%material with a neutron flux is the sum of contributions from each neutron
%energy group.

\section{Moving Geometries and Sources} \label{sec:moving_sys}
%This section will discuss how systems that involve moving components are
%currently modeled.

\subsection{MCNP6 Moving Objects Capability} \label{sec:mcnp_move}
Historically, Monte Carlo analysis of dynamic systems was performed using a 
series of separate simulations with different input files that contained 
step-wise changes of the geometry configuration. 
The new moving object capability that will be available in a future version
of MCNP6 allows for the
motion of objects, sources, and delayed particles during a single simulation
\cite{mcnp_moving_1}, \cite{mcnp_moving_2}.
This capability allows for rigid body transformations of 
objects including rectilinear translations and curvilinear translations and rotations.
The objects can move with constant velocity, constant acceleration, or be
relocated.
Object kinetics are not treated so the user must use caution and supply
transformations that will not cause objects to overlap.
This capability is currently applicable to MCNP's native geometry format,
constructive solid geometry (CSG), and is not available for mesh-based
geometries.

Sources can be assigned to moving objects, therefore can move with the
same dynamics as other objects in the problem.
This capability also allows for the treatment of secondary particles emitted by
objects in motion. This treatment is only approximate because the geometry is fixed during the
transport of source or delayed particles. This is a valid approximation due to
the assumption that the geometry movement is orders of magnitude slower than
particle transport.

During the MCNP simulation, source particles are tracked through the geometry
from the time of emission to termination.  If any of the source particle's
interactions result in the creation of a prompt or delayed secondary particle,
that information is stored.  After the source particle has terminated, any
stored secondary particles are retrieved and transported.
%The geometry is updated to the time a source or delayed particle is created.
In the case of delayed particles emitted from moving objects, the location, direction, energy, and
time are stored at the time of fission or activation and then at the time of
emission, the geometry configuration is updated to provide the correct 
location and orientation of the delayed particle at the time of emission. 

% would not allow for parts of the geometry to be removed or added (e.g. adding
% equipment to move modules around facility, adding shielding, )

% Reference presentation on FLUKA simulation w/ moving geom?

\subsection{MCR2S with Geometry Movement}
The Mesh Coupled implementation of R2S (MCR2S), developed by the Culham Science Center, 
was updated to allow geometry components to change location after shutdown \cite{mcr2s}.
This capability was developed to facilitate SDR calculation during maintenance and
intervention activities.  MCR2S relies on MCNP for both neutron and photon
transport steps and FISPACT for the activation calculations. 
It allows multiple components to be moved to
different locations prior to the photon transport step.

These geometry translations occur by creating a copy of the
components that will move.  Transform cards are applied to the copies. 
The original components remain in their original locations and their material
is changed to vacuum.  Any source particle that starts in one of the
components that moves after shutdown is automatically translated to the correct
location. %according to the same transformation.

The requirement that both the original component (set as void) and its
transformed copy are present during the photon transport step means that there
can be no overlap between the parts which could be problematic for small
transformations.

