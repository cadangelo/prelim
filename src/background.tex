\chapter{Literature Review} \label{ch:litrev}

% Section introducing SDR calculations, particularly for FES
\section{SDR calculations}\label{sec:sdr_calcs}
Shutdown dose rate (SDR) calculations are necessary to quantify the potential
dose to personnel working in facilities exposed to intense radiation fields
like fusion energy systems (FES).  The dose rate is caused by decay photons
that are emitted by materials irradiated by neutrons.  Therefore, these calculations involve a neutron
transport step, then an activation analysis to determine the decay photon
spectrum for a specific irradiation and decay scenario,
and finally a photon transport step to determine the dose rate.
One methodology for calculating the SDR is known as the Rigorous Two Step (R2S)
method.  This method relies on a MC code for both neutron and photon transport
and a nuclear inventory code for activation analysis.
The goal of the neutron transport step is to determine the neutron flux
discretized by space and energy.  This neutron flux along with an irradiation
and decay scenario of interest are used as input into a nuclear inventory
code.  This code will give the photon flux as a function of decay time which
is then used as the source for the photon transport step.  A photon flux tally
fitted with flux-to-dose-rate conversion factors is used during this step to
determine the final SDR \cite{Chen_2002}.


% Section about analog MC and the challenges faced when trying to calculate SDR in
% FES with no VR.
\section{Analog Monte Carlo Calculations} \label{sec:analog_mc}

To quantify the dose rate, we need to know the flux of photons
produced by the decaying radionuclides in the activated components. The
activation is caused by neutron irradiation.
We need to know what the neutron flux looks like in every region of phase
space.

We need a detailed distribution of the neutron flux throughout all regions of phase
space.
The neutron distribution along with and irradiation and decay scheme of interest 
can then be input for a nuclear inventory code to generate the photon
emission density.  This photon emission density is used as a source for the
photon transport step which ultimately produces the photon flux in every region.
The photon flux coupled with a flux-to-dose-rate conversion factor will give us
the dose rate.


There are two ways to solve the Boltzmann transport equation: deterministically
and stochastically.
The most optimal way to perform accurate, full-scale analysis of FES is
through Monte Carlo radiation transport *need source*.
In general, Monte Carlo (MC) calculations rely on repeated, random sampling to solve
mathematical problems.  The MC method can be applied to radiation transport by
solving the Boltzmann transport equation through the simulation of random particle
walks through phase space.  In analog operation mode (i.e. no variance reduction), 
the source particle's position, energy, direction
and subsequent collisions are sampled from probability
distribution functions (PDFs).  Quantities of interest such as flux can be
scored, or tallied, by averaging particle behavior
in discrete regions of phase space.

Radiation transport simulations in geometries that have thick shielding, such as
FES, become challenging for MC codes.  The particles have lots of interactions
(absorption and scattering) in the shielding regions.  This results in low
particle fluxes in these discrete regions of phase space.  

The statistical error is a function of the relative error, R, which is defined as
\begin{equation} \label{eq:1.2}
		R = \frac{S_{\overline{x}}}{{\overline{x}}}
\end{equation}
where $\overline{x}$ is the average of the tally scores, and $S_{\overline{x}}$ is the standard
deviation of the tally scores.  For a well behaved tally, R is proportional
to $1/\sqrt{N}$ where N is the number of tally scores \cite{mcnp_manual}.

To reliably predict results in these regions, many particle
histories need to be simulated which may require large amounts of computer time
\cite{haghighat_wagner_2003}.

MC calculation efficiency is measured by a quantity known as the figure of merit
(FOM).  The FOM is a function of relative error, R, and computer processing
time, T, 
as given by
\begin{equation} \label{eq:1.3}
		FOM = \frac{1}{{{R^2}T}}
\end{equation}
It is desirable to have a high FOM because it means that less computation time is needed to achieve
a reasonably low error, less than 0.1 according to the MCNP manual
\cite{mcnp_manual}. 
The relative error is inversely proportional to N.
VR methods
aim to increase the FOM by increasing N and decreasing $S_{\overline{x}}$ by sampling from biased PDFs; this effectively
forces more collisions in regions of phase space that are
important to the tally of interest. 


\section{VR Methods}\label{sec:vr_methods}

Certain MC calculations require the use of variance reduction methods in order
to complete or improve the efficiency of the calculation. 
This is accomplished
by preferentially sampling trajectories that are likely to contribute to the tallies of
interest.
In order to compensate for this biased sampling, the particle statistical weight
is adjusted accordingly.
The relationship between the particle statistical weight, w, and the PDF that
governs particle behavior is as follows
\begin{equation} \label{eq:2.1}
		w_{biased} pdf_{biased} = w_{unbiased} pdf_{unbiased}
\end{equation}

One of the earliest and still commonly used methods of VR is particle splitting
and rouletting.  This is particularly useful in deep penetration simulations,
like FES.  Generally speaking, to increase the number of particle
histories that can contribute to tallies of interest, it is desirable to split
particles as the enter more important regions and roulette particles as the
enter less important regions. 
This requires assigning an importance, I, to every
region in the geometry and adjusting the weight, w, of the new particles. 
When a particle moves from a region A to a region B,
the ratio of importances is calculated.  If region B is more important than
region A such that 
$I_{B}/I_{A} \geqslant 1$, 
the particle with original weight $w_{0}$ is split into 
$n = I_{B}/I_{A}$
particles, each with weight $w_{0}/n$.  If instead region B is less important
than
region A such that
$I_{B}/I_{A} < 1$, 
the particle will undergo roulette. 
The particle will survive with a probability n and weight $w_{0}/n$
\cite{Carter_Cashwell_1975}.
The weight window method in MCNP utilizes both splitting and
rouletting.  A weight window is a region of phase-space that is assigned an upper and
lower bound.  The windows can be assigned to cells in the geometry, on a
superimposed mesh, and to energy bins.  When a particle enters a weight window, its weight is assessed; if
its weight is above the upper bound or below the lower bound, it is either split
or rouletted, respectively.


\section{Automated VR Techniques}\label{sec:auto_vr}

%Historically, variance reduction methods require a vast amount of user expertise
%and time.
VR techniques require the user to have a priori knowledge of the problem physics
in order to assign importance parameters.
Many techniques have been developed over the years to automate the selection and
assignment of these parameters to reduce the time and expertise required by
the user.

One class of VR techniques, known as hybrid deterministic/MC methods,
is based upon
the solution to the adjoint Boltzmann transport equation having physical
significance as the measure of importance of a particle to some specified
objective function.  
Because deterministic solutions to the transport equation %are less precise but
require much less computation time, they are useful as an estimate of
particle flux throughout phase space which can then be used to determine
importance of specific regions.
To demonstrate the use of the adjoint solution as an importance function,
we will first start with the linear, time-independent Boltzmann transport
equation shown below
\begin{equation} \label{eq:3.1a}
  H\Psi = q
\end{equation}
where $\Psi$ is the angular flux, q is the source of particles, and the operator H
is given by
\begin{equation} \label{eq:3.1b}
		H = \widehat{\Omega} \cdot \nabla +
		    \sigma_{t}(\overrightarrow{r},E) - 
			\int_{0}^{\infty} dE'
			\int_{4\pi} d\Omega'
			\sigma_{s}( \overrightarrow{r}, E' 
			\rightarrow E, \widehat{\Omega}' 
			\rightarrow \widehat{\Omega} )
\end{equation}
where $\sigma_{t}$ is the total cross-section and $\sigma_{s}$ is the
differential scattering cross-section.
The adjoint identity states that
\begin{equation} \label{eq:3.2}
		\langle \Psi^{+} , H\Psi \rangle =
		\langle \Psi, H^{+}\Psi^{+} \rangle
\end{equation}
where $ \langle \cdot \rangle$ refers to the integration over space,
energy, and angle and the adjoint operator $H^{+}$ is given by
\begin{equation} \label{eq:3.2b}
		H^{+} = -\widehat{\Omega} \cdot \nabla +
		    \sigma_{t}(\overrightarrow{r},E) - 
			\int_{0}^{\infty} dE'
			\int_{4\pi} d\Omega'
			\sigma_{s}( \overrightarrow{r}, E 
			\rightarrow E', \widehat{\Omega} 
			\rightarrow \widehat{\Omega}' )
\end{equation}
The identity can also be written as
\begin{equation} \label{eq:3.2b}
		\langle \Psi^{+} , q \rangle =
		\langle \Psi, q^{+} \rangle
\end{equation}
As mentioned, the adjoint solution to the transport
equation will be used as an importance function therefore we need to solve
\begin{equation} \label{eq:3.3}
		H^{+}\Psi{+} = q^{+}
\end{equation}
which requires the thoughtful selection of an adjoint source $q^{+}$.
To demonstrate the physical significance of the adjoint, we will consider the
detector response, R
\begin{equation} \label{eq:3.4}
		R = \langle \Psi, \sigma_{d}\rangle 
\end{equation}
where $\sigma_{d}$ is a detector response function.
If we choose the adjoint source to be equivalent to the detector response
function,
\begin{equation} \label{eq:3.5}
		q^{+} = \sigma_{d}
\end{equation}
and substitute this into Eq. \ref{eq:3.4} and Eq. \ref{eq:3.2b} 
\begin{equation}
		R = \langle \Psi, q^{+} \rangle = \langle \Psi^{+}, q \rangle
\end{equation}
the adjoint flux $\Psi^{+}$ represents the importance of a region to
$\sigma_{d}$.
This final relation allows us to know the response R for any source q once the
adjoint flux $\Psi^{+}$ for a detector of interest is known.

The Consistent Adjoint Driven Importance Sampling (CADIS) method is one of the
hybrid deterministic/MC  VR
techniques that uses the adjoint solution to 
formulate source and transport biasing parameters for MC transport.
More specifically, CADIS determines the parameters for source biasing and the
weight window lower bounds in a consistent manner.
The response, or tally, of interest in a transport calculation can be represented by
the following equation
\begin{equation} \label{eq:3.6}
	R = \int_{V}d\overrightarrow{r} \int_{E} dE
	\int_{4\pi} d\widehat{\Omega}
	\sigma_{d}(\overrightarrow{r}, E, \widehat{\Omega})
	\Psi(\overrightarrow{r}, E, \widehat{\Omega})
\end{equation}
and in terms of the adjoint flux, 
\begin{equation} \label{eq:3.7}
	R = \int_{V}d\overrightarrow{r} \int_{E} dE
	\int_{4\pi} d\widehat{\Omega}
	q(\overrightarrow{r}, E, \widehat{\Omega})
	\Psi^{+}(\overrightarrow{r}, E, \widehat{\Omega})
\end{equation}
The MC solution of the response relies upon the sampling of the particle source
distribution, $q(\overrightarrow{r}, E, \widehat{\Omega})$, represented by a PDF.
In an effort to decrease the variance, it is possible to sample from a biased
PDF which is given by
\begin{equation} \label{eq:3.8}
	\widehat{q}(\overrightarrow{r}, E, \widehat{\Omega}) =
	\frac{\Psi^{+}(\overrightarrow{r}, E,\widehat{\Omega})
	q(\overrightarrow{r}, E, \widehat{\Omega})}{R}
\end{equation}
This biased PDF represents the contribution of particles from phase-space
$(\overrightarrow{r}, E, \widehat{\Omega})$ to the total detector response, R.
As previously mentioned, when sampling from a biased PDF, the particle weight
needs to be adjusted to eliminate systematic bias.  
\begin{equation} \label{eq:3.9}
	w(\overrightarrow{r}, E,
	\widehat{\Omega})\widehat{q}(\overrightarrow{r}, E, \widehat{\Omega})=
	w_{0}q(\overrightarrow{r}, E, \widehat{\Omega})
\end{equation}
Substituting in Eq. \ref{eq:3.8}, the corrected particle weight is shown to have
an inverse relation to the adjoint flux, or importance function.
\begin{equation} \label{eq:3.10}
	w(\overrightarrow{r}, E, \widehat{\Omega})=
	\frac{R}{\Psi^{+}(\overrightarrow{r}, E, \widehat{\Omega})}
\end{equation}
%The biased transport operator is derived consistently from the biased source
%parameter.
In the weight window technique, particles are either split or
rouletted as they move from region to region based on the ratio of their
importances and their weight is updated accordingly.  
%This can be extended to
%the transport biasing parameter which uses the adjoint flux as an importance
%function as follows
The weights are used for both the source and transport biasing parameters and
are derived %from importance sampling 
in a consistent manner.  
The transport is biased according to the following relationship
\begin{equation} \label{eq:3.11}
	w(\overrightarrow{r}, E, \widehat{\Omega})=
	w(\overrightarrow{r}', E', \widehat{\Omega}')
	\left [ \frac{\Psi^{+}(\overrightarrow{r}', E', \widehat{\Omega}')}
	{\Psi^{+}(\overrightarrow{r}, E, \widehat{\Omega})} \right ]
\end{equation}
The width of the weight windows is determined by a parameter defined to be the
ratio between upper and lower bounds $\alpha =
w_{u}/w_{l}$.  MCNP uses a default value for $\alpha$ and the weight window lower
bounds are given by 
\begin{equation} \label{eq:3.12}
	w_{l}(\overrightarrow{r}, E, \widehat{\Omega}) = 
	\frac{R}{\Psi^{+}(\overrightarrow{r}, E, \widehat{\Omega})
	(\frac{\alpha + 1}{2})}
\end{equation}




\section{Automated VR for SDR Calculations}\label{sec:auto_vr_sdr}
There are two transport steps involved in SDR calculations; the initial
neutron transport to simulate the irradiation and the subsequent photon
transport simulating the decaying gammas.  The full-scale simulations in
large, complex FES models are very computationally expensive.

The R2S method applied to a full-scale, 3D FES becomes impractical due to the
need for accurate space- and energy-dependent fluxes generated by MC codes 
throughout the geometry.  Optimizing the photon transport step can be done
through a
straightforward application of the CADIS or FW-CADIS method *need source*.  An
adjoint transport calculation can be performed where the detector
response function  is set equal to the photon flux tally fitted with
flux-to-dose-rate conversion factors and the resulting adjoint flux can be used as an
importance function to determine source and transport biasing parameters *need
source*.  
Optimizing the neutron transport step, however, is not as
straightforward
%Because the response function depends on the subsequent
%computational steps of activation and decay photon transport \cite{Ibrahim_2015}.
because the importance function needs to represent the importance of the neutrons to the
final quantity of interest\cite{Ibrahim_2015}.
When applied to SDR calculations, the Multi-Step Consistent Adjoint Driven Importance Sampling (MS-CADIS) method
aims to increase the efficiency of the neutron transport step using an importance function
that captures the potential of regions to become activated and the importance of
the resulting decay photons to the final SDR\cite{Ibrahim_2015}.

As shown before in Eq. *need eqn number*, the detector response can be
expressed as the integral of the importance function, I, times the source
distribution, q
\begin{equation} \label{eq:4.1}
R = \int_{V}\int_{E} 
    I(\overrightarrow{r}, E)
    q(\overrightarrow{r}, E)
    dV dE
\end{equation}
MS-CADIS provides a method to calculate an approximation of this importance function
where the response is the final response of the multi-step process.  In the case
of an R2S calculation, the final response is the SDR caused by the decay
photons.  The SDR is defined as 
\begin{equation} \label{eq:4.2}
  SDR =  \langle \sigma_{d}(\overrightarrow{r},E_{p}),
  \phi_{p}(\overrightarrow{r}, E_{p}) \rangle
\end{equation}
where $\sigma_{d}$ is the flux-to-dose-rate conversion factor at the position of
the detector and $\phi_{p}$ is photon flux.
Consider Eq. \ref{eq:3.4} where it was shown that the response R is
equal to the product of the flux and adjoint source.  
%If the adjoint source function is chosen such that the product of the adjoint
%source and the flux is equal to the response, 
If equation \ref{eq:4.1}
is taken to have the same form as Eq. \ref{eq:4.2}, and the adjoint photon
source is set equal to $\sigma_d$, the importance function, I,
is defined as the solution of the adjoint transport equation, $\phi_{p}^{+}$
leading to the following relationship
\begin{equation} \label{eq:4.3}
	SDR =  \langle q_{p}^{+}(\overrightarrow{r},E_{p}),
  \phi_{p}(\overrightarrow{r}, E_{p}) \rangle 
	= \langle q_{p}(\overrightarrow{r},E_{p}),
  \phi_{p}^{+}(\overrightarrow{r}, E_{p}) \rangle 
\end{equation}
The goal of MS-CADIS is to develop an importance function that represents the
importance of the neutrons to the final SDR, which is done by setting the
neutron adjoint identity equal to the photon response in a relationship
equivalent to that shown in \ref{eq:4.3}
\begin{equation} \label{eq:4.4}
  SDR =  \langle q_{n}^{+}\overrightarrow{r},E_{n}),
  \phi_{n}(\overrightarrow{r}, E_{n}) \rangle 
  = \langle q_{n}\overrightarrow{r},E_{n}),
  \phi_{n}^{+}(\overrightarrow{r}, E_{n}) \rangle 
\end{equation}
Combining these adjoint identities in Eq. \ref{eq:4.3} and \ref{eq:4.4}, 
gives a relationship
\begin{equation} \label{eq:4.5}
  %SDR =  
	\langle q_{n}^{+}(\overrightarrow{r},E_{n}),
  \phi_{n}(\overrightarrow{r}, E_{n}) \rangle 
%	= \langle q_{n}(\overrightarrow{r},E_{n}),
%  \phi_{n}^{+}(\overrightarrow{r}, E_{n}) \rangle 
  %SDR 
%	=  \langle q_{p}^{+}(\overrightarrow{r},E_{p}),
%  \phi_{p}(\overrightarrow{r}, E_{p}) \rangle 
	= \langle q_{p}(\overrightarrow{r},E_{p}),
  \phi_{p}^{+}(\overrightarrow{r}, E_{p}) \rangle 
\end{equation}

In order to solve for the adjoint neutron source, $q_n^{+}$, an equation
relating the photon source, $q_p$, to the neutron flux, $\phi_n$, is needed.
%One can solve for the photon source by performing a full activation analysis,
%but as an approximation, 
The decay photon source is a result of neutron irradiation and 
the two quantities can be related by the following
definition:
\begin{equation} \label{eq:4.6}
	q_p(\overrightarrow{r}, E_{p}) = 
	\int_{E_n}T(\overrightarrow{r}, E_{n}, E_{p})
	\phi_{n}(\overrightarrow{r}, E_{n}) dE_{n}
\end{equation}
where $T(\overrightarrow{r}, E_{n}, E_{p})$ is a quantity that represents the
transmutation process.  If an adequate T can be found, this photon source can
be used to solve for the adjoint neutron source in Eq. \ref{eq:4.5}.
\begin{equation} \label{eq:4.7}
        q_{n}^{+}\overrightarrow{r},E_{n})
        = \int_{E_p}T(\overrightarrow{r}, E_{n}, E_{p})
	\phi_{p}^{+}(\overrightarrow{r}, E_{p}) dE_{p}
\end{equation}
Concentrating on the photon source at a single point, Eq. \ref{eq:4.6} can be
expressed as this non-linear function of $\phi_n$ 
\begin{equation} \label{eq:4.8}
	q_{p}(E_{p}) = \int_{E_n} f(\phi_{n}) dE_{n}
\end{equation}
The full transmutation process is complex and would be
tedious to capture.  As the purpose of calculating T is only a step in
calculating $q_{n}^{+}$ which will be used to generate our biasing parameters,
an approximation of a linear relationship between $q_p$ and $\phi_n$ will suffice.  


\section{Moving Systems} \label{sec:moving_sys}
This section will discuss how systems that involve moving components are
currently modeled.
Should reference MCNP moving CSG, delayed gamma capability.
Should reference presentation on FLUKA simulation w/ moving geom.


