\graphicspath{{/home/chelsea/prelim/src/figs}}
\myexternaldocument{prelim}

\chapter{Experiment} \label{ch:experiment}

\section{Demonstration of GT-CADIS} \label{sec:gtcadis}
%% This section will show the GT-CADIS experiment and discuss why this method
%% is insufficient for dynamic systems
GT-CADIS has proven to be an effective method 
for optimizing the neutron transport step of 
SDR analysis in static FES where the SNILB criteria are met.  As it stands, this method
will not provide appropriate VR parameters for the cases where activated
components are moving
after shutdown.  The following experiment will demonstrate the need for a
time-integrated adjoint photon solution in order to provide useful VR parameters for
dynamic systems.

\subsection{Problem Description} \label{sec:description}
The geometry chosen for this demonstration is a simplified representation of a 
fusion energy device shown in Fig. \ref{fig:ex.geom}.
It is composed of a chamber of Stainless Steel 316 (SS-316) with a central cavity measuring
2m x 2m x 2 m.  The walls are 2 m thick.  The chamber is surrounded by air and
there is helium in the central cavity.  A CAD model of this geometry was created and tagged
with Trelis \cite{trelis}.  A spatially uniform source of
13.8-14.2 MeV neutrons was placed in the central cavity. 
%The total neutron source intensity is $1.0*10^{20}$ n/s.
The SDR is measured with a detector after a single pulse irradiation of $10^5$ s and
decay period of $10^5$ s.  The detector is a sphere, 10 cm in radius, located
2 m outside of the steel chamber in the x-direction. The detector material was chosen to
be the same as that used in a previous GT-CADIS experiment (52.34 at. \% H-1,
47.66 at. \% C-12) \cite{gtcadis}.  

First, the R2S workflow was performed with analog (except for implicit capture)
MC neutron and photon transport steps.
Then, the GT-CADIS method was used to generate VR parameters
to optimize the neutron transport step.

\begin{figure}
	\centering
	\includegraphics[scale=0.7]{/home/chelsea/prelim/src/figs/orig_geom.png}
	\caption[Experimental Geometry] 
	{Planar view of the geometry.  Steel
	chamber with 2 m thick and central cavity measuring 2 m x 2 m x 2 m.
	The central cavity is filled with helium and chamber is surrounded by air.
	A SDR detector is located 2 m in the x-direction from the chamber.\label{fig:ex.geom}}
\end{figure}

%\subsection{Results}
\subsection{Analog R2S}\label{sec:analog}
The main steps of the R2S workflow are as follows:
\begin{enumerate}
	\item MC Neutron Transport
	\item Activation Analysis
	\item MC Photon Transport
\end{enumerate}
MCNP5 \cite{mcnp} was chosen as the MC code and ALARA \cite{alara} as the activation code.  

First, a DAGMCNP5 \cite{dagmc} simulation with $10^7$ histories was run using 
the CAD geometry generated by Trelis and an input
file that contained a Cartesian mesh flux tally over the entire geometry.  The
neutron flux tally was binned into the 175 group VITAMIN-J energy structure
\cite{vitaminj}.  The resulting
neutron flux and relative error for the 13.8-14.2 MeV energy group is shown in
Fig. \ref{fig:ex.nflux}.
\begin{figure} 
	\includegraphics[scale=0.4] {/home/chelsea/prelim/src/figs/analog_tot_n_f.png}
	\includegraphics[scale=0.4] {/home/chelsea/prelim/src/figs/analog_tot_n_err.png}
	\caption [Analog neutron flux and error] 
	{Neutron flux and relative error resulting from analog MC simulation.\label{fig:ex.ana_nflux}}
\end{figure}

The Python for Nuclear Engineering (PyNE) tookit has many useful functions and
scripts to assist in nuclear analysis \cite{pyne}. 
One script in the PyNE toolkit's R2S module was used
to construct the ALARA input files using the neutron flux mesh.
ALARA was run using FENDL2.0 nuclear data \cite{fendl}. PyNE R2S
was used again to generate a mesh-based photon source from the ALARA output.  
The photon source is shown in Fig. \ref{fig:ex.analog_psrc}.
\begin{figure} 
	\includegraphics[scale=0.4]{/home/chelsea/prelim/src/figs/analog_p_src_g1.png}
	\caption [Analog photon source]
	{Photon source generated after ALARA activation calculation using the
	 analog MC neutron transport result.\label{fig:ex.analog_psrc}}
\end{figure}


\subsection{GT-CADIS}\label{sec:gtcadis}
After the the R2S workflow was carried out with analog MC transport, the
GT-CADIS method was used to generate a biased source and weight windows 
to optimize the neutron transport step. 
The main steps of the GT-CADIS method are as follows:
\begin{enumerate}
	\item Deterministic adjoint photon transport
	\item Calculation of the GT-CADIS adjoint neutron source
	\item Deterministic adjoint neutron transport
	\item Generation of biased source and weight windows from adjoint neutron flux
\end{enumerate}

The deterministic code PARTISN \cite{partisn} was chosen to perform the adjoint transport
steps.  The source used for adjoint photon transport was a 42 energy group 
VITAMIN-J discretization of the ICRP-74 flux-to-dose conversion factors \cite{f2d}.  
The CAD geometry was discretized onto
a conformal Cartesian mesh.  The resulting adjoint photon flux mesh is shown in
Fig. \ref{fig:ex.adj_p_flux}.

\begin{figure}
	\includegraphics[scale=0.4]{/home/chelsea/prelim/src/figs/gtcadis_adj_p_g41.png}
	\caption [GT-CADIS adjoint photon flux] 
	{Adjoint photon flux used to generate adjoint neutron source.\label{fig:ex.adj_p_flux}}
\end{figure}

%This shows that......

Next, the coupling term T is calculated for each material: SS316, air, and
helium.  This is done by performing separate 1E5 s irradiation and 1E5 s decay
ALARA simulations for each of 175 neutron groups in each of the materials to 
obtain the photon source in each photon energy group as a function of the
neutron flux in each neutron energy group.  T was then calculated using 
Eq. \ref{eq:Tdef}.

This T is combined with the adjoint photon flux to generate the GT-CADIS
adjoint neutron source via Eq. \ref{eq:gt_adj_nsrc}. 
PARTISN is run again using this adjoint neutron source and the resulting
adjoint neutron flux for the 13.8-14.2 MeV is shown in Fig.
\ref{fig:ex.adj_n_flux}.

\begin{figure}
	\includegraphics[scale=0.4]{/home/chelsea/prelim/src/figs/gtcadis_adj_n_g216.png}
	\caption [GT-CADIS adjoint neutron flux] 
	{Adjoint neutron flux used to generate the biased source and weight
	windows.\label{fig:ex.adj_n_flux} }
\end{figure}

This adjoint neutron flux functions as an importance map of neutrons to the
final SDR.  In the region of the steel chamber near the detector, there is a 
high flux, therefore neutrons in this region have a high importance to the SDR.
In contrast, there is a low flux in the regions on the
far side of the detector.  Neutrons in this region are not as important to the
SDR.

The adjoint neutron flux is used to generate the biased source and weight
windows via the CADIS method.  These are seen in Fig. \ref{fig:ex.biased_src} and
Fig. \ref{fig:ex.wwinp}.

\begin{figure} 
	\includegraphics[scale=0.45]{/home/chelsea/prelim/src/figs/biased_n_src.png}
	\caption [GT-CADIS biased neutron source] 
	{Biased neutron source generated with GT-CADIS method.\label{fig:ex.biased_src}}
\end{figure}

\begin{figure} 
	\includegraphics[scale=0.4]{/home/chelsea/prelim/src/figs/gtcadis_wwn.png}
	\caption [GT-CADIS weight window mesh]
	{Weight window mesh generated with GT-CADIS method.\label{fig:ex.wwinp}}
\end{figure}

The biased source and weight window files were used to optimize the neutron
transport step of R2S.  A DAGMCNP5 simulation with $10^7$ histories was performed using these VR
parameters and the resulting neutron flux and relative error are shown in Fig.
\ref{fig:ex.gt_nflux}.

\begin{figure} 
	\includegraphics[scale=0.4]{/home/chelsea/prelim/src/figs/gtcadis_tot_n_f.png}
	\includegraphics[scale=0.4]{/home/chelsea/prelim/src/figs/gtcadis_tot_n_err.png}
	\caption [GT-CADIS neutron flux and relative error] 
	{Neutron flux and relative error resulting from MC simulation using
	 GT-CADIS biased source and weight window mesh.\label{fig:ex.gt_nflux}}
\end{figure}

ALARA was run using the neutron flux and the same irradiation and decay scenario used to calculate
T.  The photon source distribution generated is shown in Fig. \ref{fig:ex.gt_psrc}

\begin{figure} 
	\includegraphics[scale=0.4]{/home/chelsea/prelim/src/figs/gtcadis_photon_src_g1.png}
	\caption [GT-CADIS photon source]
	{Photon source generated after ALARA activation calculation using the
	GT-CADIS optimized neutron transport result.\label{fig:ex.gt_psrc}}
\end{figure}

%\begin{figure} \label{fig:ex.gt_pflux}
%  \caption Photon flux and relative error resulting from MC simulation using
%  GT-CADIS biased source and weight window mesh.
%\end{figure}


%\subsection{Discussion}

\section{Limitations of GT-CADIS for Moving Systems}

Comparing the neutron flux and relative error obtained by the analog MC 
transport \ref{fig:ex.ana_nflux}  and that obtained using the GT-CADIS method
\ref{fig:ex.gt_nflux}, it is clear to see that given the same number of
histories, the GT-CADIS method not only reduces the error in regions of the
steel chamber that are important to the SDR, but allows a solution to be
calculated in the detector region.


Now, consider if the steel
chamber was not a monolithic block, and instead
made of modular components that move after shutdown, during the
photon decay process.
%this importance map is no longer valid.  

For example,
a component of the chamber located on the far side of the detector moving to a
location near the detector as shown in Fig. \ref{fig:path}.
The photons produced in that component now become
important to the SDR at the detector.  This also means that the neutrons in
this region are important because it is the neutron irradiation that results
in photon emission.

\begin{figure} 
	\includegraphics[scale=0.6]{/home/chelsea/prelim/src/figs/moving_comp_path.png}
	\caption[Path of moving component]
	{Path of activated component moving from the far side of the SDR
	detector to position next to it.\label{fig:path}}
\end{figure}

Highlighting the region of the moving component in the  adjoint neutron flux
produced by GT-CADIS, Fig. \ref{fig:highlight}, it can be seen that this 
is no longer a valid importance map
of the neutrons to the final SDR. There is a low flux, therefore low
importance in the moving component that will eventually be positioned 
near the detector.  Because this adjoint neutron flux is used to generate
source and transport biasing parameters, neutrons will be steered away from 
interactions in this component, increasing the uncertainty in a region that will
ultimately be important to the SDR. 

\begin{figure} 
	\includegraphics[scale=0.6]{/home/chelsea/prelim/src/figs/adj_n_highlight.png}
	\caption[Adjoint neutron flux, highlighted region of moving component]
	{Adjoint neutron flux, highlighted region of moving component
	\label{fig:highlight}}
\end{figure}


%\begin{itemize}
%	\item reproduce adj flux fig, but w/ highlighted "moving component"
%	\item need to have good statistics in p src regions moving
%			close to detector
%\end{itemize}
