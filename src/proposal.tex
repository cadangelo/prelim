\myexternaldocument{prelim}

\chapter{Proposal}\label{ch:proposal}

As demonstrated by the experiment in Chapter \ref{ch:experiment},  the variance
reduction parameters generated by the GT-CADIS method are insufficient for
optimizing the neutron transport step of SDR analysis in cases that involve the movement 
of activated components after shutdown.  
This chapter will first discuss the progress towards CAD geometry movement
thus far.
Then, an updated R2S workflow that accounts for the movement of
activated geometry components after shutdown.  Next, the implementation
of the time-integrated solution to the adjoint photon transport derived in
chapter \ref{ch:tgt} is added to the GT-CADIS process of generating the
optimal adjoint neutron source. The propagation of error
in SDR analysis and some practical considerations for
using these methods will also be addressed. 
Finally, a demonstration of these methods 
summary of goals to accomplish will be given.

\section{Progress: DAGMC Simulations with Geometry Transformations}\label{sec:dagmc_trans}
%% This section will discuss the progress that has been made toward the moving
%% geometry capability (DAGMC tool and ability to read from MCNP input)
This section will discuss the progress of 1) a tool to create geometries in
different configurations and 2) an update to DAGMCNP that facilitates the
implementation of user-supplied geometry transformations. Both are fundamental to 
this thesis work.

\subsection{Production of Stepwise Geometry Files}\label{sec:timestep_geoms}
There are various scenarios that involve the motion of geometry components 
during a radiation transport simulation.  One example is the movement of
activated components of a fusion energy device during a maintenance operation.
A tool has be developed to generate CAD geometry files that capture the 
movement of components over time.  

First, a geometry file of the model in its 
original position is created. The geometry components are tagged with 
transformation numbers that correspond to stepwise motion vectors.
%that are also given as input to this tool.  
The tool loads the original geometry and a file containing the motion vectors.
The position of the components 
is updated according to these transformations and a new geometry file is 
produced for each time step.  This tool only handles rigid-body
transformations; no geometric deformations or scaling.  It also does not handle kinetics, 
so the user must be cautious to not cause any overlap of components during the
motion.  

The new geometry files that contain stepwise changes of the geometry 
configuration are used as the input geometry for
transport calculations to determine the response of interest at each time step.


\subsection{DAGMCNP Geometry Transformations}\label{sec:mcnp_tr}
The ability to read transformation (TRn) cards from the MCNP input file and 
update the position of the geometry has been added to DAGMCNP.  This
 capability relies upon the same tagging of components in the CAD geometry 
file discussed in the previous section.  The transformation will be applied to any 
geometry component that is tagged with the number on the TR card.  If using 
this capability to perform stepwise radiation transport calculations over time, 
the user would create one MCNP input file per time step.  Each input file 
should contain the TR cards necessary to update the position of each component 
to its new location for that time step.  During a DAGMCNP simulation, the 
geometry is loaded, the MCNP input file is read, the geometry position is 
updated and then transport is performed.

\section{Implementation Plan: Time-integrated SDR Analysis}\label{sec:implementation}

This section introduces a plan to generate fully optimized, time-integrated
SDR maps.
\todo{Change to time-dependent? Time-integrated gives the impression that
the SDR at each time step will be averaged together to give a total dose
during the movement.  What is more useful? Separate SDR
map at each time step or a single, averaged dose?}
%\subsection{Optimized R2S Workflow for Dynamic Systems}
%This section will walk through the implementation of a
%time-integrated R2S workflow with optimized neutron transport via
%Time-integrated Groupwise Transmutation (TGT)-CADIS. 
%First, the steps taken to generate an optimal adjoint neutron source are
%described and schematic is given in Fig. \ref{fig:gen_q_n_+}.

%5\todo{Need to make the distinction between different times.  First, there is the
%decay time, $t_{dec}$ which is the amount of time that the components have been
%decaying fter shutdown.  There is $t_{mov}$ which will be used to denote the
%discrete time step during the geometry movement.}

\subsection{Time-integrated R2S}\label{sec:tr2s}
To produce time-integrated SDR maps, the
first two steps of the R2S workflow remain unchanged and
%Because the geometry configuration is changing after shutdown,
N photon transport calculations need to be performed at each
time step of geometry movement.  The main steps are listed below and
a full implementation flowchart is given in Fig. \ref{fig:tr2s_flow}.
\begin{enumerate}
	\item{MC neutron transport simulation on geometry at time step $t_{mov}=0$}
	\item{ALARA activation analysis} 
	\item{MC photon transport simulations on geometry at each time step $t_{mov}=1..N$}
\end{enumerate}
%The MC neutron transport will be performed on the configuration of the geometry 
%during irradiation.
A tetrahedral mesh that conforms to
the original geometry configuration at $t_{mov}=0$ is used as a tally to score the neutron flux.
This mesh is tagged with the geometry transformations that will occur after
shutdown.  The tetrahedral mesh neutron flux tally along with an irradiation and decay
scenario of interest are given as input to the PyNE R2S
script to generate ALARA input files.  ALARA produces photon source
files for each decay time of interest.  These ALARA photon source files are
converted to tetrahedral mesh based sources by another PyNE R2S script.  

Because all
source mesh files generated will reflect the original position of the geometry,
they, along with the DAGMC geometry files, need to be transformed to the 
correct locations for each $t_{mov}$ with the DAGMC transform tool.  The
transformed sources and geometries can then be used as input for the MC photon
transport simulations to calculate the SDR at each time step during geometry movement.

\begin{figure}
\centering

        \tikzstyle{script} = [ellipse, fill=blue!20, draw, text centered, node distance=3cm]
        \tikzstyle{code} = [ellipse, fill=green!20, draw, text centered, node distance=3cm]
        \tikzstyle{file} = [rectangle, draw, text centered, node distance=3cm]
        \tikzstyle{line} = [draw, -latex]
        \tikzstyle{arrow} = [thick, ->, >=stealth]
        
        \begin{tikzpicture}[node distance = 2 cm, auto]
		% Optimized neutron transport
		\node [code] (nmc) {DAGMCNP neutron transport};
%		\node [file, above =0.9 of nmc] (biased) {Biased neutron
%		source};
%		\node [file, right = 0.3 of biased] (wwinp) {Weight window
%		mesh};
		\node [file, above = 0.4 of nmc] (mcinp) {DAGMCNP input
		file};
        	\node [file, right = 0.7 of mcinp] (ngeom) {DAGMC geom at
		$t_{mov}=0$};
        	\node [file, left = 0.3 of nmc] (ntally) {Tet mesh tally};
		\node [file, below =0.4 of nmc] (nflux) {$\phi_n$ tet mesh};
%		\path [line] (biased) -- (nmc);
%		\path [line] (wwinp) -- (nmc);
		\path [line] (ngeom) -- (nmc);
		\path [line] (mcinp) -- (nmc);
		\path [line] (ntally) -- (nmc);
		\path [line] (nmc) -- (nflux);
		% ALARA
		\node [script, below =0.4 of nflux] (step1) {pyne.r2s
		step1};
		\node [file, below =0.4 of step1] (alaraflux) {ALARA flux
		file};
		\node [file, left = 0.9 of alaraflux] (alarainp) {ALARA input
		file};
		\node [file, right =0.9 of alaraflux] (alaramat) {ALARA
		material library};
		\node [code, below =0.4 of alaraflux] (alara) {ALARA};
		\node [file, below =0.4 of alara] (psrc) {ALARA photon
		source};
		\node [script, below =0.4 of psrc] (step2) {pyne.r2s step2};
		\node [file, below =0.4 of step2] (psrcmesh) {$q_{\gamma}$
		tet mesh};
		\path [line] (nflux) -- (step1);
		\path [line] (step1) -- (alarainp);
		\path [line] (step1) -- (alaraflux);
		\path [line] (step1) -- (alaramat);
		\path [line] (alarainp) -- (alara);
		\path [line] (alaraflux) -- (alara);
		\path [line] (alaramat) -- (alara);
		\path [line] (alara) -- (psrc);
		\path [line] (psrc) -- (step2);
		\path [line] (step2) -- (psrcmesh);
		% Update source mesh position
		\node [script, below =0.4 of psrcmesh] (trtool) {DAGMC
		transform tool};
		\node [file, left = 0.9 of psrcmesh] (tr) {Transformation
		file};
		\node [file, below =0.4 of trtool] (psrcmoved) {$q_{\gamma,t_{mov}}$ tet mesh};
		\path [line] (psrcmesh) -- (trtool);
		\path [line] (tr) -- (trtool);
		\path [line] (trtool) -- (psrcmoved);
		% Photon Transport
%		\node [script, below = 0.4 of psrcmoved] (cadis) {cadis.py};
%		\node [file, left = 0.9 of cadis] (adjpfmesh) {$\phi_{\gamma,
%		t_{mov}}^{+}$ tet mesh};
%		\node [file, below = 0.4 of cadis] (biasedp) {Biased
%		$q_{\gamma, t_{mov}}$};
%		\node [file, right = 0.9 of biasedp] (wwinp) {Weight window
%		mesh};
		\node [file, right = 0.9 of psrcmoved] (geomtd) {DAGMC geom at
		$t_{mov}$};
		\node [file, left = 0.3 of psrcmoved] (mcpinp) {DAGMCNP input
		file};
		\node [code, below =0.4 of psrcmoved] (pmc) {DAGMCNP photon
		transport};
                \node [file, left=0.9 of pmc] (f2d) {Flux-to-dose};
		\node [file, below =0.4 of pmc] (sdr) {$SDR_{t_{mov}}$ tet mesh};
%		\path [line] (ngeom) -- (trtool);
		\draw [arrow] (ngeom) |-
		([xshift=0.5cm,yshift=0.5cm]alaramat.north east) |- (trtool);
		\path [line] (trtool) -- (geomtd);
%		\draw [arrow] (geomtd) |-
%		([xshift=0.5cm,yshift=0.5cm]wwinp.north east) |- (pmc);
		\path [line] (geomtd) -- (pmc);
%		\path [line] (geomtd) -- (pmc);
		\path [line] (mcpinp) -- (pmc);
%		\path [line] (psrcmoved) -- (cadis);
		\path [line] (psrcmoved) -- (pmc);
%		\path [line] (adjpfmesh) -- (cadis);
%		\path [line] (cadis) -- (biasedp);
%		\path [line] (biasedp) -- (pmc);
%		\path [line] (cadis) -- (wwinp);
%		\path [line] (wwinp) -- (pmc);
		\path [line] (f2d) -- (pmc);
		\path [line] (pmc) -- (sdr);

	\end{tikzpicture}

	\caption[Time-integrated R2S (TR2S) workflow]
	{Time-integrated R2S (TR2S) workflow for calculating the SDR at each 
	time step of geometry movement after shutdown, $t_{mov}$.
	Scripts are shown in
	blue ovals, physics codes in green ovals, and files in white
	rectangles.\label{fig:tr2s_flow}}
\end{figure}

\subsection{Generation of the TGT-CADIS Variance Reduction Parameters}
%The GT-CADIS VR parameters are based on the adjoint neutron source.
This section discusses the implementation of the time-integrated (T)GT-CADIS 
adjoint neutron source derived in Chapter \ref{ch:tgt}.
First, a CAD model of the geometry in its original position, that during device
operation time, is created.  This geometry is tagged with transformation
numbers corresponding to the stepwise motion vectors.
Together, these motion vectors construct a path 
from original to final location.  The DAGMC
transformation tool discussed in section \ref{sec:dagmc_trans} is used to apply 
the motion vectors and create HDF5 mesh files of the geometry at each time step.   
A tetrahedral mesh file that conforms to the geometry in its original
configuration is also created.  This mesh file is tagged with the same
transformations as the CAD geometry so that a mesh file is also
generated for each time step. 

The geometry file at each time step along with the adjoint photon source (the
flux-to-dose conversion factors) are given as input into a script that
generates a Partisn input file.  Deterministic adjoint photon transport is
carried out and a Partisn output of the resulting adjoint photon flux in each
photon energy group, h, is
returned.  This Partisn output is converted to an HDF5 Cartesian (voxel) mesh file
through a PyNE conversion method.  

Because we ultimately need to combine the contribution from each time step in each
volume element, the adjoint photon flux voxel mesh is mapped onto the
tetrahedral mesh of the geometry at that same time step.  Each tetrahedral
mesh element has an ID associated with it.  Those IDs are constant across all
tetrahedral mesh files.  This allows the contribution of the adjoint photon 
flux from each time step to be summed in each tetrahedral mesh element.

%\todo create figure showing mapping of voxel to tet mesh
%The time integration is approximated by the following discrete sum over all
%time steps

\begin{equation}\label{eq:sum}
	\phi_{\gamma,v, h}^{+} = \sum_{t_{mov}}{\phi_{\gamma,v,h,t_{mov}}^{+}}
\end{equation}

%After the contributions from each time step are summed, 
The time-integrated 
adjoint photon flux tetrahedral mesh is then converted back to a voxel mesh
to use as input for the PyNE GT-CADIS script. 
%This script uses the 
%adjoint photon flux along with the T matrix to generate an
%adjoint neutron source for deterministic transport.
%If the SNILB criteria are met, 

%\begin{equation}\label{eq:T}
%	T_{g,h} = \dfrac{q_{\gamma, h}(\phi_{n,g})}{\phi_{n,g}}
%\end{equation}
Next, the T value for each voxel, $T_{v,g,h}$, is found with Eq.
\ref{eq:T}.
To obtain the source of photons in each photon energy group, h, 
single pulse irradiations are performed with ALARA.
Each material in the problem is irradiated with a single energy group of
neutrons, g, and allowed to decay to the time of interest.  The value
of $T_{v,g,h}$ is assigned to each voxel by finding the underlying material.
If the voxel is composed of more than one material, the T value assigned is
a volume-weighted average of the composite material.
%This assumes that T does not change during the time the geometry is
%moving, $t_{mov}$.  

Combining the calculated T with the time-integrated adjoint photon solution,
yields the TGT-CADIS adjoint neutron source given by Eq.
\ref{eq:tgt_n_src}.  The full implementation workflow is shown in Fig.
\ref{fig:gen_q_n_+}.
%\begin{equation}\label{eq:tgt_n_src}
%	q_{n,v,g}^{+} =
%	\sum_{t_{mov}}\Bigg(\sum_{h} T_{v,g,h} \phi_{\gamma,v,h,t_{mov}}^{+}\Bigg)
%\end{equation}



\begin{figure}
\centering

        \tikzstyle{script} = [ellipse, fill=blue!20, draw, text centered, node distance=3cm]
        \tikzstyle{code} = [ellipse, fill=green!20, draw, text centered, node distance=3cm]
        \tikzstyle{file} = [rectangle, draw, text centered, node distance=3cm]
        \tikzstyle{line} = [draw, -latex']
        
        \begin{tikzpicture}[node distance = 2 cm, auto]
        	%Generate adj gamma Partisn input
                \node [file] (pgeom_0) {DAGMC geom at $t_{mov}=0$};
		\node [script, below = 0.4 of pgeom_0] (dagtrans) {DAGMC
		transform tool};
                \node [script, below = 0.4 of dagtrans] (partgen) {part.py};
                \node [file, left = 0.9 of partgen] (pgeom_i) {DAGMC geom
		at $t_{mov}$};
                \node [file, right=0.9 of partgen] (f2d) {Flux-to-dose conversion factors};
                \node [file, below =0.4 of partgen] (ppartinp) {Partisn input file};
                \path [line] (pgeom_0) -- (dagtrans);
		\path [line] (dagtrans) -- (pgeom_i);
                \path [line] (pgeom_i) -- (partgen);
                \path [line] (f2d) -- (partgen);
        	\path [line] (partgen) -- (ppartinp);
        	% Partisn adj gamma transport
                \node [code, below =0.4 of ppartinp] (partadjp) {Partisn adjoint $\gamma$ transport};
        	\node [file, below =0.4 of partadjp] (padjf)
		{Partisn$\phi_{\gamma, t_{mov}}^{+}$ file};
        	\node [script, below =0.4 of padjf] (pf2mesh) {atflux\_to\_mesh.py};
        	\node [file, below =0.4 of pf2mesh] (padjfvmesh)
		{$\phi_{\gamma,t_{mov}}^{+}$ voxel mesh};
        	\node [script, below =0.4 of padjfvmesh] (conv_v_t){Convert voxel to tet};
        	\node [file, below =0.4 of
		conv_v_t](padjftmesh){$\phi_{\gamma,t_{mov}}^{+}$ tet mesh};
		\node [script, below =0.4 of padjftmesh](sum){Sum
		$\phi_{\gamma,t_{mov}}^{+}$over all time steps};
		\node [file, below =0.4 of sum](avgadjft){$\phi_{\gamma}^{+}$ tet mesh};
        	\node [script, below =0.4 of avgadjft] (conv_t_v){Convert tet to voxel};
		\node [file, below =0.4 of conv_t_v](avgadjfv){${\phi_{\gamma}^{+}}$ voxel mesh};
        	\path [line] (ppartinp) -- (partadjp);
        	\path [line] (partadjp) -- (padjf);
        	\path [line] (padjf) -- (pf2mesh);
        	\path [line] (pf2mesh) -- (padjfvmesh);
		\path [line] (padjfvmesh) -- (conv_v_t);
		\path [line] (conv_v_t) -- (padjftmesh);
		\path [line] (padjftmesh) -- (sum);
		\path [line] (sum) -- (avgadjft);
		\path [line] (avgadjft) -- (conv_t_v);
		\path [line] (conv_t_v) -- (avgadjfv);
        	% Generate adj n source
        	\node [script, below =0.4 of avgadjfv] (gtcadis) {gtcadis.py}; 
		\node [file, left =0.9 of gtcadis] (ngeom) {DAGMC geom at
		$t_{0}$};
        	\node [file, right =0.9 of gtcadis] (matlib) {ALARA material library};
        	\node [file, below =0.4 of gtcadis] (nadjs) {$q_{n}^{+}$ voxel mesh};
        	\path [line] (ngeom) -- (gtcadis);
        	\path [line] (avgadjfv) -- (gtcadis);
        	\path [line] (matlib) -- (gtcadis);
        	\path [line] (gtcadis) -- (nadjs);
        
        \end{tikzpicture}

	\caption [Workflow to generate TGT-CADIS adjoint neutron source]
	{Workflow for generating the optimal
	adjoint neutron source via the time-integrated (T)GT-CADIS method.  Scripts are shown in
	blue ovals, physics codes in green ovals, and files in white
	rectangles.\label{fig:gen_q_n_+}}
\end{figure}


Once the TGT-CADIS adjoint neutron source has been calculated, deterministic
adjoint transport is carried out and an adjoint neutron flux Partisn file is
returned.  This file is then converted to a voxel mesh.
This functions as an importance map for the forward neutron transport.
%Areas that have a high adjoint neutron flux will have high importance and those with
%low flux will have low importance.  
This map will reflect the movement of geometry during the decay period and give
appropriate importance to regions that contribute to the SDR at any point
during geometry movement.
This adjoint neutron flux mesh is used to generate a biased neutron source and a 
weight window mesh via the CADIS method.

\begin{figure}\label{gen_biased}
\centering

        \tikzstyle{script} = [ellipse, fill=blue!20, draw, text centered, node distance=3cm]
        \tikzstyle{code} = [ellipse, fill=green!20, draw, text centered, node distance=3cm]
        \tikzstyle{file} = [rectangle, draw, text centered, node distance=3cm]
        \tikzstyle{line} = [draw, -latex]
        
        \begin{tikzpicture}[node distance = 2 cm, auto]
        	%Generate adj neutron Partisn input
                \node [script] (partgen) {part.py};
		\node [file, left =0.9 of partgen] (ngeom) {DAGMC geom at
		$t_{mov}=0$};
                \node [file, right=0.9 of partgen] (nadjs) {$q_{n}^{+}$ voxel mesh};
                \node [file, below =0.4 of partgen] (npartinp) {Partisn input file};
                \path [line] (ngeom) -- (partgen);
                \path [line] (nadjs) -- (partgen);
        	\path [line] (partgen) -- (npartinp);
        	% Partisn adj neutron transport
                \node [code, below =0.4 of npartinp] (partadjn) {Partisn adjoint neutron transport};
        	\node [file, below =0.4 of partadjn] (nadjf) {Partisn $\phi_{n}^{+}$ file};
        	\node [script, below =0.4 of nadjf] (pf2mesh) {atflux\_to\_mesh.py};
        	\node [file, below =0.4 of pf2mesh] (nadjfmesh) {$\phi_{n}^{+}$
		voxel mesh};
        	\path [line] (npartinp) -- (partadjn);
        	\path [line] (partadjn) -- (nadjf);
        	\path [line] (nadjf) -- (pf2mesh);
        	\path [line] (pf2mesh) -- (nadjfmesh);
		%Generate biased source and WW
		\node [script, below =0.4 of nadjfmesh] (cadis) {cadis.py};
		\node [file, left =0.9 of cadis ] (unbiased) {Unbiased	$q_n$ mesh};
		\node [file, below =0.4 of cadis] (biased) {Biased $q_n$ mesh};
		\node [file, right = 0.9 of biased] (wwinp) {Weight window
		mesh};
		\path [line] (unbiased) -- (cadis);
		\path [line] (nadjfmesh) -- (cadis);
		\path [line] (cadis) -- (biased);
		\path [line] (cadis) -- (wwinp);


        \end{tikzpicture}

	\caption[Workflow to generate TGT-CADIS biased source and weight
	windows]
	{Workflow for generating a biased source and weight windows to
	optimize the neutron transport step.  Scripts are shown in
	blue ovals, physics codes in green ovals, and files in white
	rectangles.}
\end{figure}

\subsection{Fully-optimized, Time-integrated R2S Workflow}
After the biased source and weight window mesh are generated with the TGT-CADIS
method, they are used to optimize the forward neutron transport step of TR2S.

The next steps of the TR2S process, from neutron transport through activation
analysis are performed in the standard manner.  At this point, there is a
photon source tetrahedral mesh file that corresponds to the 
geometry configuration at each time step of the movement.

The source mesh file in its updated position along with the previously
generated adjoint photon flux mesh are used to generate a biased photon source 
and weight window mesh are generated via the CADIS method. 
These VR parameters along with the DAGMCNP input file %that has a photon flux tally modified by flux-to-dose conversion 
at each time step,
$t_{mov}$, are all used 
as input for the final DAGMCNP photon transport step.  This results in
a SDR mesh for each $t_{mov}$.  

In summary, the TGT-CADIS biased source and weight windows can be used to optimize the 
neutron transport and the CADIS method can be used to optimize each photon
transport step.  The fully optimized TR2S implementation is shown in Fig.
\ref{fig:otr2s_flow}.

%Each of the flux values can be multiplied by
%the interval of time spent at each position and then summed to 
 

\begin{figure}
\centering

        \tikzstyle{script} = [ellipse, fill=blue!20, draw, text centered, node distance=3cm]
        \tikzstyle{code} = [ellipse, fill=green!20, draw, text centered, node distance=3cm]
        \tikzstyle{file} = [rectangle, draw, text centered, node distance=3cm]
        \tikzstyle{line} = [draw, -latex]
        \tikzstyle{arrow} = [thick, ->, >=stealth]
        
        \begin{tikzpicture}[node distance = 2 cm, auto]
		% Optimized neutron transport
		\node [code] (nmc) {DAGMCNP neutron transport};
		\node [file, above =0.9 of nmc] (biased) {Biased neutron
		source};
		\node [file, right = 0.3 of biased] (wwinp) {Weight window
		mesh};
		\node [file, left = 0.3 of biased] (mcinp) {DAGMCNP input
		file};
        	\node [file, right = 0.3 of nmc] (ngeom) {DAGMC geom at
		$t_{mov}=0$};
        	\node [file, left = 0.3 of nmc] (ntally) {Tet mesh tally};
		\node [file, below =0.4 of nmc] (nflux) {$\phi_n$ tet mesh};
		\path [line] (biased) -- (nmc);
		\path [line] (wwinp) -- (nmc);
		\path [line] (ngeom) -- (nmc);
		\path [line] (mcinp) -- (nmc);
		\path [line] (ntally) -- (nmc);
		\path [line] (nmc) -- (nflux);
		% ALARA
		\node [script, below =0.4 of nflux] (step1) {pyne.r2s
		step1};
		\node [file, below =0.4 of step1] (alaraflux) {ALARA flux
		file};
		\node [file, left = 0.9 of alaraflux] (alarainp) {ALARA input
		file};
		\node [file, right =0.9 of alaraflux] (alaramat) {ALARA
		material library};
		\node [code, below =0.4 of alaraflux] (alara) {ALARA};
		\node [file, below =0.4 of alara] (psrc) {ALARA photon
		source};
		\node [script, below =0.4 of psrc] (step2) {pyne.r2s step2};
		\node [file, below =0.4 of step2] (psrcmesh) {$q_{\gamma,t_{dec}}$
		tet mesh};
		\path [line] (nflux) -- (step1);
		\path [line] (step1) -- (alarainp);
		\path [line] (step1) -- (alaraflux);
		\path [line] (step1) -- (alaramat);
		\path [line] (alarainp) -- (alara);
		\path [line] (alaraflux) -- (alara);
		\path [line] (alaramat) -- (alara);
		\path [line] (alara) -- (psrc);
		\path [line] (psrc) -- (step2);
		\path [line] (step2) -- (psrcmesh);
		% Update source mesh position
		\node [script, below =0.4 of psrcmesh] (trtool) {DAGMC
		transform tool};
		\node [file, left = 0.9 of psrcmesh] (tr) {Transformation
		file};
		\node [file, below =0.4 of trtool] (psrcmoved) {$q_{\gamma,t_{dec},t_{mov}}$ tet mesh};
		\path [line] (psrcmesh) -- (trtool);
		\path [line] (tr) -- (trtool);
		\path [line] (trtool) -- (psrcmoved);
		% Photon Transport
		\node [script, below = 0.4 of psrcmoved] (cadis) {cadis.py};
		\node [file, left = 0.9 of cadis] (adjpfmesh) {$\phi_{\gamma,
		t_{mov}}^{+}$ tet mesh};
		\node [file, below = 0.4 of cadis] (biasedp) {Biased
		$q_{\gamma, t_{mov}}$};
		\node [file, right = 0.9 of biasedp] (wwinp) {Weight window
		mesh};
		\node [file, right = 0.9 of psrcmoved] (geomtd) {DAGMC geom at
		$t_{mov}$};
		\node [file, left = 0.3 of biasedp] (mcpinp) {DAGMCNP input
		file};
		\node [code, below =0.4 of biasedp] (pmc) {DAGMCNP photon
		transport};
                \node [file, left=0.9 of pmc] (f2d) {Flux-to-dose};
		\node [file, below =0.4 of pmc] (sdr) {$SDR_{t_{mov}}$ tet mesh};
%		\path [line] (ngeom) -- (trtool);
		\draw [arrow] (ngeom) |-
		([xshift=0.5cm,yshift=0.5cm]alaramat.north east) |- (trtool);
		\path [line] (trtool) -- (geomtd);
		\draw [arrow] (geomtd) |-
		([xshift=0.5cm,yshift=0.5cm]wwinp.north east) |- (pmc);
%		\path [line] (geomtd) -- (pmc);
		\path [line] (mcpinp) -- (pmc);
		\path [line] (psrcmoved) -- (cadis);
		\path [line] (adjpfmesh) -- (cadis);
		\path [line] (cadis) -- (biasedp);
		\path [line] (biasedp) -- (pmc);
		\path [line] (cadis) -- (wwinp);
		\path [line] (wwinp) -- (pmc);
		\path [line] (f2d) -- (pmc);
		\path [line] (pmc) -- (sdr);

	\end{tikzpicture}

	\caption[Fully optimized, time-integrated R2S workflow]
	{Fully-optimized, time-integrated R2S workflow for calculating the SDR.  
	This particular flow used the TGT-CADIS biased source and weight windows to
	optimize the neutron transport and the CADIS method to optimize the 
	photon transport steps.
	Scripts are shown in
	blue ovals, physics codes in green ovals, and files in white
	rectangles.\label{fig:otr2s_flow}}
\end{figure}


\subsection{Error Propagation}\label{sec:error}
The total statistical error in the SDR
arises from the MC calculations of the neutron and photon flux.
\begin{equation}\label{eq:toterr}
	\sigma_{SDR}^{2} = \sigma_n^2 + \sigma_{\gamma}^2
\end{equation}
The uncertainty in the SDR due to the uncertainty in the photon transport can
be calculated during MC transport. However, the 
uncertainty in the SDR due to the uncertainty in the neutron MC
calculation is more complicated and an area of research currently under investigation by Harb et.
al \cite{harb}.  This work aims to produce a methodology for propagating the error from the
neutron transport to the photon source and then from the photon source to the
SDR.
%This work explores the degree to which the neutron flux calculated in
%each point in space are correlated.

If a full R2S simulation is carried out, the final SDR has an error associated
with the MC neutron and photon transport steps.  Introducing error from the photon transport
step can be avoided by not performing MC photon transport and calculating the
SDR using a form of Eq. \ref{eq:resp_adjf}
\begin{equation}\label{eq:resp_v}
	SDR_v = \phi_{\gamma, v}^{+} q_{\gamma, v}
\end{equation}
where $SDR_v$ is the response from a discrete volume element, v.
Recall that $\phi_{\gamma, v}$ is the result of the sum of adjoint photon flux 
over all time steps and required to form the TGT-CADIS adjoint neutron source
and $q_{\gamma, v}$ is the photon source produced by ALARA, so both quantities
are already available.

There is some relative error associated with the photon source term that
originates from the MC forward neutron calculation.  
The relative error in the
SDR, $\Re_{SDR}$ is then expressed as 
\begin{equation}\label{eq:err}
	\Re_{SDR} = \frac{1}{SDR} \sqrt{\sum_v{(SDR_v \Re_{SDR_v})^2}}
\end{equation}

This error in the SDR can then be used to calculate the neutron transport
FOM \cite{eb_prelim}
\todo{This is shown in Elliott's prelim, but not thesis.  Is this still 
an appropriate way to calculate the FOM?}
which is important to quantifying the efficiency of TGT-CADIS.
\begin{equation}\label{eq:fom}
	FOM = \frac{SDR^2}{t_{proc}\sum_v{(SDR_v \Re_{SDR_v})^2}}
\end{equation}




\subsection{Assumptions and Practical Considerations}
There are three very different time scales: photon transport,
geometry movement, and decay involved in the full SDR analysis of a moving
system.  Photon transport happens on a much
faster time scale than the geometry movement, therefore it is reasonable to
perform radiation transport on static step changes of the geometry
configuration.  It is also assumed that the
time for the photon emission density to change is much longer than the
period of geometry movement%time between the initial and final geometry configurations so the 
so the same photon source generated for a particular decay time
is used for all photon transport steps $t_{mov} = 1..N$.
%transport steps during the geometry movement. 
This assumption also allows the same T to be used at each time step of geometry movement.
%This depends on the length of time between shutdown and geometry movement.
%Will need to verify that for a certain window of time, perhaps an eight-hour
%work shift, the source does not change appreciably 
%Discuss need to explore criteria for situations that this new method will
%effectively optimize the neutron transport step.
%It is assumed that the source strength does not change during geometry
%movement, but a metric will need to be defined to be sure this is true.

In the case that the source strength is changing appreciably during geometry
movement, a  series of photon sources that capture these changes over time
will need to be used.  This will also require the calculation of different T
values for each of these time steps.
It is also assumed that the geometry is moving at a constant acceleration
and therefore the components are located at each position for an equal amount of time.  If
this was not true, the sum of adjoint photon fluxes would need to be weighted
by time. 

The exact degree to which these assumptions about time constants hold true
needs to be explored. A set of recommendations %or a metric to establish 
that state the criteria 
necessary for the TGT-CADIS method to effectively optimize the neutron
transport  will
be established.

\subsubsection{Data management}
The fully-optimized TR2S workflow involves the generation of many large
CAD geometry and volume mesh files to use for deterministic and MC transport.
The decision to store or delete and regenerate all or a subset of these data files needs to
be investigated.
%Will be generating a geometry file at each time step, 3-D flux maps for every adjoint
%		photon time step, for the adjoint neutron flux, tet mesh for
%		forward neutron flux, tet mesh of photon source. Need to run
%		adjoint photon transport for many time steps, adjoint neutron,
%		then forward neutron then ALARA.

One of the acceleration techniques employed by DAGMC is the use of Oriented
Bounding Box (OBB) trees.  This is a hierarchy of boxes that bound the facets 
of the CAD geometry.  The process of building the OBB tree can be
time-intensive for large geometries.  Because there is a DAGMC simulation for
%the neutron transport as well as a simulation of photon transport for every time 
%step of geometry movement, 
configuration of the geometry, an OBB tree will be built each time.  This could
become rather cumbersome unless the OBB tree building was more efficient.
Instead of building a new OBB tree for each geometry configuration, the OBB
tree built for the original configuration could be stored and then the geometry
transformation tool could be used to update the position of the boxes for each 
new configuration.

Depending on the level of discretization in time of the motion after shutdown,
there could potentially be a large number of adjoint photon transport
simulations to perform.  This may require the use of High Throughput Computing
sources.

\section{Demonstration} \label{sec:demo}
This section will outline the experiments proposed to demonstrate the utility
of TGT-CADIS.  

\subsection{Toy Problem}
The same geometry used in the GT-CADIS experiment shown in \ref{ch:experiment}
will be used to demonstrate the efficacy of TGT-CADIS.  The only difference
will be a modular component of the chamber located on the far side of the
detector will be moved to a location close to the detector.  

\begin{figure} \label{fig:mov_geom}
    \caption [Geometry for TGT-CADIS Demonstration]
	{Geometry to be used in TGT-CADIS Demonstration.}
\end{figure}
%\missingfigure
\todo{Missing demo figure}

First, the TR2S process without any MC VR for the neutron or photon transport
steps will be applied to this problem.  This will ultimately result in
time-integrated SDR maps.
Next, the photon transport steps will be optimized via the CADIS method.  This
will require deterministic adjoint photon simulations at each time step to
produce VR parameters for the forward MC photon transport runs.
Finally, TGT-CADIS will be applied to optimize the neutron transport step.  The
same adjoint photon flux solutions used in the last step can be used to
calculate T which is then used to calculate the adjoint neutron source for
adjoint neutron transport. The resulting adjoint neutron flux is then used to
produce the VR parameters.

These incremental additions of optimization will be fundamental in assessing the
utility of TGT-CADIS. 

%\subsection{Time-integrated Dose Map}

%\subsubsection{Analog Monte Carlo Neutron and Photon Transport}\label{sec:analog_n_p}

%\subsubsection{Variance Reduction to Optimize Photon Transport}\label{sec:vr_p}

%\subsubsection{Variance Reduction to Optimize both Neutron and Photon Transport}\label{sec:vr_n_p}


\subsection{Full-scale FES Model} \label{sec:full_scale}
As the intended purpose of this work is  
calculating the SDR during a maintenance or
operation in a FES, the final challenge problem will involve the
TGT-CADIS optimization of the neutron transport step of TR2S for a full-scale
FES.



\section{Summary}\label{sec:summary}

