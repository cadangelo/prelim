\chapter{Proposal}\label{ch:proposal}

%% This section will discuss how to adapt the MS-CADIS method to dynamic
%% systems
\section{Adapt MS-CADIS for Moving Geometries}\label{sec:adapt}
The MS-CADIS method can be used to generate an adjoint neutron source 
that can be used to generate biasing parameters that will optimize the neutron transport step of a coupled,
multi-step process. In previous studies *need ref*, this method has been shown to decrease
the variance in the neutron transport step of the R2S workflow.  These studies,
however, have only involved static systems with the same geometry used for all
transport steps.  
As seen in section *need number*, it was demonstrated that the variance
reduction parameters generated with the GT-CADIS method are insufficient for
optimizing the neutron transport step in cases that involve the movement 
of irradiated components after shutdown.  
The following section *need number* will discuss a proposed adaptation to the
MS-CADIS method that will show the derivation of a time-integrated coupling
term.

\subsection{Generalized MS-CADIS Method}
In the current literature, MS-CADIS is primarily discussed as it applies to SDR
analysis.  In actuality, MS-CADIS can be applied to any multi-step process in
which the primary radiation transport is coupled to a secondary physical process.
The addition of time integration to this methodology can therefore also be applied to
any coupled multiphysics process. For this reason, it is prudent to discuss
MS-CADIS in a more generalized manner.

Begin with the adjoint identity for the neutral particle transport
equation, 
 \begin{equation}\label{eq:adj_identity}
%		\langle \phi(\overrightarrow{r},E), q^{+}(\overrightarrow{r},E) \rangle =
%		\langle \phi^{+}(\overrightarrow{r},E) , q(\overrightarrow{r},E) \rangle
		\langle \phi, q^{+} \rangle =
		\langle \phi^{+}, q \rangle
 \end{equation}
where $\phi$ is the forward and 
$\phi^{+}$ is the adjoint particle flux, and
$q$ is the forward and 
$q^{+}$ is the adjoint source distribution.
The left-hand side of Eq. \ref{eq:adj_identity} has the same form as the
equation for detector response
if a detector response function is chosen as the adjoint source.
For a coupled, multi-step process, MS-CADIS says that the adjoint identity for
both primary and secondary processes is equal to the final response of the
system as shown below
 \begin{equation}\label{eq:adj_1}
%	 R_{final} = \langle \phi_{1}(\overrightarrow{r},E), q_{1}^{+}(\overrightarrow{r},E) \rangle =
%		\langle \phi_{1}^{+}(\overrightarrow{r},E) , q_{1}(\overrightarrow{r},E) \rangle
	 R_{final} = \langle \phi_{1}, q_{1}^{+} \rangle =
		\langle \phi_{1}^{+} , q_{1} \rangle
 \end{equation}

	 
 \begin{equation}\label{eq:adj_2}
%	 R_{final} = \langle \phi_{2}(\overrightarrow{r},E), q_{2}^{+}(\overrightarrow{r},E) \rangle =
%		\langle \phi_{2}^{+}(\overrightarrow{r},E) , q_{2}(\overrightarrow{r},E) \rangle
	 R_{final} = \langle \phi_{2}, q_{2}^{+} \rangle =
		\langle \phi_{2}^{+} , q_{2} \rangle
 \end{equation}

Subscripts 1 and 2 denote primary and secondary physical processes,
respectively.  Ultimately, the goal is to find a solution for the adjoint
transport equation of the primary physical process to use as an importance
function to generate source and transport biasing parameters. Therefore, a
source for the primary adjoint transport is needed.  Using the set of adjoint
identities in Eq. \ref{eq:adj_1} and \ref{eq:adj_2}, find the following expression

 \begin{equation}\label{eq:adj_src}
%	 \langle \phi_{1}(\overrightarrow{r},E), q_{1}^{+}(\overrightarrow{r},E) \rangle =
%	 \langle \phi_{2}^{+}(\overrightarrow{r},E) , q_{2}(\overrightarrow{r},E) \rangle
	 \langle \phi_{1}, q_{1}^{+} \rangle =
	 \langle \phi_{2}^{+}, q_{2} \rangle
 \end{equation}

Given that the sources of the forward primary and the adjoint secondary physics
are known, both solutions, $ \phi_{1}(\overrightarrow{r},E) $ and
$\phi_{2}^{+}(\overrightarrow{r},E) $ can be found through transport operations.
The source of the forward secondary and adjoint primary physics are both
unknown, therefore a second equation is needed.  This is a coupled, multi-step
system where the source of the secondary physics depends on - is a response of-
the solution of the primary physics.  This relationship can be described by the following equation

 \begin{equation}\label{eq:fwd_src}
%	 q_{2}(\overrightarrow{r},E_{2}) =
%	 \langle \sigma_{1,2}, \phi_{1}(\overrightarrow{r},E_{1}) \rangle
	 q_{2} =
	 \langle \sigma_{1\rightarrow2}, \phi_{1} \rangle
 \end{equation}

Consider the process of prompt photon production from neutron irradiation.
The response function, $sigma_{1\rightarrow2}$, is the neutron-gamma production
cross section $\sigma_{n\rightarrow\gamma}$.  For the process of delayed gamma
production, $\sigma_{1\rightarrow2}$ is a
coupling term, $T$, which has been defined by the GT-CADIS method as an approximation of the transmutation
process *need GT-CADIS source*.  As long as there is a solution for the
response function that couples the primary and secondary physics together,
there also exists a solution for the adjoint primary source as shown below.

 \begin{equation}\label{eq:adj_src_1}
%	 q_{1}^{+}(\overrightarrow{r},E_{1}) = 
%	 \langle \sigma_{1,2}, \phi_{2}^{+}(\overrightarrow{r},E_{2}) \rangle
	 q_{1}^{+} = 
	 \langle \sigma_{1\rightarrow2}, \phi_{2}^{+} \rangle
 \end{equation}



\subsection{Time-integrated Coupling Term}
If the position of the components in the geometry is changing over time during
the secondary physics, it affects the way the adjoint primary source is
constructed.  Instead of a single solution to the 
adjoint secondary physics transport, there is a 
solution at each position and each time step represented by the following term

	$ \phi_{2}^{+}(\overrightarrow{r}_{v}(t), t)$

where $\overrightarrow{r}_{v}(t)$ refers to the position of volume element, v,
at time, t.  To solve for the adjoint primary source in each volume element
$q_{1,v}^{+}$, the solutions of the 
adjoint secondary transport from each time step are combined by
integrating over time

 \begin{equation}\label{eq:adj_src_1_avg}
	 q_{1,v}^{+} =
	 \int_{t}  \phi_{2}^{+}(\overrightarrow{r}_{v}(t), t)
	 \sigma_{1\rightarrow 2,v}(t)\, dt
 \end{equation}

% This assumes that the coupling term $\sigma_{1 \rightarrow 2,v}$ does not change over time.
 
\subsection{Apply Time-integrated Coupling to GT-CADIS}

GT-CADIS is an implementation of MS-CADIS that is specific to SDR analysis.  It
provides a method to calculate a coupling term, T, that relates the neutron
flux to the photon source.
Applying time integration to the GT-CADIS methodology results in the following
solution for the adjoint neutron source

 \begin{equation}\label{eq:adj_src_1_avg}
	 q_{n,v}^{+}(E_{n}) =
	 \int_{t}     \phi_{\gamma}^{+}(\overrightarrow{r}_{v}(t), E_{\gamma},t)
	 T_{v}(E_n, E_{\gamma})\, dt
 \end{equation}

\section{Implementation: Time-integrated GT-CADIS}\label{sec:implementation}

\subsection{Optimized R2S Workflow for Dynamic Systems}
\begin{figure}\label{gen_q_n_+}
\centering

        \tikzstyle{script} = [ellipse, fill=blue!20, draw, text centered, node distance=3cm]
        \tikzstyle{code} = [ellipse, fill=green!20, draw, text centered, node distance=3cm]
        \tikzstyle{file} = [rectangle, draw, text centered, node distance=3cm]
        \tikzstyle{line} = [draw, -latex']
        
        \begin{tikzpicture}[node distance = 2 cm, auto]
        	%Generate adj gamma Partisn input
                \node [script] (partgen) {part.py};
                \node [file, left =0.9 of partgen] (pgeom) {$\gamma$ DAGMC
		geometry at t};
                \node [file, right=0.9 of partgen] (f2d) {Flux-to-dose
		converstion factors};
                \node [file, below =0.4 of partgen] (ppartinp) {Partisn input file};
                \path [line] (pgeom) -- (partgen);
                \path [line] (f2d) -- (partgen);
        	\path [line] (partgen) -- (ppartinp);
        	% Partisn adj gamma transport
                \node [code, below =0.4 of ppartinp] (partadjp) {Partisn adjoint $\gamma$ transport};
        	\node [file, below =0.4 of partadjp] (padjf) {Partisn$\phi_{\gamma, t}^{+}$ file};
        	\node [script, below =0.4 of padjf] (pf2mesh) {atflux\_to\_mesh.py};
        	\node [file, below =0.4 of pf2mesh] (padjfvmesh)		{$\phi_{\gamma,t}^{+}$ voxel mesh};
        	\node [script, below =0.4 of padjfvmesh] (conv_v_t){Convert voxel to tet};
        	\node [file, below =0.4 of conv_v_t](padjftmesh){$\phi_{\gamma,t}^{+}$ tet mesh};
		\node [script, below =0.4 of padjftmesh](sum){Sum $\phi_{\gamma,t}^{+}$over all time steps};
		\node [file, below =0.4 of sum](avgadjft){$\overline{\phi_{\gamma}^{+}}$ tet mesh};
        	\node [script, below =0.4 of avgadjft] (conv_t_v){Convert tet to voxel};
		\node [file, below =0.4 of conv_t_v](avgadjfv){$\overline{\phi_{\gamma}^{+}}$ voxel mesh};
        	\path [line] (ppartinp) -- (partadjp);
        	\path [line] (partadjp) -- (padjf);
        	\path [line] (padjf) -- (pf2mesh);
        	\path [line] (pf2mesh) -- (padjfvmesh);
		\path [line] (padjfvmesh) -- (conv_v_t);
		\path [line] (conv_v_t) -- (padjftmesh);
		\path [line] (padjftmesh) -- (sum);
		\path [line] (sum) -- (avgadjft);
		\path [line] (avgadjft) -- (conv_t_v);
		\path [line] (conv_t_v) -- (avgadjfv);
        	% Generate adj n source
        	\node [script, below =0.4 of avgadjfv] (gtcadis) {gtcadis.py}; 
		\node [file, left =0.9 of gtcadis] (ngeom) {DAGMC geometry at
		$t_{0}$};
        	\node [file, right =0.9 of gtcadis] (matlib) {ALARA material library};
        	\node [file, below =0.4 of gtcadis] (nadjs) {$q_{n}^{+}$ voxel mesh};
        	\path [line] (ngeom) -- (gtcadis);
        	\path [line] (avgadjfv) -- (gtcadis);
        	\path [line] (matlib) -- (gtcadis);
        	\path [line] (gtcadis) -- (nadjs);
        
        \end{tikzpicture}

	\caption{Workflow for generating the optimal
	adjoint neutron source via the time-integrated GT-CADIS method.  Scripts are shown in
	blue ovals, physics codes in green ovals, and files in white
	rectangles.}
\end{figure}


\begin{figure}\label{gen_q_n_+}
\centering

        \tikzstyle{script} = [ellipse, fill=blue!20, draw, text centered, node distance=3cm]
        \tikzstyle{code} = [ellipse, fill=green!20, draw, text centered, node distance=3cm]
        \tikzstyle{file} = [rectangle, draw, text centered, node distance=3cm]
        \tikzstyle{line} = [draw, -latex]
        
        \begin{tikzpicture}[node distance = 2 cm, auto]
        	%Generate adj neutron Partisn input
                \node [script] (partgen) {part.py};
		\node [file, left =0.9 of partgen] (ngeom) {DAGMC geometry at
		$t_{0}$};
                \node [file, right=0.9 of partgen] (nadjs) {$q_{n}^{+}$ voxel mesh};
                \node [file, below =0.4 of partgen] (npartinp) {Partisn input file};
                \path [line] (ngeom) -- (partgen);
                \path [line] (nadjs) -- (partgen);
        	\path [line] (partgen) -- (npartinp);
        	% Partisn adj neutron transport
                \node [code, below =0.4 of npartinp] (partadjn) {Partisn adjoint neutron transport};
        	\node [file, below =0.4 of partadjn] (nadjf) {Partisn $\phi_{n}^{+}$ file};
        	\node [script, below =0.4 of nadjf] (pf2mesh) {atflux\_to\_mesh.py};
        	\node [file, below =0.4 of pf2mesh] (nadjfmesh) {$\phi_{n}^{+}$
		voxel mesh};
        	\path [line] (npartinp) -- (partadjn);
        	\path [line] (partadjn) -- (nadjf);
        	\path [line] (nadjf) -- (pf2mesh);
        	\path [line] (pf2mesh) -- (nadjfmesh);
		%Generate biased source and WW
		\node [script, below =0.4 of nadjfmesh] (cadis) {cadis.py};
		\node [file, left =0.9 of cadis ] (unbiased) {Unbiased
		neutron source mesh};
		\node [file, below =0.4 of cadis] (biased) {Biased neutron
		source mesh};
		\node [file, right = 0.9 of biased] (wwinp) {Weight window
		mesh};
		\path [line] (unbiased) -- (cadis);
		\path [line] (nadjfmesh) -- (cadis);
		\path [line] (cadis) -- (biased);
		\path [line] (cadis) -- (wwinp);


        \end{tikzpicture}

	\caption{Workflow for generating a biased source and weight windows to
	optimize the neutron transport step.  Scripts are shown in
	blue ovals, physics codes in green ovals, and files in white
	rectangles.}
\end{figure}

\begin{figure}\label{r2s_flow}
\centering

        \tikzstyle{script} = [ellipse, fill=blue!20, draw, text centered, node distance=3cm]
        \tikzstyle{code} = [ellipse, fill=green!20, draw, text centered, node distance=3cm]
        \tikzstyle{file} = [rectangle, draw, text centered, node distance=3cm]
        \tikzstyle{line} = [draw, -latex]
        
        \begin{tikzpicture}[node distance = 2 cm, auto]
		% Optimized neutron transport
		\node [code] (nmc) {DAGMCNP neutron transport};
		\node [file, above =0.9 of nmc] (biased) {Biased neutron
		source};
		\node [file, right = 0.3 of biased] (wwinp) {Weight window
		mesh};
		%\node [code, below =0.4 of biased] (nmc) {DAGMCNP neutron
		%transport};
		\node [file, left = 0.3 of biased] (mcinp) {DAGMCNP input
		file};
        	\node [file, right = 0.3 of nmc] (ngeom) {DAGMC geometry at
		$t_0$};
		\node [file, below =0.4 of nmc] (nflux) {$\phi_n$ tetmesh file};
		\path [line] (biased) -- (nmc);
		\path [line] (wwinp) -- (nmc);
		\path [line] (ngeom) -- (nmc);
		\path [line] (mcinp) -- (nmc);
		\path [line] (nmc) -- (nflux);
		% ALARA
		\node [script, below =0.4 of nflux] (step1) {pyne.r2s
		step1};
		\node [file, below =0.4 of step1] (alaraflux) {ALARA flux
		file};
		\node [file, left = 0.9 of alaraflux] (alarainp) {ALARA input
		file};
		\node [file, right =0.9 of alaraflux] (alaramat) {ALARA
		material library};
		\node [code, below =0.4 of alaraflux] (alara) {ALARA};
		\node [file, below =0.4 of alara] (psrc) {ALARA photon
		source};
		\node [script, below =0.4 of psrc] (step2) {pyne.r2s step2};
		\node [file, below =0.4 of step2] (psrcmesh) {$q_{p,t_{d}}$
		tetmesh file};
		\path [line] (nflux) -- (step1);
		\path [line] (step1) -- (alarainp);
		\path [line] (step1) -- (alaraflux);
		\path [line] (step1) -- (alaramat);
		\path [line] (alarainp) -- (alara);
		\path [line] (alaraflux) -- (alara);
		\path [line] (alaramat) -- (alara);
		\path [line] (alara) -- (psrc);
		\path [line] (psrc) -- (step2);
		\path [line] (step2) -- (psrcmesh);
		% Update source mesh position
		\node [script, below =0.4 of psrcmesh] (trtool) {DAGMC
		transform tool};
		\node [file, left = 0.9 of psrcmesh] (tr) {Transformation
		file};
		\node [file, below =0.4 of trtool] (psrcmoved) {Updated
		$q_{p,t_{d}}$};
		\path [line] (psrcmesh) -- (trtool);
		\path [line] (tr) -- (trtool);
		\path [line] (trtool) -- (psrcmoved);
		% Photon Transport
		\node [file, right = 0.9 of psrcmoved] (geomtd) {DAGMC Geometry at
		$t_{d}$};
%		\path [line] (ngeom) -- (trtool);
		\path [line] (trtool) -- (geomtd);






        
	\end{tikzpicture}

	\caption{R2S workflow for calculating the SDR.  This particular flow
	chart shows the use of a biased source and weight windows to
	optimize the neutron transport step.
	Scripts are shown in
	blue ovals, physics codes in green ovals, and files in white
	rectangles.}
\end{figure}

\begin{itemize}
\item tet mesh n flux tally
\item tag tet mesh w transformations
\item source.h5m for each decay time
\item update position
\item run transport
\end{itemize}

\subsection{Practical Considerations}
Discuss need to explore criteria for situations that this new method will
optimize the neutron transport step.  What does this depend on?  Time constants
of geometric movement, change in source strength, photon transport.  source
strength will depend on at which point during decay the movement is occuring.
How to prove that for an 8 hr window of time the source strength doesn't change
enough to effect SDR.
\begin{itemize}
        \item efficient obb tree
	\item htc scripts to run many photon MCNP and compile results

\end{itemize}


\section{Demonstration} \label{sec:demo}


\subsection{Time-integrated Dose Map}

\subsubsection{Analog Monte Carlo Neutron and Photon Transport}\label{sec:analog_n_p}

\subsubsection{Variance Reduction to Optimize Photon Transport}\label{sec:vr_p}

\subsubsection{Variance Reduction to Optimize both Neutron and Photon Transport}\label{sec:vr_n_p}

\subsection{Error Propagation}\label{sec:error}

\subsection{Full-scale FES Model} \label{sec:full_scale}

\section{Summary}\label{sec:summary}

