\chapter{Proposal}\label{ch:proposal}

%% This section will discuss how to adapt the MS-CADIS method to dynamic
%% systems
\section{Adapt MS-CADIS for Moving Geometries}\label{sec:adapt}
As seen in section *need number*, it was demonstrated that the variance
reduction parameters generated with the GT-CADIS method are insufficient for
optimizing the neutron transport step in cases that involve the movement 
of irradiated components after shutdown.  
The following section *need number* will discuss a proposed adaptation to the
MS-CADIS method that will show the derivation of a time-integrated coupling
term.

\begin{figure}\label{gen_q_n_+}
\centering

        \tikzstyle{script} = [ellipse, fill=blue!20, draw, text centered, node distance=3cm]
        \tikzstyle{code} = [ellipse, fill=green!20, draw, text centered, node distance=3cm]
        \tikzstyle{file} = [rectangle, draw, text centered, node distance=3cm]
        \tikzstyle{line} = [draw, -latex']
        
        \begin{tikzpicture}[node distance = 2 cm, auto]
        	%Generate adj gamma Partisn input
                \node [script] (partgen) {part.py};
                \node [file, left =1.2 of partgen] (pgeom) {$\gamma$ DAGMC geometry};
                \node [file, right=1.2 of partgen] (f2d) {Flux-to-dose
		converstion factors};
                \node [file, below=1.2 of partgen] (ppartinp) {Partisn input file};
                \path [line] (pgeom) -- (partgen);
                \path [line] (f2d) -- (partgen);
        	\path [line] (partgen) -- (ppartinp);
        	% Partisn adj gamma transport
                \node [code, below =1.2 of ppartinp] (partadjp) {Partisn adjoint $\gamma$ transport};
        	\node [file, below =7.0 of pgeom] (padjf) {Partisn $\phi_{\gamma}^{+}$ file};
        	\node [script, right =1.2 of padjf] (pf2mesh) {atflux\_to\_mesh.py};
        	\node [file, right =1.2 of pf2mesh] (padjfmesh) {$\phi_{\gamma}^{+}$ mesh file};
        	\path [line] (ppartinp) -- (partadjp);
        	\path [line] (partadjp) -- (padjf);
        	\path [line] (padjf) -- (pf2mesh);
        	\path [line] (pf2mesh) -- (padjfmesh);
        	% Generate adj n source
        	\node [file, below of=padjf] (ngeom) {Neutron DAGMC geometry};
        	\node [script, right =1.2 of ngeom] (gtcadis) {gtcadis.py}; 
        	\node [file, right =1.2 of gtcadis] (matlib) {ALARA material library};
        	\node [file, below of =gtcadis] (nadjs) {$q_{n}^{+}$ mesh file};
        	\path [line] (ngeom) -- (gtcadis);
        	\path [line] (padjfmesh) -- (gtcadis);
        	\path [line] (matlib) -- (gtcadis);
        	\path [line] (gtcadis) -- (nadjs);
        
        \end{tikzpicture}

	\caption{Workflow for generating the optimal
	adjoint neutron source via the GT-CADIS method.  Scripts are shown in
	blue ovals, physics codes in green ovals, and files in white
	rectangles.}
\end{figure}


\begin{figure}\label{gen_q_n_+}
\centering

        \tikzstyle{script} = [ellipse, fill=blue!20, draw, text centered, node distance=3cm]
        \tikzstyle{code} = [ellipse, fill=green!20, draw, text centered, node distance=3cm]
        \tikzstyle{file} = [rectangle, draw, text centered, node distance=3cm]
        \tikzstyle{line} = [draw, -latex]
        
        \begin{tikzpicture}[node distance = 2 cm, auto]
        	%Generate adj neutron Partisn input
                \node [script] (partgen) {part.py};
		\node [file, left =1.2 of partgen] (ngeom) {Neutron DAGMC geometry};
                \node [file, right=1.2 of partgen] (nadjs) {$q_{n}^{+}$ mesh file};
                \node [file, below=1.2 of partgen] (npartinp) {Partisn input file};
                \path [line] (ngeom) -- (partgen);
                \path [line] (nadjs) -- (partgen);
        	\path [line] (partgen) -- (npartinp);
        	% Partisn adj neutron transport
                \node [code, below =1.2 of npartinp] (partadjn) {Partisn adjoint neutron transport};
        	\node [file, below =7.0 of pgeom] (nadjf) {Partisn $\phi_{n}^{+}$ file};
        	\node [script, right =1.2 of nadjf] (pf2mesh) {atflux\_to\_mesh.py};
        	\node [file, right =1.2 of pf2mesh] (nadjfmesh) {$\phi_{n}^{+}$ mesh file};
        	\path [line] (npartinp) -- (partadjn);
        	\path [line] (partadjn) -- (nadjf);
        	\path [line] (nadjf) -- (pf2mesh);
        	\path [line] (pf2mesh) -- (nadjfmesh);
		%Generate biased source and WW
		\node [file, below = 1.5 of nadjf] (unbiased) {Unbiased
		neutron source};
		\node [script, right = 1.2 of unbiased] (cadis) {cadis.py};
		\node [file, below = 1.2 of cadis] (biased) {Biased neutron
		source};
		\node [file, right = 1.2 of biased] (wwinp) {Weight window
		mesh};
		\path [line] (unbiased) -- (cadis);
		\path [line] (nadjfmesh) -- (cadis);
		\path [line] (cadis) -- (biased);
		\path [line] (cadis) -- (wwinp);


        \end{tikzpicture}

	\caption{Workflow for generating a biased source and weight windows to
	optimize the neutron transport step.  Scripts are shown in
	blue ovals, physics codes in green ovals, and files in white
	rectangles.}
\end{figure}
