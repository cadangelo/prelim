\chapter{Proposal}\label{ch:proposal}

%% This section will discuss how to adapt the MS-CADIS method to dynamic
%% systems
\section{Adapt MS-CADIS for Moving Geometries}\label{sec:adapt}
The MS-CADIS method can be used to generate an adjoint neutron source 
that can be used to generate biasing parameters that will optimize the neutron transport step of a coupled,
multi-step process. In previous studies *need ref*, this method has been shown to decrease
the variance in the neutron transport step of the R2S workflow.  These studies,
however, have only involved static systems with the same geometry used for all
transport steps.  
As seen in section *need number*, it was demonstrated that the variance
reduction parameters generated with the GT-CADIS method are insufficient for
optimizing the neutron transport step in cases that involve the movement 
of irradiated components after shutdown.  
The following section *need number* will discuss a proposed adaptation to the
MS-CADIS method that will show the derivation of a time-integrated coupling
term.

\subsection{Generalized MS-CADIS Method}

\subsection{Time-integrated Coupling Term}

\subsection{Apply Time-integrated Coupling to GT-CADIS}

\section{Implementation}\label{sec:implementation}

\subsection{Optimized R2S Workflow for Dynamic Systems}
\begin{figure}\label{gen_q_n_+}
\centering

        \tikzstyle{script} = [ellipse, fill=blue!20, draw, text centered, node distance=3cm]
        \tikzstyle{code} = [ellipse, fill=green!20, draw, text centered, node distance=3cm]
        \tikzstyle{file} = [rectangle, draw, text centered, node distance=3cm]
        \tikzstyle{line} = [draw, -latex']
        
        \begin{tikzpicture}[node distance = 2 cm, auto]
        	%Generate adj gamma Partisn input
                \node [script] (partgen) {part.py};
                \node [file, left =0.9 of partgen] (pgeom) {$\gamma$ DAGMC geometry};
                \node [file, right=0.9 of partgen] (f2d) {Flux-to-dose
		converstion factors};
                \node [file, below=0.9 of partgen] (ppartinp) {Partisn input file};
                \path [line] (pgeom) -- (partgen);
                \path [line] (f2d) -- (partgen);
        	\path [line] (partgen) -- (ppartinp);
        	% Partisn adj gamma transport
                \node [code, below=0.9 of ppartinp] (partadjp) {Partisn adjoint $\gamma$ transport};
        	\node [file, below =0.9 of partadjp] (padjf) {Partisn $\phi_{\gamma}^{+}$ file};
        	\node [script, below =0.9 of padjf] (pf2mesh) {atflux\_to\_mesh.py};
        	\node [file, below =0.9 of pf2mesh] (padjfmesh) {$\phi_{\gamma}^{+}$ mesh file};
        	\path [line] (ppartinp) -- (partadjp);
        	\path [line] (partadjp) -- (padjf);
        	\path [line] (padjf) -- (pf2mesh);
        	\path [line] (pf2mesh) -- (padjfmesh);
        	% Generate adj n source
        	\node [script, below =0.9 of padjfmesh] (gtcadis) {gtcadis.py}; 
        	\node [file, left =0.9 of gtcadis] (ngeom) {Neutron DAGMC geometry};
        	\node [file, right =0.9 of gtcadis] (matlib) {ALARA material library};
        	\node [file, below =0.9 of gtcadis] (nadjs) {$q_{n}^{+}$ mesh file};
        	\path [line] (ngeom) -- (gtcadis);
        	\path [line] (padjfmesh) -- (gtcadis);
        	\path [line] (matlib) -- (gtcadis);
        	\path [line] (gtcadis) -- (nadjs);
        
        \end{tikzpicture}

	\caption{Workflow for generating the optimal
	adjoint neutron source via the GT-CADIS method.  Scripts are shown in
	blue ovals, physics codes in green ovals, and files in white
	rectangles.}
\end{figure}


\begin{figure}\label{gen_q_n_+}
\centering

        \tikzstyle{script} = [ellipse, fill=blue!20, draw, text centered, node distance=3cm]
        \tikzstyle{code} = [ellipse, fill=green!20, draw, text centered, node distance=3cm]
        \tikzstyle{file} = [rectangle, draw, text centered, node distance=3cm]
        \tikzstyle{line} = [draw, -latex]
        
        \begin{tikzpicture}[node distance = 2 cm, auto]
        	%Generate adj neutron Partisn input
                \node [script] (partgen) {part.py};
		\node [file, left =0.9 of partgen] (ngeom) {Neutron DAGMC geometry};
                \node [file, right=0.9 of partgen] (nadjs) {$q_{n}^{+}$ mesh file};
                \node [file, below=0.9 of partgen] (npartinp) {Partisn input file};
                \path [line] (ngeom) -- (partgen);
                \path [line] (nadjs) -- (partgen);
        	\path [line] (partgen) -- (npartinp);
        	% Partisn adj neutron transport
                \node [code, below=0.9 of npartinp] (partadjn) {Partisn adjoint neutron transport};
        	\node [file, below =0.9 of partadjn] (nadjf) {Partisn $\phi_{n}^{+}$ file};
        	\node [script, below =0.9 of nadjf] (pf2mesh) {atflux\_to\_mesh.py};
        	\node [file, below =0.9 of pf2mesh] (nadjfmesh) {$\phi_{n}^{+}$ mesh file};
        	\path [line] (npartinp) -- (partadjn);
        	\path [line] (partadjn) -- (nadjf);
        	\path [line] (nadjf) -- (pf2mesh);
        	\path [line] (pf2mesh) -- (nadjfmesh);
		%Generate biased source and WW
		\node [script, below = 0.9 of nadjfmesh] (cadis) {cadis.py};
		\node [file, left =0.9 of cadis ] (unbiased) {Unbiased
		neutron source};
		\node [file, below =0.9 of cadis] (biased) {Biased neutron
		source};
		\node [file, right = 0.9 of biased] (wwinp) {Weight window
		mesh};
		\path [line] (unbiased) -- (cadis);
		\path [line] (nadjfmesh) -- (cadis);
		\path [line] (cadis) -- (biased);
		\path [line] (cadis) -- (wwinp);


        \end{tikzpicture}

	\caption{Workflow for generating a biased source and weight windows to
	optimize the neutron transport step.  Scripts are shown in
	blue ovals, physics codes in green ovals, and files in white
	rectangles.}
\end{figure}

\begin{figure}\label{r2s_flow}
\centering

        \tikzstyle{script} = [ellipse, fill=blue!20, draw, text centered, node distance=3cm]
        \tikzstyle{code} = [ellipse, fill=green!20, draw, text centered, node distance=3cm]
        \tikzstyle{file} = [rectangle, draw, text centered, node distance=3cm]
        \tikzstyle{line} = [draw, -latex]
        
        \begin{tikzpicture}[node distance = 2 cm, auto]
		% Optimized neutron transport
		\node [code] (nmc) {DAGMCNP neutron transport};
		\node [file, above =0.9 of nmc] (biased) {Biased neutron
		source};
		\node [file, right = 0.3 of biased] (wwinp) {Weight window
		mesh};
		%\node [code, below=0.9 of biased] (nmc) {DAGMCNP neutron
		%transport};
		\node [file, left = 0.3 of biased] (mcinp) {DAGMCNP input
		file};
        	\node [file, right = 0.3 of nmc] (ngeom) {DAGMC geometry at
		$t_0$};
		\node [file, below=0.9 of nmc] (nflux) {$\phi_n$ tetmesh file};
		\path [line] (biased) -- (nmc);
		\path [line] (wwinp) -- (nmc);
		\path [line] (ngeom) -- (nmc);
		\path [line] (mcinp) -- (nmc);
		\path [line] (nmc) -- (nflux);
		% ALARA
		\node [script, below=0.9 of nflux] (step1) {pyne.r2s
		step1};
		\node [file, below=0.9 of step1] (alaraflux) {ALARA flux
		file};
		\node [file, left = 0.9 of alaraflux] (alarainp) {ALARA input
		file};
		\node [file, right =0.9 of alaraflux] (alaramat) {ALARA
		material library};
		\node [code, below=0.9 of alaraflux] (alara) {ALARA};
		\node [file, below=0.9 of alara] (psrc) {ALARA photon
		source};
		\node [script, below=0.9 of psrc] (step2) {pyne.r2s step2};
		\node [file, below=0.9 of step2] (psrcmesh) {$q_{p,t_{d}}$
		tetmesh file};
		\path [line] (nflux) -- (step1);
		\path [line] (step1) -- (alarainp);
		\path [line] (step1) -- (alaraflux);
		\path [line] (step1) -- (alaramat);
		\path [line] (alarainp) -- (alara);
		\path [line] (alaraflux) -- (alara);
		\path [line] (alaramat) -- (alara);
		\path [line] (alara) -- (psrc);
		\path [line] (psrc) -- (step2);
		\path [line] (step2) -- (psrcmesh);
		% Update source mesh position
		\node [script, below = 0.9 of psrcmesh] (trtool) {DAGMC
		transform tool};
		\node [file, left = 0.9 of psrcmesh] (tr) {Transformation
		file};
		\node [file, below =0.9 of trtool] (psrcmoved) {Updated
		$q_{p,t_{d}}$};
		\path [line] (psrcmesh) -- (trtool);
		\path [line] (tr) -- (trtool);
		\path [line] (trtool) -- (psrcmoved);
		% Photon Transport
		\node [file, right = 0.9 of psrcmoved] (geomtd) {DAGMC Geometry at
		$t_{d}$};
%		\path [line] (ngeom) -- (trtool);
		\path [line] (trtool) -- (geomtd);






        
	\end{tikzpicture}

	\caption{R2S workflow for calculating the SDR.  This particular flow
	chart shows the use of a biased source and weight windows to
	optimize the neutron transport step.
	Scripts are shown in
	blue ovals, physics codes in green ovals, and files in white
	rectangles.}
\end{figure}

tet mesh n flux tally
-tag tet mesh w transformations
source.h5m for each decay time
update position
run transport

\subsection{Practical Considerations}
Discuss need to explore criteria for situations that this new method will
optimize the neutron transport step.  What does this depend on?  Time constants
of geometric movement, change in source strength, photon transport.  source
strength will depend on at which point during decay the movement is occuring.
How to prove that for an 8 hr window of time the source strength doesn't change
enough to effect SDR.
\section{Demonstration} \label{sec:demo}


\subsection{Time-integrated Dose Map}

\subsubsection{Analog Monte Carlo Neutron and Photon Transport}\label{sec:analog_n_p}

\subsubsection{Variance Reduction to Optimize Photon Transport}\label{sec:vr_p}

\subsubsection{Variance Reduction to Optimize both Neutron and Photon Transport}\label{sec:vr_n_p}

\subsection{Error Propagation}\label{sec:error}

\subsection{Full-scale FES Model} \label{sec:full_scale}

\subsection{Summary}\label{sec:summary}

