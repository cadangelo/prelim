\chapter{Proposal}\label{ch:proposal}

As demonstrated in the experiment in chapter \ref{ch:experiment},  the variance
reduction parameters generated with the GT-CADIS method are insufficient for
optimizing the neutron transport step of SDR analysis in cases that involve the movement 
of activated components after shutdown.  
Finally, a plan to generate fully optimized, time-integrated SDR maps will be
proposed.



\section{Implementation: Time-integrated GT-CADIS}\label{sec:implementation}

\subsection{Optimized R2S Workflow for Dynamic Systems}
This section will walk through the implementation of a
time-integrated R2S workflow with optimized neutron transport via
Time-integrated Groupwise Transmutation (TGT)-CADIS. 
First, the steps taken to generate an optimal adjoint neutron source are
described and schematic is given in Fig. \ref{fig:gen_q_n_+}.

Need to make the distinction between different times.  First, there is the
decay time, $t_{dec}$ which is the amount of time that the components have been
decaying fter shutdown.  There is $t_{mov}$ which will be used to denote the
discrete time step during the geometry movement.

\subsubsection{Generation of the Adjoint Neutron Source}
A CAD model of the geometry in its original position, that during device
operation time, is created.  This geometry is tagged with transformation
numbers corresponding to the stepwise motion vectors that are applied to move the
geometry components from their original to final locations.  The DAGMC
transformation tool is used to apply the motion vectors and create HDF5 mesh
files of the geometry at each time step.   
The geometry file at each time step along with the adjoint photon source, the
Flux-to-dose conversion factors, are given as input into a script that
generates a Partisn input file.  Deterministic adjoint photon transport is
carried out and a Partisn output of the resulting adjoint photon flux is
returned.  This Partisn output is converted to a an HDF5 structured mesh file
through a PyNE conversion method.  
Because we ultimately need to sum the contribution from each time step in each
volume element, the adjoint photon flux voxel mesh is mapped onto the
tetrahedral mesh of the geometry at that same time step.  Each tet
mesh element has an ID associated with it.  Those IDs are constant across all
geometry files.  This allows the contribution of the adjoint flux from each
time step to be summed in each tet mesh element.

%\todo create figure showing mapping of voxel to tet mesh
The time integration is approximated by the following discrete sum over all
time steps

\begin{equation}\label{eq:sum}
	\phi_{\gamma,v, h}^{+} = \sum_{t_{mov}}{\phi_{\gamma,v,h,t_{mov}}}
\end{equation}

The adjoint photon flux tetrahedral mesh is converted back to a voxel mesh
to use as input for the PyNE gtcadis.py script. This script ultimately generates  a
discrete adjoint neutron source for deterministic transport.
If the SNILB criteria are met, the coupling term $T_{g,h}$ is found with Eq.
\ref{eq:T}

\begin{equation}\label{eq:T}
	T_{g,h} = \dfrac{q_{\gamma, h}(\phi_{n,g})}{\phi_{n,g}}
\end{equation}
To obtain the source of photons in each photon energy group, h, 
single pulse irradiations are performed with ALARA.
Each material in the problem is irradiated with a single energy group of
neutrons, g, and allowed to decay to the time of interest, $t_dec$.  The value
of $T_{v,g,h}$ is assigned by finding the material in each voxel, v.
This also assumes that T does not change during the time the geometry is
moving, $t_{mov}$.  

The TGT-CADIS adjoint neutron source can be found via Eq.
\ref{eq:tgt_n_src}
\begin{equation}\label{eq:tgt_n_src}
	q_{n,v,g}^{+} =
	\sum_{t_{mov}}\Bigg(\sum_{h} T_{v,g,h} \phi_{\gamma,v,h,t_{mov}}^{+}\Bigg)
\end{equation}



\begin{figure}\label{fig:gen_q_n_+}
\centering

        \tikzstyle{script} = [ellipse, fill=blue!20, draw, text centered, node distance=3cm]
        \tikzstyle{code} = [ellipse, fill=green!20, draw, text centered, node distance=3cm]
        \tikzstyle{file} = [rectangle, draw, text centered, node distance=3cm]
        \tikzstyle{line} = [draw, -latex']
        
        \begin{tikzpicture}[node distance = 2 cm, auto]
        	%Generate adj gamma Partisn input
                \node [file] (pgeom_0) {DAGMC geom at $t_{mov}=0$};
		\node [script, below = 0.4 of pgeom_0] (dagtrans) {DAGMC
		transform tool};
                \node [script, below = 0.4 of dagtrans] (partgen) {part.py};
                \node [file, left = 0.9 of partgen] (pgeom_i) {DAGMC geom
		at $t_{mov}$};
                \node [file, right=0.9 of partgen] (f2d) {Flux-to-dose conversion factors};
                \node [file, below =0.4 of partgen] (ppartinp) {Partisn input file};
                \path [line] (pgeom_0) -- (dagtrans);
		\path [line] (dagtrans) -- (pgeom_i);
                \path [line] (pgeom_i) -- (partgen);
                \path [line] (f2d) -- (partgen);
        	\path [line] (partgen) -- (ppartinp);
        	% Partisn adj gamma transport
                \node [code, below =0.4 of ppartinp] (partadjp) {Partisn adjoint $\gamma$ transport};
        	\node [file, below =0.4 of partadjp] (padjf)
		{Partisn$\phi_{\gamma, t_{mov}}^{+}$ file};
        	\node [script, below =0.4 of padjf] (pf2mesh) {atflux\_to\_mesh.py};
        	\node [file, below =0.4 of pf2mesh] (padjfvmesh)
		{$\phi_{\gamma,t_{mov}}^{+}$ voxel mesh};
        	\node [script, below =0.4 of padjfvmesh] (conv_v_t){Convert voxel to tet};
        	\node [file, below =0.4 of
		conv_v_t](padjftmesh){$\phi_{\gamma,t_{mov}}^{+}$ tet mesh};
		\node [script, below =0.4 of padjftmesh](sum){Sum
		$\phi_{\gamma,t_{mov}}^{+}$over all time steps};
		\node [file, below =0.4 of sum](avgadjft){$\phi_{\gamma}^{+}$ tet mesh};
        	\node [script, below =0.4 of avgadjft] (conv_t_v){Convert tet to voxel};
		\node [file, below =0.4 of conv_t_v](avgadjfv){${\phi_{\gamma}^{+}}$ voxel mesh};
        	\path [line] (ppartinp) -- (partadjp);
        	\path [line] (partadjp) -- (padjf);
        	\path [line] (padjf) -- (pf2mesh);
        	\path [line] (pf2mesh) -- (padjfvmesh);
		\path [line] (padjfvmesh) -- (conv_v_t);
		\path [line] (conv_v_t) -- (padjftmesh);
		\path [line] (padjftmesh) -- (sum);
		\path [line] (sum) -- (avgadjft);
		\path [line] (avgadjft) -- (conv_t_v);
		\path [line] (conv_t_v) -- (avgadjfv);
        	% Generate adj n source
        	\node [script, below =0.4 of avgadjfv] (gtcadis) {gtcadis.py}; 
		\node [file, left =0.9 of gtcadis] (ngeom) {DAGMC geom at
		$t_{0}$};
        	\node [file, right =0.9 of gtcadis] (matlib) {ALARA material library};
        	\node [file, below =0.4 of gtcadis] (nadjs) {$q_{n}^{+}$ voxel mesh};
        	\path [line] (ngeom) -- (gtcadis);
        	\path [line] (avgadjfv) -- (gtcadis);
        	\path [line] (matlib) -- (gtcadis);
        	\path [line] (gtcadis) -- (nadjs);
        
        \end{tikzpicture}

	\caption{Workflow for generating the optimal
	adjoint neutron source via the time-integrated GT-CADIS method.  Scripts are shown in
	blue ovals, physics codes in green ovals, and files in white
	rectangles.}
\end{figure}

Once the TGT-CADIS adjoint neutron source has been calculated, deterministic
adjoint transport is carried out and an adjoint neutron flux Partisn file is
returned.  This file is then converted to a voxel mesh.
This functions as an importance map for the forward neutron transport.  Areas
that have a high adjoint neutron flux will have high importance and those with
low flux will have low importance.  This map will reflect the movement of
geometry during the decay period.  This adjoint neutron flux mesh is used to
generate a biased neutron source and a weight window mesh via the CADIS method.


\begin{figure}\label{gen_biased}
\centering

        \tikzstyle{script} = [ellipse, fill=blue!20, draw, text centered, node distance=3cm]
        \tikzstyle{code} = [ellipse, fill=green!20, draw, text centered, node distance=3cm]
        \tikzstyle{file} = [rectangle, draw, text centered, node distance=3cm]
        \tikzstyle{line} = [draw, -latex]
        
        \begin{tikzpicture}[node distance = 2 cm, auto]
        	%Generate adj neutron Partisn input
                \node [script] (partgen) {part.py};
		\node [file, left =0.9 of partgen] (ngeom) {DAGMC geom at
		$t_{mov}=0$};
                \node [file, right=0.9 of partgen] (nadjs) {$q_{n}^{+}$ voxel mesh};
                \node [file, below =0.4 of partgen] (npartinp) {Partisn input file};
                \path [line] (ngeom) -- (partgen);
                \path [line] (nadjs) -- (partgen);
        	\path [line] (partgen) -- (npartinp);
        	% Partisn adj neutron transport
                \node [code, below =0.4 of npartinp] (partadjn) {Partisn adjoint neutron transport};
        	\node [file, below =0.4 of partadjn] (nadjf) {Partisn $\phi_{n}^{+}$ file};
        	\node [script, below =0.4 of nadjf] (pf2mesh) {atflux\_to\_mesh.py};
        	\node [file, below =0.4 of pf2mesh] (nadjfmesh) {$\phi_{n}^{+}$
		voxel mesh};
        	\path [line] (npartinp) -- (partadjn);
        	\path [line] (partadjn) -- (nadjf);
        	\path [line] (nadjf) -- (pf2mesh);
        	\path [line] (pf2mesh) -- (nadjfmesh);
		%Generate biased source and WW
		\node [script, below =0.4 of nadjfmesh] (cadis) {cadis.py};
		\node [file, left =0.9 of cadis ] (unbiased) {Unbiased	$q_n$ mesh};
		\node [file, below =0.4 of cadis] (biased) {Biased $q_n$ mesh};
		\node [file, right = 0.9 of biased] (wwinp) {Weight window
		mesh};
		\path [line] (unbiased) -- (cadis);
		\path [line] (nadjfmesh) -- (cadis);
		\path [line] (cadis) -- (biased);
		\path [line] (cadis) -- (wwinp);


        \end{tikzpicture}

	\caption{Workflow for generating a biased source and weight windows to
	optimize the neutron transport step.  Scripts are shown in
	blue ovals, physics codes in green ovals, and files in white
	rectangles.}
\end{figure}

\subsection{Optimized, Time-integrated R2S Workflow}

A few adjustments need to be made to the R2S workflow to produce time-integrated
SDR maps.  First, the TGT-CADIS biased source and weight windows are used to
optimize the forward neutron transport.  A tetrahedral mesh that conforms to
the geometry position at $t_{mov}=0$ is used as a tally to score the neutron flux.
This mesh is tagged with the geometry transformations that will occur after
shutdown.  The tetrahedral mesh neutron flux tally along with an irradiation and decay
scenario of interest are given as input to the PyNE R2S
script to generate ALARA input files.  ALARA is run and produces photon source
files for each decay time of interest.  These ALARA photon source files are
converted to tetrahedral mesh based sources by another PyNE R2S script.  All
source mesh files generated will reflect the original position of the geometry.
Both the source mesh and DAGMC geometry files are transformed to the 
correct locations for each $t_{mov}$ with the DAGMC transform tool.
The source mesh file in its updated position along with the previously
generated adjoint photon flux mesh are used to generate a biased photon source 
and weight window mesh are generated via the CADIS method.  At this point,
there is a biased photon source mesh, a weight window mesh, and a DAGMC
geometry for every time step $t_{mov}$.  These are all used along with a DAGMC
input file that has a photon flux tally modified by flux-to-dose conversion
factors as input for the final DAGMCNP photon transport step.  This results in
a SDR mesh for each $t_{mov}$.  

%Each of the flux values can be multiplied by
%the interval of time spent at each position and then summed to 
 

\begin{figure}\label{r2s_flow}
\centering

        \tikzstyle{script} = [ellipse, fill=blue!20, draw, text centered, node distance=3cm]
        \tikzstyle{code} = [ellipse, fill=green!20, draw, text centered, node distance=3cm]
        \tikzstyle{file} = [rectangle, draw, text centered, node distance=3cm]
        \tikzstyle{line} = [draw, -latex]
        \tikzstyle{arrow} = [thick, ->, >=stealth]
        
        \begin{tikzpicture}[node distance = 2 cm, auto]
		% Optimized neutron transport
		\node [code] (nmc) {DAGMCNP neutron transport};
		\node [file, above =0.9 of nmc] (biased) {Biased neutron
		source};
		\node [file, right = 0.3 of biased] (wwinp) {Weight window
		mesh};
		\node [file, left = 0.3 of biased] (mcinp) {DAGMCNP input
		file};
        	\node [file, right = 0.3 of nmc] (ngeom) {DAGMC geom at
		$t_{mov}=0$};
        	\node [file, left = 0.3 of nmc] (ntally) {Tet mesh tally};
		\node [file, below =0.4 of nmc] (nflux) {$\phi_n$ tet mesh};
		\path [line] (biased) -- (nmc);
		\path [line] (wwinp) -- (nmc);
		\path [line] (ngeom) -- (nmc);
		\path [line] (mcinp) -- (nmc);
		\path [line] (ntally) -- (nmc);
		\path [line] (nmc) -- (nflux);
		% ALARA
		\node [script, below =0.4 of nflux] (step1) {pyne.r2s
		step1};
		\node [file, below =0.4 of step1] (alaraflux) {ALARA flux
		file};
		\node [file, left = 0.9 of alaraflux] (alarainp) {ALARA input
		file};
		\node [file, right =0.9 of alaraflux] (alaramat) {ALARA
		material library};
		\node [code, below =0.4 of alaraflux] (alara) {ALARA};
		\node [file, below =0.4 of alara] (psrc) {ALARA photon
		source};
		\node [script, below =0.4 of psrc] (step2) {pyne.r2s step2};
		\node [file, below =0.4 of step2] (psrcmesh) {$q_{\gamma,t_{dec}}$
		tet mesh};
		\path [line] (nflux) -- (step1);
		\path [line] (step1) -- (alarainp);
		\path [line] (step1) -- (alaraflux);
		\path [line] (step1) -- (alaramat);
		\path [line] (alarainp) -- (alara);
		\path [line] (alaraflux) -- (alara);
		\path [line] (alaramat) -- (alara);
		\path [line] (alara) -- (psrc);
		\path [line] (psrc) -- (step2);
		\path [line] (step2) -- (psrcmesh);
		% Update source mesh position
		\node [script, below =0.4 of psrcmesh] (trtool) {DAGMC
		transform tool};
		\node [file, left = 0.9 of psrcmesh] (tr) {Transformation
		file};
		\node [file, below =0.4 of trtool] (psrcmoved) {$q_{\gamma,t_{dec},t_{mov}}$ tet mesh};
		\path [line] (psrcmesh) -- (trtool);
		\path [line] (tr) -- (trtool);
		\path [line] (trtool) -- (psrcmoved);
		% Photon Transport
		\node [script, below = 0.4 of psrcmoved] (cadis) {cadis.py};
		\node [file, left = 0.9 of cadis] (adjpfmesh) {$\phi_{\gamma,
		t_{mov}}^{+}$ tet mesh};
		\node [file, below = 0.4 of cadis] (biasedp) {Biased
		$q_{\gamma, t_{mov}}$};
		\node [file, right = 0.9 of biasedp] (wwinp) {Weight window
		mesh};
		\node [file, right = 0.9 of psrcmoved] (geomtd) {DAGMC geom at
		$t_{mov}$};
		\node [file, left = 0.3 of biasedp] (mcpinp) {DAGMCNP input
		file};
		\node [code, below =0.4 of biasedp] (pmc) {DAGMCNP photon
		transport};
                \node [file, left=0.9 of pmc] (f2d) {Flux-to-dose};
		\node [file, below =0.4 of pmc] (sdr) {$SDR_{t_{mov}}$ tet mesh};
%		\path [line] (ngeom) -- (trtool);
		\draw [arrow] (ngeom) |-
		([xshift=0.5cm,yshift=0.5cm]alaramat.north east) |- (trtool);
		\path [line] (trtool) -- (geomtd);
		\draw [arrow] (geomtd) |-
		([xshift=0.5cm,yshift=0.5cm]wwinp.north east) |- (pmc);
%		\path [line] (geomtd) -- (pmc);
		\path [line] (mcpinp) -- (pmc);
		\path [line] (psrcmoved) -- (cadis);
		\path [line] (adjpfmesh) -- (cadis);
		\path [line] (cadis) -- (biasedp);
		\path [line] (biasedp) -- (pmc);
		\path [line] (cadis) -- (wwinp);
		\path [line] (wwinp) -- (pmc);
		\path [line] (f2d) -- (pmc);
		\path [line] (pmc) -- (sdr);

	\end{tikzpicture}

	\caption{Time-integrated R2S workflow for calculating the SDR.  This particular flow
	chart shows the use of a biased source and weight windows to
	optimize both neutron and photon transport steps.
	Scripts are shown in
	blue ovals, physics codes in green ovals, and files in white
	rectangles.}
\end{figure}

\begin{itemize}
\item tet mesh n flux tally
\item tag tet mesh w transformations
\item source.h5m for each decay time
\item update position
\item run transport
\end{itemize}

\subsection{Practical Considerations}
Discuss need to explore criteria for situations that this new method will
optimize the neutron transport step.  What does this depend on?  Time constants
of geometric movement, change in source strength, photon transport.  source
strength will depend on at which point during decay the movement is occuring.
How to prove that for an 8 hr window of time the source strength doesn't change
enough to effect SDR.
\begin{itemize}
		\item list of assumptions: all time steps are of equal length;
			if they weren't, would need a time-weighted average.
			Assume large difference in time constants of photon
			transport, geometry movement, and photon source decay
        \item efficient obb tree
	\item htc scripts to run many photon MCNP and compile results

\end{itemize}


\section{Demonstration} \label{sec:demo}
This section will outline the experiments proposed to demonstrate the utility
of TGT-CADIS.  

\subsection{Toy Problem}
The same geometry used in the GT-CADIS experiment shown in \ref{ch:experiment}
will be used to demonstrate the efficacy of TGT-CADIS.  The only difference
will be a modular component of the chamber located on the far side of the
detector will be moved to a location close to the detector.  

First, the TR2S process without any MC VR for the neutron or photon transport
steps will be applied to this problem.  This will ultimately result in a
time-integrated dose map.
Next, the photon transport steps will be optimized via the CADIS method.  This
will require deterministic adjoint photon simulations at each time step to
produce VR parameters for the forward MC photon transport runs.
Finally, TGT-CADIS will be applied to optimize the neutron transport step.  The
same adjoint photon flux solutions used in the last step can be used to
calculate T which is then used to calculate the adjoint neutron source for
adjoint neutron transport which results in the adjoint neutron flux used to
produce the VR parameters.

These incremental additions of optimization will be fundamental in assessing the
utility of TGT-CADIS. 

%\subsection{Time-integrated Dose Map}

%\subsubsection{Analog Monte Carlo Neutron and Photon Transport}\label{sec:analog_n_p}

%\subsubsection{Variance Reduction to Optimize Photon Transport}\label{sec:vr_p}

%\subsubsection{Variance Reduction to Optimize both Neutron and Photon Transport}\label{sec:vr_n_p}

\subsection{Error Propagation}\label{sec:error}
Error Propagation in SDR analysis is a topic currently being investigated.
If a full TR2S simulation was carried out, the final SDR would have an error associated
with the MC neutron photon transport steps.  The error from the photon transport
step can be avoided by not performing MC photon transport and calculating the
SDR using Eq. \ref{eq:} *need eqn from background*.


\subsection{Full-scale FES Model} \label{sec:full_scale}
As the intended purpose of this work is  
calculating the SDR during a maintenance or
intervention activity in a FES, the final challenge problem will involve 
TGT-CADIS optimization of the neutron transport step of TR2S for a full-scale
FES.

\section{Summary}\label{sec:summary}

