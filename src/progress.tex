\chapter{Progress} \label{ch:progress}

\section{DAGMC Simulations with Geometry Transformations}\label{sec:dagmc_trans}
%% This section will discuss the progress that has been made toward the moving
%% geometry capability (DAGMC tool and ability to read from MCNP input)

\section{Demonstration of GT-CADIS} \label{sec:gtcadis}
%% This section will show the GT-CADIS experiment and discuss why this method
%% is insufficient for dynamic systems
The GT-CADIS method has proven to be an effective form of VR for calculating the
SDR in static FES where the SNILB criteria are met.  As it stands, this method
will not provide appropriate VR parameters for the cases where there is movement
after shutdown.  The follow experiment will demonstrate the need for a
time-integrated coupling term in order to provide useful VR parameters for
dynamic systems.
\subsection{Problem Description} \label{sec:description}
The geometry chosen is a simplified representation of a fusion energy device.
It is composed of a chamber of stainless steel with a central cavity measuring
2m x 2m x 2 m.  The walls are 2 m thick.  The chamber is surrounded by air and
there is helium in the central cavity.  A uniform, volumetric source of 14 MeV
neutrons was placed in the central cavity.  First, the R2S workflow was
performed in analog and then the GT-CADIS method was used to generate VR parameters
to optimize the neutron transport step of R2S.

\begin{figure}\label{fig:geom}
  \caption Cutaway view of the geometry.  
\end{figure}

\subsection{Results}
\begin{figure} \label{fig:nflux}
  \caption Neutron flux and relative error resulting from analog MC simulation.
\end{figure}

\begin{figure} \label{fig:pflux}
  \caption Photon flux and relative error resulting from analog MC simulation.
\end{figure}

\begin{figure}\label{fig:adj_p_flux}
  \caption Adjoint photon flux used to generate adjoint neutron source.  
\end{figure}

\begin{figure} \label{fig:biased_src}
  \caption Biased source generated with GT-CADIS method.
\end{figure}

\begin{figure} \label{fig:wwinp}
  \caption Weight window mesh generated with GT-CADIS method.
\end{figure}

\begin{figure} \label{fig:gt_nflux}
  \caption Neutron flux and relative error resulting from MC simulation using
  GT-CADIS biased source and weight window mesh.
\end{figure}

\begin{figure} \label{fig:gt_pflux}
  \caption Photon flux and relative error resulting from MC simulation using
  GT-CADIS biased source and weight window mesh.
\end{figure}


\subsection{Discussion}

