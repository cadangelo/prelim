\chapter{Progress} \label{ch:progress}

\section{DAGMC Simulations with Geometry Transformations}\label{sec:dagmc_trans}
%% This section will discuss the progress that has been made toward the moving
%% geometry capability (DAGMC tool and ability to read from MCNP input)

\subsection{Production of Stepwise Geometry Files}\label{sec:timestep_geoms}
A tool has be developed to generate CAD geometry files that capture the 
movement of components over time.  First, a geometry file of the model in its 
original position it loaded.  The geometry components are tagged with 
transformation numbers that correspond to stepwise transformation vectors
 that are also given as input to this tool.  The position of the components 
is updated according to these transformations and a new geometry file is 
produced for each time step.  This tool only handles rigid-body transformations
 and no geometric deformations or scaling.  It also does not handle kinetics, 
so the user must be cautious to not cause any overlap of components during the
 motion.  As an example, consider an activated component of a fusion device 
that needs to be moved around the facility during a maintenance procedure.  
The movement of the component would be captured in a stepwise manner and each of
 the new geometry files generated will be used as the input geometry for the
 photon transport calculations to determine the SDR at each time step

\subsection{DAGMCNP Geometry Transformations}\label{sec:mcnp_tr}
The ability to read transformation (TRn) cards from the MCNP input file and 
update the position of the geometry has also been added to DAGMC.  This
 capability relies upon the same tagging of components in the CAD geometry 
file as previously discussed.  The transformation will be applied to any 
geometry component that is tagged with the number on the TR card.  If using 
this capability to perform stepwise radiation transport calculations over time, 
the user would create one MCNP input file per time step.  Each input file 
should contain the TR cards necessary to update the position of each component 
to its new location for that time step.  During a DAGMCNP simulation, the 
geometry is loaded, the MCNP input file is read, the geometry position is 
updated and then transport is performed.


