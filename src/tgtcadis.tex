\myexternaldocument{prelim}
%% This section will discuss how to adapt the MS-CADIS method to dynamic
%% systems
%\chapter{Adapt MS-CADIS for Moving Geometries}\label{ch:adapt}
\chapter{Variance Reduction for Time-integrated Multi-physics Analysis}\label{ch:tgt}

The MS-CADIS method of variance reduction 
was developed to optimize the primary radiation transport
%is used to formulate an adjoint neutron source
%that represents the importance of the neutrons to the final quantity of interest
%that captures both the potential of regions to become activated and produce
%decay photons that contribute to the SDR at the detector location.
%The solution resulting from the adjoint neutron transport with this source
%is used to generate biasing parameters that will optimize the neutron 
%transport step of
in a coupled, multi-step process. The first implementation of this method 
was applied to the coupled neutron activation-photon decay process that occurs
in FES.  
%Specifically, it
%was shown to decrease
%the variance in the neutron transport step of the R2S workflow
%was used to optimize the neutron transport step in the 
%SDR analysis of the 
%ITER SDR benchmark experiment \cite{mscadis}.
In its current form, MS-CADIS is only applicable to static systems where
the geometry remains unchanged in all steps of the multi-step process.

This chapter will first discuss MS-CADIS outside of the context of SDR analysis. 
Next, a time-integrated solution to the adjoint of the 
physical process occurring during geometry movement will be derived.  
Finally, this time-integrated solution will be applied to the GT-CADIS method to
form the Time-integrated (T)GT-CADIS adjoint neutron source that will
ultimately be used to optimize the neutron transport step of SDR analysis.

\section{Generalized MS-CADIS Method}\label{sec:gen_mscadis}
In the current literature, MS-CADIS is primarily discussed as it applies to SDR
analysis \cite{mscadis}.  In actuality, MS-CADIS has always been intended to
apply to any multi-step process in
which the primary radiation transport is coupled to a secondary physical process.
The addition of time integration to this methodology can also be applied to
any coupled, multi-physics process. For this reason, it is prudent to discuss
MS-CADIS in a more generalized manner.

To describe the system of coupled, multi-physics,
the operator notation of the Boltzmann transport equation
\begin{equation}
	H\phi = q
\end{equation}
where H operates on the particle flux $\phi$ and q is a source of particles, 
will be used to represent the initial radiation transport.
An equation of the same form 
\begin{equation}
	L\Psi = b
\end{equation}
where L operates
%\todo{requirements for operator?-- not exactly sure what this means}
on some function $\Psi$ and b is a source term, will be used
to describe a generic secondary physics.
Because this is a coupled system, the source of secondary physics, is a function
of the primary particle flux, $b = f(\phi)$.
%, $b(\phi)$.

The adjoint identity for the neutral particle transport
equation was given in Eq. \ref{eq:3.2}.
% \begin{equation}\label{eq:adj_identity}
%		\langle \phi(\overrightarrow{r},E), q^{+}(\overrightarrow{r},E) \rangle =
%		\langle \phi^{+}(\overrightarrow{r},E) , q(\overrightarrow{r},E) \rangle
%		\langle q^{+}, H\phi \rangle =
%		\langle H^{+}\phi^{+}, q \rangle
% \end{equation}
%where $\phi$ is the forward and 
%$\phi^{+}$ is the adjoint particle flux, and
%$q$ is the forward and 
%$q^{+}$ is the adjoint source distribution and 
%$\langle \cdot \rangle$ signifies the
%integration over all dependent variables.
This identity is valid for an arbitrary
adjoint source function
\cite{l_m}, therefore the secondary physics has an adjoint identity of the same
form
\begin{center}
{$\langle \Psi^{+}, L\Psi \rangle = \langle \Psi, L^{+}\Psi^{+} \rangle $}
\end{center}
\begin{equation}\label{eq:adj_2_identity}
	\langle \Psi^{+}, b \rangle =
	\langle \Psi, b^{+} \rangle 
\end{equation}
where $\langle \cdot \rangle$ signifies the integration over all phase space.
%\todo{Haven't found anything that states one variable can not be a function of
%the other variable in the inner product notation }

In order to complete the generalized MS-CADIS derivation, it is necessary to
assume that all responses of interest for both the primary physics and the
secondary physics can be expressed as inner products of their solutions with
some specific response functions.  Considering primary physics response $R_x$
and secondary physics response $R_y$, there should be response functions,
$\sigma_x$ and $\omega_y$, respectively, such that:
\begin{eqnarray}
   R_x(\phi) &=& \langle \sigma_x , \phi \rangle\\
   R_y(\psi) &=& \langle \omega_y , \psi \rangle
\end{eqnarray}
This is not strictly true in all cases.  However, since the
MS-CADIS method is used only to derive variance reduction parameters, it is
only necessary that an approximation exist that is sufficienctly accurate to
provide benefit from such parameters.  This benefit would need to be
demonstrated in any particular application of MS-CADIS.

In particular, MS-CADIS requires this requirement to be true of the
relationship between the source term for the secondary physics and the
solution to the primary phyics,
\begin{equation}\label{eq:coupling}
  b = \langle \sigma_b, \phi \rangle,
\end{equation}
and of the relationship between the ultimate response of interest and the
solution to the secondary physics,
\begin{equation} \label{eq:response}
  R_{final} = \langle \omega_R, \psi \rangle.
\end{equation}

For either physics, when the adjoint source is defined to be equal to a
particular response function, the adjoint solution can be interpretted as the
importance function for that particular response.  Therefore, for an adjoint source,
$b^{+}$, that is defined as the response function, $\omega_R$, applying the
adjoint identity to equation \ref{eq:response} results in
\begin{equation}
  R_{final} = \langle \omega_R, \psi \rangle = \langle b, \psi_R^{+}\rangle,
\end{equation}
where the subscript $R$ denotes that the adjoint solution, $\psi_R^{+}$, is an
importance function for response $R$.

Substituting equation \ref{eq:coupling} then gives:
\begin{equation}
  R_{final} = \left \langle \langle \sigma_b , \phi \rangle, \psi_R^{+} \right\rangle.
\end{equation}
By changing the order of integration between the phase space of the primary
physics and that of the secondary physics, this can be rewritten as:
\begin{equation}\label{eq:pseudo-response}
  R_{final} = \left \langle \langle \sigma_b , \psi_R^{+} \rangle, \phi \right\rangle.
\end{equation}
Once again invoking the adjoint identity gives
\begin{equation}
  R_{final} = \left \langle \langle \sigma_b , \psi_R^{+} \rangle, \phi \right\rangle = \langle q, \phi_R^{+} \rangle,
\end{equation}
if
\begin{equation}
  q^{+} \equiv \langle \sigma_b , \psi_R^{+} \rangle.
\end{equation}
This implies that $\phi_R^{+}$ describes the importance function of the
primary physics to the response of the secondary physics, and can be used in
the CADIS methodology to find weight window parameters for the first physics
that will ultimately accelerate the statistical convergence of the secondary
physics.

For a specific implementation in which the ultimate response is the prompt
photon dose response, the secondary physics is photon transport, and the
coupling term, $\sigma_b$, is the prompt photon production cross-section,
$\sigma(E_n,E_p)$.  This is generally implemented as a single physics problem
since the transport equations for neutrons and photons are identical, and this
coupling term can appear as a scattering-like between neutrons and photons.

For a specific implementation in which the ultimate response is the delayed
photon dose response, the secondary physics is also photon transport, but the
coupling term does not exist exactly.  
\todo{Delete these last two paragraphs}

Consider the process of neutron-induced prompt photon production.
In this case, the function $\sigma_b$ is the neutron-gamma production
cross section, $\sigma_{n,\gamma}(E_n,E_\gamma)$.  Because the transport
equations for neutrons and photons are identical, this is generally
implemented as a single-physics problem, in which the coupling term
$\sigma_{n,\gamma}$ appears as a scattering-like term between neutrons
and photons.

The primary focus of SDR analysis
is the process of neutron-induced delayed gamma production. 
GT-CADIS provides a method for calculating $\sigma_b$ when
certain conditions (known as SNILB) hold true. In this case, $\sigma_b$ is the
coupling term $T(E_n,E_\gamma)$, an approximation of the transmutation
process \cite{gtcadis}.  

An additional implication of this derivation is that there exists a response
function that allows the direct calculation of the secondary physics response
from the primary physics solution, as expressed in
equation \ref{eq:pseudo-response}.  This is exact for prompt photons generated
by a neutron source.  For delayed photons, this provides the underpinnings of
the D1S methodology and the more recent NASCA implementation\cite{kit-nasca}.


%\section{Time-integrated Solution of Adjoint Secondary Transport}
\section{Time-integrated MS-CADIS}
If the configuration of the geometry is changing over time during
the secondary physics, it will affect the construction of the adjoint radiation
transport source, $q^{+}$.  
The solutions to both forward and adjoint transport will be calculated in discrete
volume elements. There is a solution to the adjoint secondary
physics at each position and each time.% represented by the term
\begin{itemize}
	\item	{$ \Psi^{+}(\overrightarrow{r}_{v}(t), t)$ Adjoint flux in volume
		element v at time t}
	\item  {$\overrightarrow{r}_{v}(t)$ Position of volume element v at
		time t}
\end{itemize}
To solve for the adjoint radiation source in each volume element,
$q_{v}^{+}$, the time-dependent solutions of the 
adjoint secondary physics are combined by integrating over time.

 \begin{equation}\label{eq:adj_src_1_avg}
	 q_{v}^{+} =
	 \int_{t}  \Psi^{+}(\overrightarrow{r}_{v}(t), t)
	 \sigma_{c,v}(t)\, dt
 \end{equation}
%\todo{explain only sum, not average, is needed}
%\todo{This assumes that the coupling term $\sigma_{c,v}$
%does not change over time-- is that a fair assumption?}
This time-integrated source term is then used
for adjoint radiation transport to obtain $\phi_{v}^{+}$.

\section{Time-integrated GT-CADIS}
GT-CADIS is an implementation of MS-CADIS that is specific to SDR analysis.  It
provides a method to calculate a coupling term, T, that relates the neutron
flux to the photon source.
T is then used to solve for the adjoint neutron source as shown in Eq.
\ref{eq:gt_adj_nsrc}.
If the geometry configuration changes after shutdown, the time-integrated
MS-CADIS methodology shown in the previous section can be applied to the
GT-CADIS adjoint neutron source.  Adjoint photon transport at each time step
during geometry movement, t, will provide the adjoint flux 
of photons of energy $E_{\gamma}$, in volume element v, at time t,
$\phi_{\gamma}^{+}(\overrightarrow{r}_{v}(t), E_{\gamma},t) $

 \begin{equation}\label{eq:adj_src_1_avg}
	 q_{n,v}^{+}(E_{n}) =\frac
	 {\int_{t}  \int_{E_{\gamma}}
	 T_{v}(E_n, E_{\gamma}, t) 
	 \phi_{\gamma}^{+}(\overrightarrow{r}_{v}(t), E_{\gamma},t)
	 \, dE_{\gamma} \, dt}
	 {\int_t dt}
 \end{equation}
%\begin{equation}\label{eq:sum}
%	\phi_{\gamma,v, h}^{+} = \sum_{t_{mov}}{\phi_{\gamma,v,h,t_{mov}}}
%\end{equation}

$T_{v}(E_n, E_{\gamma}, t) $ is the T value of the material in volume
element v, at time t.
%There is a T value calculated for every volume element for every decay time of
%interest.  
For many practical problems, T will not change over the course of
geometry movement  because the time constants of decay and geometry
motion are very different.  The motion of components occurs over a very short
period of time relative to photon decay.  This is assuming that the geometry
movement will not begin until at least $10^5 s$ after shutdown when the remaining
photons have longer half-lives.  
Discretizing the energy spectrum into groups, the coupling term that relates 
the irradiation of the material in volume
element v, by a flux of neutrons in energy group g, to the corresponding source of
photons in energy group h, 
%$T_{v,g,h}$, 
is given by
\begin{equation}\label{eq:T}
	T_{v,g,h} = \dfrac{q_{\gamma,v,h}(\phi_{n,v,g})}{\phi_{n,v,g}}
\end{equation}
Using this groupwise calculation of T, the integral in Eq. \ref{eq:adj_src_1_avg} can be estimated by the sum
\begin{equation}\label{eq:tgt_n_src}
	q_{n,v,g}^{+} = \frac
	{\sum_{t_{mov}}\big(\sum_{h} T_{v,g,h}
	\phi_{\gamma,v,h,t_{mov}}^{+}\big) \Delta t_{mov}}
	{t_{tot}}
\end{equation}
where $t_{mov}$ is a time step after shutdown that corresponds to a change in
geometry configuration, $\Delta t_{mov}$ is the duration of the time step, and $\phi_{\gamma,v,h,t_{mov}}^{+}$ is the adjoint flux
of photons in energy group h, in volume element v, at that time step, and
$t_{tot}$ is the total duration of all the time steps.
