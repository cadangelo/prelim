\myexternaldocument{prelim}
%% This section will discuss how to adapt the MS-CADIS method to dynamic
%% systems
%\chapter{Adapt MS-CADIS for Moving Geometries}\label{ch:adapt}
\chapter{Variance Reduction for Time-integrated Multi-physics Analysis}\label{ch:tgt}

The Multi-Step Consistent Adjoint Weighted Importance Sampling (MS-CADIS)
method of variance reduction 
was developed to optimize the primary radiation transport
%is used to formulate an adjoint neutron source
%that represents the importance of the neutrons to the final quantity of interest
%that captures both the potential of regions to become activated and produce
%decay photons that contribute to the SDR at the detector location.
%The solution resulting from the adjoint neutron transport with this source
%is used to generate biasing parameters that will optimize the neutron 
%transport step of
in a coupled, multi-step process. The first implementation of this method 
was applied to the coupled neutron activation-photon decay process that occurs
in FES.  
%Specifically, it
%was shown to decrease
%the variance in the neutron transport step of the R2S workflow
%was used to optimize the neutron transport step in the 
%SDR analysis of the 
%ITER SDR benchmark experiment \cite{mscadis}.
In its current form, MS-CADIS is only applicable to static systems where
the geometry remains unchanged in all steps of the multi-step process.

This chapter will first discuss MS-CADIS outside of the context of SDR analysis. 
Next, a time-integrated solution to the adjoint of the 
physical process occurring during geometry movement will be derived.  
Finally, this time-integrated solution will be applied to the GT-CADIS method to
form the Time-integrated (T)GT-CADIS adjoint neutron source that will
ultimately be used to optimize the neutron transport step of SDR analysis.

\section{Generalized MS-CADIS Method}
In the current literature, MS-CADIS is primarily discussed as it applies to SDR
analysis \cite{mscadis}.  In actuality, MS-CADIS has always been intended to
apply to any multi-step process in
which the primary radiation transport is coupled to a secondary physical process.
The addition of time integration to this methodology can also be applied to
any coupled, multi-physics process. For this reason, it is prudent to discuss
MS-CADIS in a more generalized manner.

To describe the system of coupled, multi-physics,
the operator notation of the Boltzmann transport equation
\begin{equation}
	H\phi = q
\end{equation}
where H operates on the particle flux $\phi$ and q is a source of particles, 
will be used to represent the initial radiation transport.
An equation of the same form 
\begin{equation}
	L\Psi = b
\end{equation}
where L operates
\todo{requirements for operator?-- not exactly sure what this means}
on some function $\Psi$ and b is a source term, will be used
to describe a generic secondary physics.
Because this is a coupled system, the source of secondary physics, is a function
of the primary particle flux.
%, $b(\phi)$.

The adjoint identity for the neutral particle transport
equation was given in Eq. \ref{eq:3.2}.
% \begin{equation}\label{eq:adj_identity}
%		\langle \phi(\overrightarrow{r},E), q^{+}(\overrightarrow{r},E) \rangle =
%		\langle \phi^{+}(\overrightarrow{r},E) , q(\overrightarrow{r},E) \rangle
%		\langle q^{+}, H\phi \rangle =
%		\langle H^{+}\phi^{+}, q \rangle
% \end{equation}
%where $\phi$ is the forward and 
%$\phi^{+}$ is the adjoint particle flux, and
%$q$ is the forward and 
%$q^{+}$ is the adjoint source distribution and 
%$\langle \cdot \rangle$ signifies the
%integration over all dependent variables.
This identity is valid for an arbitrary
adjoint source function
\cite{l_m}, therefore the secondary physics has an adjoint identity of the same
form
\begin{center}
{$\langle \Psi^{+}, L\Psi \rangle = \langle \Psi, L^{+}\Psi^{+} \rangle $}
\end{center}
\begin{equation}\label{eq:adj_2_identity}
	\langle \Psi^{+}, b \rangle =
	\langle \Psi, b^{+} \rangle 
\end{equation}
where $\langle \cdot \rangle$ signifies the integration over all phase space.
\todo{Haven't found anything that states one variable can not be a function of
the other variable in the inner product notation }
For simplicity, a one-dimensional form of both primary and secondary physics
will be used going forward.

Assume this is a two-step system and the ultimate quantity of interest is the 
response of the secondary physics
which takes the form
\begin{equation}\label{eq:response}
%	R_{final} = \langle \sigma_b , \Psi \rangle
	R_{final} = \int_s \omega_R(s)  \Psi(s) ds 
\end{equation}
where $\omega_R$ is a detector response function and $s$ is the single independent
variable.
Defining the adjoint source to be equal to the detector response
function results in the adjoint solution having physical meaning as the
importance function for that detector. 
%\todo{Not sure if this is clear.}
Therefore, $b^{+}$ is substituted into Eq. \ref{eq:response}.
%the left side of the identity in Eq. \ref{eq:adj_2_identity} has the same
%form as Eq. \ref{eq:response}
%if a detector response function is chosen as the adjoint source.
 \begin{equation}\label{eq:response_2}
%	 R_{final} = \langle \phi_{2}(\overrightarrow{r},E), q_{2}^{+}(\overrightarrow{r},E) \rangle =
%		\langle \phi_{2}^{+}(\overrightarrow{r},E) , q_{2}(\overrightarrow{r},E) \rangle
	 %R_{final} = \langle \Psi, b^{+} \rangle 
	 R_{final} = \int_s \Psi(s) b^{+}(s) ds 
%		=\langle \Psi^{+} , b \rangle
 \end{equation}
Combining Eq. \ref{eq:response_2} with the adjoint identity yields
 \begin{equation}\label{eq:adj_2}
%	 R_{final} = \langle \phi_{2}(\overrightarrow{r},E), q_{2}^{+}(\overrightarrow{r},E) \rangle =
%		\langle \phi_{2}^{+}(\overrightarrow{r},E) , q_{2}(\overrightarrow{r},E) \rangle
	 %R_{final}=\langle \Psi^{+} , b \rangle
	 R_{final}= \int_s \Psi^{+}(s) b(s) ds
 \end{equation}

For a coupled, multi-step process, the MS-CADIS method 
sets the primary adjoint response equal to  
the final response of the system.  This results in the adjoint solution
representing
the importance of the primary physics to the final response of the system.
%(i.e. the response of the secondary physics)
 \begin{equation}\label{eq:adj_1}
%	 R_{final} = \langle \phi_{1}(\overrightarrow{r},E), q_{1}^{+}(\overrightarrow{r},E) \rangle =
%		\langle \phi_{1}^{+}(\overrightarrow{r},E) , q_{1}(\overrightarrow{r},E) \rangle
	 R_{final} = \int_r \phi(r) q^{+}(r)  dr
	 =\int_r \phi^{+}(r) q(r) dr
 \end{equation}

	 
 
% \todo{Unsure about symbols chosen for secondary physics and not listing
% explicit depedent variables here}

Ultimately, the goal is to find a solution to the adjoint
primary radiation transport to use as an importance
function for the forward calculation. Therefore, a
source for the adjoint radiation transport is needed.
Because this is a coupled, multi-step process, $q^{+}$ is a function of the
secondary physics and therefore can not be uniquely defined.   
MS-CADIS posits that there exists a way to represent an approximation of
$q^{+}$ that can be used to speed up the forward calculation. 

Begin by equating the adjoint
response equations in Eq. \ref{eq:adj_1} and \ref{eq:adj_2}, yields the
following expression.
 \begin{equation}\label{eq:adj_src}
%	 \langle \phi_{1}(\overrightarrow{r},E), q_{1}^{+}(\overrightarrow{r},E) \rangle =
%	 \langle \phi_{2}^{+}(\overrightarrow{r},E) , q_{2}(\overrightarrow{r},E) \rangle
	 \int_r \phi(r) q^{+}(r) dr =
	 \int_s \Psi^{+}(s) b(s) ds
 \end{equation}
In a practical scenario, the forward source of radiation (e.g. neutron
source resulting from D-T fusion) will be known.  
The adjoint source of secondary physics is chosen to be the response function for the
detector of interest (e.g. flux-to-dose conversion factors).
Given that these sources are known, both solutions, $ \phi $ and
$\Psi^{+} $, can be found through transport operations.
The source of the forward secondary, $b$,  and adjoint primary physics,
$q^{+}$, are both unknown; therefore, a second equation is needed.  
The source of the secondary physics is
a response of the primary radiation transport therefore the two processes can be related
by the following definition% \ref{eq:fwd_2src}

 \begin{equation}\label{eq:fwd_2src}
%	 q_{2}(\overrightarrow{r},E_{2}) =
%	 \langle \sigma_{1,2}, \phi_{1}(\overrightarrow{r},E_{1}) \rangle
	 b(s) = \int_r \sigma_b(r,s) \phi(r) dr
 \end{equation}
where $\sigma_b(r,s)$ is a function that couples the primary
radiation transport to the secondary physics.

Consider the process of neutron-induced prompt photon production.
In this case, the function $\sigma_b$ is the neutron-gamma production
cross section, $\sigma_{n,\gamma}$.  The primary focus of SDR analysis
is the process of neutron-induced delayed gamma production. 
GT-CADIS provides a method for calculating $\sigma_b$ when
certain conditions (known as SNILB) hold true. In this case, $\sigma_b$ is the
coupling term $T$, an approximation of the transmutation
process \cite{gtcadis}.  

As long as there is a solution for $\sigma_b$, the
function that couples the primary and secondary physics together,
there also exists a solution for the adjoint radiation transport source.
This can be discovered by substituting Eq. \ref{eq:fwd_2src} into Eq. 
\ref{eq:adj_src} 
\begin{equation}\label{eq:sub_b}
	\int_r q^{+}(r) \phi(r) dr 
	= \int_s \bigg[ \int_r \sigma_b(r,s) \phi(r) dr \bigg] \Psi^{+}(s) ds
\end{equation}
and switching the order of integration
\begin{equation}\label{eq:switch}
	\int_r q^{+}(r) \phi(r) dr 
	= \int_r \bigg[ \int_s \sigma_b(r,s) \Psi^{+}(s) ds \bigg] \phi(r) dr
\end{equation}
to solve for $q^{+}$.

 \begin{equation}\label{eq:adj_1src}
%	 q_{1}^{+}(\overrightarrow{r},E_{1}) = 
%	 \langle \sigma_{1,2}, \phi_{2}^{+}(\overrightarrow{r},E_{2}) \rangle
	 q^{+} = \int_s \sigma_b(r,s) \Psi^{+}(s) ds
 \end{equation}


%\section{Time-integrated Solution of Adjoint Secondary Transport}
\section{Time-integrated MS-CADIS}
If the configuration of the geometry is changing over time during
the secondary physics, it will affect the construction of the adjoint radiation
transport source, $q^{+}$.  
The solutions to both forward and adjoint transport will be calculated in discrete
volume elements. There is a solution to the adjoint secondary
physics at each position and each time.% represented by the term
\begin{itemize}
	\item	{$ \Psi^{+}(\overrightarrow{r}_{v}(t), t)$ Adjoint flux in volume
		element v at time t}
	\item  {$\overrightarrow{r}_{v}(t)$ Position of volume element v at
		time t}
\end{itemize}
To solve for the adjoint radiation source in each volume element,
$q_{v}^{+}$, the time-dependent solutions of the 
adjoint secondary physics are combined by integrating over time.

 \begin{equation}\label{eq:adj_src_1_avg}
	 q_{v}^{+} =
	 \int_{t}  \Psi^{+}(\overrightarrow{r}_{v}(t), t)
	 \sigma_{c,v}(t)\, dt
 \end{equation}
%\todo{explain only sum, not average, is needed}
%\todo{This assumes that the coupling term $\sigma_{c,v}$
%does not change over time-- is that a fair assumption?}
This time-integrated source term is then used
for adjoint radiation transport to obtain $\phi_{v}^{+}$.

\section{Time-integrated GT-CADIS}
GT-CADIS is an implementation of MS-CADIS that is specific to SDR analysis.  It
provides a method to calculate a coupling term, T, that relates the neutron
flux to the photon source.
T is then used to solve for the adjoint neutron source as shown in Eq.
\ref{eq:gt_adj_nsrc}.
If the geometry configuration changes after shutdown, the time-integrated
MS-CADIS methodology shown in the previous section can be applied to the
GT-CADIS adjoint neutron source.  Adjoint photon transport at each time step
during geometry movement, t, will provide the adjoint flux 
of photons of energy $E_{\gamma}$, in volume element v, at time t,
$\phi_{\gamma}^{+}(\overrightarrow{r}_{v}(t), E_{\gamma},t) $

 \begin{equation}\label{eq:adj_src_1_avg}
	 q_{n,v}^{+}(E_{n}) =\frac
	 {\int_{t}  \int_{E_{\gamma}}
	 T_{v}(E_n, E_{\gamma}, t) 
	 \phi_{\gamma}^{+}(\overrightarrow{r}_{v}(t), E_{\gamma},t)
	 \, dE_{\gamma} \, dt}
	 {\int_t dt}
 \end{equation}
%\begin{equation}\label{eq:sum}
%	\phi_{\gamma,v, h}^{+} = \sum_{t_{mov}}{\phi_{\gamma,v,h,t_{mov}}}
%\end{equation}

$T_{v}(E_n, E_{\gamma}, t) $ is the T value of the material in volume
element v, at time t.
%There is a T value calculated for every volume element for every decay time of
%interest.  
For many practical problems, T will not change over the course of
geometry movement  because the time constants of decay and geometry
motion are very different.  The motion of components occurs over a very short
period of time relative to photon decay.  This is assuming that the geometry
movement will not begin until at least $10^5 s$ after shutdown when the remaining
photons have longer half-lives.  
Discretizing the energy spectrum into groups, the coupling term that relates 
the irradiation of the material in volume
element v, by a flux of neutrons in energy group g, to the corresponding source of
photons in energy group h, 
%$T_{v,g,h}$, 
is given by
\begin{equation}\label{eq:T}
	T_{v,g,h} = \dfrac{q_{\gamma,v,h}(\phi_{n,v,g})}{\phi_{n,v,g}}
\end{equation}
Using this groupwise calculation of T, the integral in Eq. \ref{eq:adj_src_1_avg} can be estimated by the sum
\begin{equation}\label{eq:tgt_n_src}
	q_{n,v,g}^{+} = \frac
	{\sum_{t_{mov}}\big(\sum_{h} T_{v,g,h}
	\phi_{\gamma,v,h,t_{mov}}^{+}\big) \Delta t_{mov}}
	{t_{tot}}
\end{equation}
where $t_{mov}$ is a time step after shutdown that corresponds to a change in
geometry configuration, $\Delta t_{mov}$ is the duration of the time step, and $\phi_{\gamma,v,h,t_{mov}}^{+}$ is the adjoint flux
of photons in energy group h, in volume element v, at that time step, and
$t_{tot}$ is the total duration of all the time steps.
