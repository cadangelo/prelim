%% This section will discuss how to adapt the MS-CADIS method to dynamic
%% systems
%\chapter{Adapt MS-CADIS for Moving Geometries}\label{ch:adapt}
\chapter{Variance Reduction for Time-integrated Multiphysics Analysis}\label{ch:ti}

The Multi-Step Consistent Adjoint Weighted Importance Sampling (MS-CADIS)
method of variance reduction is used to formulate an adjoint neutron source
that represents the importance of the neutrons to the final quantity of interest
%that captures both the potential of regions to become activated and produce
%decay photons that contribute to the SDR at the detector location.
%The solution resulting from the adjoint neutron transport with this source
%is used to generate biasing parameters that will optimize the neutron 
%transport step of
in a coupled, multi-step process. The first implementation of this method 
was applied to the coupled neutron activation-photon decay process that occurs
in FES.  Specifically, it
%was shown to decrease
%the variance in the neutron transport step of the R2S workflow
was used to optimize the neutron transport step in the 
%SDR analysis of the 
ITER SDR benchmark experiment \cite{mscadis}.
In its current form, MS-CADIS is only applicable to static systems where
the geometry remains unchanged in all steps of the multi-step process.

This chapter will first discuss MS-CADIS outside of the context of SDR analysis. 
Next, a time-integrated solution to the adjoint of the 
physical process occuring during geometry movement will be derived.  
Finally, this time-integrated solution will be applied to the GT-CADIS method to
form the TGT-CADIS adjoint neutron source that will optimize the neutron
transport step of SDR analysis.

\section{Generalized MS-CADIS Method}
In the current literature, MS-CADIS is primarily discussed as it applies to SDR
analysis \cite{mscadis}.  In actuality, MS-CADIS can be applied to any multi-step process in
which the primary radiation transport is coupled to a secondary physical process.
The addition of time integration to this methodology can then also be applied to
any coupled multiphysics process. For this reason, it is prudent to discuss
MS-CADIS in a more generalized manner.

Begin with the adjoint identity for the neutral particle transport
equation, 
 \begin{equation}\label{eq:adj_identity}
%		\langle \phi(\overrightarrow{r},E), q^{+}(\overrightarrow{r},E) \rangle =
%		\langle \phi^{+}(\overrightarrow{r},E) , q(\overrightarrow{r},E) \rangle
		\langle q^{+}, \phi \rangle =
		\langle \phi^{+}, q \rangle
 \end{equation}
where $\phi$ is the forward and 
$\phi^{+}$ is the adjoint particle flux, and
$q$ is the forward and 
$q^{+}$ is the adjoint source distribution and $\langle \rangle$ signifies the
integration over all dependent variables.
The left-hand side of Eq. \ref{eq:adj_identity} has the same form as the
equation for detector response
if a detector response function is chosen as the adjoint source.
For a coupled, multi-step process, the MS-CADIS method 
constrains the primary adjoint response such that it is equal to  
the final response of the system (i.e. the response of the secondary process) as shown below
 \begin{equation}\label{eq:adj_1}
%	 R_{final} = \langle \phi_{1}(\overrightarrow{r},E), q_{1}^{+}(\overrightarrow{r},E) \rangle =
%		\langle \phi_{1}^{+}(\overrightarrow{r},E) , q_{1}(\overrightarrow{r},E) \rangle
	 R_{final} = \langle \phi_{1}, q_{1}^{+} \rangle =
		\langle \phi_{1}^{+} , q_{1} \rangle
 \end{equation}

	 
 \begin{equation}\label{eq:adj_2}
%	 R_{final} = \langle \phi_{2}(\overrightarrow{r},E), q_{2}^{+}(\overrightarrow{r},E) \rangle =
%		\langle \phi_{2}^{+}(\overrightarrow{r},E) , q_{2}(\overrightarrow{r},E) \rangle
	 R_{final} = \langle \phi_{2}, q_{2}^{+} \rangle =
		\langle \phi_{2}^{+} , q_{2} \rangle
 \end{equation}
 
 \todo{Unsure about symbols chosen for secondary physics and not listing
 explicit depedent variables here}

Subscripts 1 and 2 denote primary and secondary physical processes,
respectively.  Ultimately, the goal is to find a solution for the adjoint
transport of the primary physical process to use as an importance
function for the forward primary transport. Therefore, a
source for the adjoint primary transport is needed.  Equating the set of adjoint
identities in Eq. \ref{eq:adj_1} and \ref{eq:adj_2}, yields the expression in Eq.
\ref{eq:adj_src}.

 \begin{equation}\label{eq:adj_src}
%	 \langle \phi_{1}(\overrightarrow{r},E), q_{1}^{+}(\overrightarrow{r},E) \rangle =
%	 \langle \phi_{2}^{+}(\overrightarrow{r},E) , q_{2}(\overrightarrow{r},E) \rangle
	 \langle \phi_{1}, q_{1}^{+} \rangle =
	 \langle \phi_{2}^{+}, q_{2} \rangle
 \end{equation}

In a practical scenario, the forward source of primary physics (e.g. neutron
source resulting from D-T fusion) will be known.  According to the MS-CADIS
method, the source of adjoint
secondary physics is chosen to be the response function for the
detector of interest.
Given that these sources are known, both solutions, $ \phi_{1} $ and
$\phi_{2}^{+} $ can be found through transport operations.
The source of the forward secondary and adjoint primary physics are both
unknown, therefore a second equation is needed.  This is a coupled, multi-step
system where the source of the secondary physics is
a response of the primary transport therefore the two processes can be related
by equation \ref{eq:fwd_2src}

 \begin{equation}\label{eq:fwd_2src}
%	 q_{2}(\overrightarrow{r},E_{2}) =
%	 \langle \sigma_{1,2}, \phi_{1}(\overrightarrow{r},E_{1}) \rangle
	 q_{2} =
	 \langle \sigma_{1\rightarrow2}, \phi_{1} \rangle
 \end{equation}
where $\sigma_{1\rightarrow2}$ is a function that couples the primary
and secondary physics.
Consider the process of neutron-induced prompt photon production.
In this case, the function, $sigma_{1\rightarrow2}$, is the neutron-gamma production
cross section $\sigma_{n\rightarrow\gamma}$.  The primary focus of SDR analysis
is the process of neutron-induced delayed gamma production. 
GT-CADIS provides a method for calculating $\sigma_{1\rightarrow2}$ when
certain conditions (known as SNILB) hold true. In this case, the
coupling term, $T$, is an approximation of the transmutation
process \cite{gtcadis}.  As long as there is a solution for the
function that couples the primary and secondary physics together,
there also exists a solution for the adjoint primary source as shown in Eq.
\ref{eq:adj_1src}

 \begin{equation}\label{eq:adj_1src}
%	 q_{1}^{+}(\overrightarrow{r},E_{1}) = 
%	 \langle \sigma_{1,2}, \phi_{2}^{+}(\overrightarrow{r},E_{2}) \rangle
	 q_{1}^{+} = 
	 \langle \sigma_{1\rightarrow2}, \phi_{2}^{+} \rangle
 \end{equation}


\section{Time-integrated Solution of Adjoint Secondary Transport}
If the configuration of the geometry is changing over time during
the secondary transport, it will affect the construction of the adjoint primary
source.  Instead of a single solution to the adjoint secondary transport, there
is a solution at each position and each time represented by the term
	$ \phi_{2}^{+}(\overrightarrow{r}_{v}(t), t)$
where $\overrightarrow{r}_{v}(t)$ refers to the position of volume element, v,
at time, t. 
\todo{need to introduce volume element}
To solve for the adjoint primary source in each volume element
$q_{1,v}^{+}$, the time-dependent solutions of the 
adjoint secondary transport are combined by integrating over time

 \begin{equation}\label{eq:adj_src_1_avg}
	 q_{1,v}^{+} =
	 \int_{t}  \phi_{2}^{+}(\overrightarrow{r}_{v}(t), t)
	 \sigma_{1\rightarrow 2,v}(t)\, dt
 \end{equation}
\todo{explain only sum, not average, is needed}
\todo{This assumes that the coupling term $\sigma_{1 \rightarrow 2,v}$
does not change over time-- is that a fair assumption?}
 
\section{Time-integrated GT-CADIS}

GT-CADIS is an implementation of MS-CADIS that is specific to SDR analysis.  It
provides a method to calculate a coupling term, T, that relates the neutron
flux to the photon source.
Applying time integration to the GT-CADIS methodology results in the following
solution for the adjoint neutron source

 \begin{equation}\label{eq:adj_src_1_avg}
	 q_{n,v}^{+}(E_{n}) =
	 \int_{t}     \phi_{\gamma}^{+}(\overrightarrow{r}_{v}(t), E_{\gamma},t)
	 T_{v}(E_n, E_{\gamma}, t)\, dt
 \end{equation}
 There is a T value calculated for every volume element for every decay time of
 interest.  For many practical problems, T will not change over the course of
 $t_mov$  because the time constants of decay and geometry
 motion are very different.  The motion of components occurs over a very short
 period of time relative to photon decay.  
