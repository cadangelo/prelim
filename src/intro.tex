\myexternaldocument{prelim}

\chapter{Introduction} \label{ch:intro}

The successful completion of this project will provide the workflow and tools
necessary to efficiently calculate quantities of interest resulting from
coupled, multi-physics processes in dynamic systems.  The MC physics code will
be modified to implement rigid-body transformations on the CAD-based geometry.
MS-CADIS, a VR method for coupled, multi-physics problems, will be adapted to
incorporate dynamics.  An experiment will be contrived to demonstrate the
limitations of existing VR methods as they apply to dynamic problems and verify
the efficacy of this new method.  Given these objectives, the following chapters
will include background and theory relevant to VR methods in coupled,
multi-physics systems.  Chapter \ref{ch:litrev} will provide an introduction to computational
radiation transport and specifically the VR methods used in MC calculations.
 Chapter no. will discuss SDR analysis and the VR methods specific to these
calculations.  Chapter no. will introduce radiation transport in dynamic systems,
discuss how they are handled now and how motion affects VR in coupled
multi-physics problems like SDR calculations.  Finally, chapter no. will discuss
the progress that has been made towards this new methodology and outline a
proposal of the work to be done.

The rapid design iteration process of complex nuclear systems has long been
aided by the use of computational simulation.  Traditionally, these simulations
involve radiation transport in static geometries.  However, in certain
scenarios, it is desirable to investigate dynamic systems and the effects caused
by the motion of one or more components.  
For example, Fusion Energy Systems
(FES) are purposefully designed with modular components that can be moved in and
out of a facility after shutdown for maintenance.  
To ensure the safety of maintenance personnel, it is important to accurately quantify the
shutdown dose rate (SDR) caused by the gammas emitted by structural materials that became 
activated during the device operation time.  
This type of analysis requires neutron transport to determine the neutron flux,
activation calculation to determine the isotopic inventory, and finally a 
photon transport calculation to determine the SDR.
While MC calculations are revered to be the most accurate method for simulating
radiation transport, the computational expense of obtaining low error results 
in systems with heavy shielding can be prohibitive.  
However there are techniques, known as variance reduction (VR)
methods, that can be used to increase the computational efficiency.  
There are several types
of VR methods, but the basic theory is to artificially increase the simulation of
events that will contribute to the quantity of interest such as flux or dose
rate. 
 One class of VR techniques relies upon a deterministic estimate of the adjoint 
solution of the transport equation to
formulate biasing parameters used in the MC transport. 
The adjoint flux has physical significance as the importance of a region of
phase space to the objective function.

The purpose of this work is to create a methodology for the efficient calculation
of quantities of
interest in dynamic, geometrically complex nuclear systems.
%The motion is discretized into time steps and MC transport is performed at each
%step.  
For cases involving coupled multi-physics analysis, such as SDR calculations,
a new hybrid deterministic/MC VR technique will be proposed.
This new method will adapt the Multi-Step Consistent Adjoint Driven Importance Sampling
(MS-CADIS) method to dynamic systems.
 The basis of MS-CADIS is that the importance function used
in each step of the problem must represent the importance of the particles to
the final objective function.  As the spatial configuration of the materials
changes, the probability that they will contribute to the objective function
also changes.
In the specific case of SDR calculations, the importance function for the neutron transport step
must capture the probability of materials to become activated and subsequently emit photons that
will make a significant contribution to the SDR.
This new VR method will also take advantage of
 the Groupwise Transmutation (GT)-CADIS method 
which is an implementation of MS-CADIS
%uses single-energy-group irradiations to generate an adjoint neutron source for
that optimizes the neutron transport step of SDR calculations.
GT-CADIS generates an adjoint
neutron source based on certain assumptions and approximations about the
transmutation network.  To adapt this method for dynamic systems, the adjoint
neutron source will be calculated at each time step and then averaged
in order to generate the biasing functions for the neutron transport step.


% adjoint photon transport to get adjoint photon flux
% GT-CADIS generates soln for adjoint n source from adjoint photon flux
% average the neutron adjoint flux from every time step to generate source and
% transport biasing parameters for forward, MC neutron transport calc
% Standard GT-CADIS is for static systems and would 
% as the spatial configuration changes over time, the importance also changes?
