\myexternaldocument{prelim}

\chapter{Introduction} \label{ch:intro}

The rapid design iteration process of complex nuclear systems has long been
aided by computational simulation.  Traditionally, these simulations
involve radiation transport in static geometries.  However, in certain
scenarios, it is desirable to investigate dynamic systems and the effects caused
by the motion of one or more components.  
For example, fusion energy systems (FES) are purposefully designed with modular components that can be moved in and
out of a facility after shutdown for maintenance purposes.  
To ensure the safety of maintenance personnel, it is important to accurately quantify the
shutdown dose rate (SDR) caused by the photons emitted by structural materials that were
activated during the device operation time.  
This type of analysis requires neutron transport to determine the neutron flux,
activation analysis to determine the isotopic inventory, and finally a 
photon transport calculation to determine the SDR.
While Monte Carlo (MC) calculations are considered to be the most accurate method for simulating
radiation transport, the computational expense of obtaining results with low
error 
in systems with heavy shielding can be prohibitive.  
However, variance reduction (VR)
methods can be used to increase the computational efficiency.  
There are several types
of VR methods, but the basic theory is to artificially increase the simulation of
events that will contribute to the quantity of interest such as flux or dose
rate. 
 One class of VR techniques takes advantage of a deterministic estimate of the adjoint 
solution of the transport equation to automatically
generate biasing parameters to accelerate the MC transport. 
The adjoint flux has physical significance as the importance of a region of
phase space to the objective function.

%The purpose of this work is to create a methodology for the efficient calculation
%of quantities of
%interest in dynamic, geometrically complex nuclear systems.
%The motion is discretized into time steps and MC transport is performed at each
%step.  
%For cases involving coupled multi-physics analysis in dynamics systems, such as SDR calculations
%during maintenance activities,
%a new hybrid deterministic/MC VR technique will be proposed.
%This new method will adapt i
One hybrid VR technique used to optimize the initial transport step of a
multi-step process is known as 
the Multi-Step Consistent Adjoint Driven Importance Sampling
(MS-CADIS). 
%method to dynamic systems.
 The basis of MS-CADIS is that the importance (adjoint) function used
in each step of the problem must represent the importance of the particles to
the final objective function.  
%As the spatial configuration of the materials
%changes, the probability that they will contribute to the objective function
%also changes.
In the specific case of SDR calculations, the importance function for the neutron transport step
must capture the probability of materials to become activated and subsequently emit photons that
will make a significant contribution to the SDR.
The Groupwise Transmutation (GT)-CADIS method 
%which 
is an implementation of MS-CADIS
%uses single-energy-group irradiations to generate an adjoint neutron source for
that optimizes the neutron transport step of SDR calculations.
%, will be updated to allow for movement after shutdown.
GT-CADIS generates an adjoint
neutron source based on certain assumptions and approximations about the
transmutation network.  
This source is used for adjoint transport and the resulting flux is used to
generate the biasing parameters to optimize the forward neutron transport.
For cases involving coupled multi-physics analysis in dynamics systems, such as SDR calculations
during maintenance activities,
a new hybrid deterministic/MC VR technique will be proposed that adapts 
GT-CADIS for dynamic systems by calculating a time-integrated adjoint
neutron source.


% adjoint photon transport to get adjoint photon flux
% GT-CADIS generates soln for adjoint n source from adjoint photon flux
% average the neutron adjoint flux from every time step to generate source and
% transport biasing parameters for forward, MC neutron transport calc
% Standard GT-CADIS is for static systems and would 
% as the spatial configuration changes over time, the importance also changes?

The successful completion of this project will demonstrate the efficacy of a workflow and tools
necessary to efficiently calculate quantities of interest resulting from
coupled, multi-physics processes in dynamic systems.  
The driving force behind this work is the quantification of the
SDR resulting from the coupled neutron
irradiation-photon emission that occurs in FES;
specifically investigating how to optimize the SDR calculation when %changes as a function of time when
activated system components are moved during maintenance activities.
The Monte Carlo radiation transport code will
be modified to implement rigid-body transformations on the CAD-based geometry
and GT-CADIS will be adapted to
incorporate dynamics.  An experiment will be contrived to demonstrate the
limitations of existing VR methods as they apply to dynamic problems and verify
the efficacy of this new method.  Given these objectives, Chapter \ref{ch:litrev}
will include background and theory relevant to VR methods in coupled,
multi-physics systems.  
It begins with an introduction to computational
radiation transport, specifically the Monte Carlo method.
%It will then introduce the methods currently used for SDR analysis and the
%need for optimization.  Several MC VR techniques and how they have been applied
% to optimize SDR analysis are then discussed.
%Finally, this chapter ends with a discussion about radiation transport in
%dynamic systems.  
Next, Chapter \ref{ch:experiment} shows the experiment demonstrating the need
for a new VR method to optimize the SDR in dynamic systems.  The derivation of
VR parameters that will optimize the SDR in dynamic systems is given in
Chapter \ref{ch:tgt}. Finally, Chapter \ref{ch:proposal} details an implementation plan, 
the progress that has been made thus far, and a summary of the
work to be done.
