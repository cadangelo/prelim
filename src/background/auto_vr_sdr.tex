\subsection{Automated VR for SDR Calculations}
There are two transport steps involved in SDR calculations; the initial
neutron trasport to simulate the irradiation and the subsequent photon
transport simulating the decaying gammas.  The full-scale simulations in
large, complex FES models are very computationally expensive at best, and
impossible at worst.
The R2S method applied to a full-scale, 3D FES become impractical due to the
need for accurate space- and energy-dependent fluxes generated by MC codes 
throughout the geometry.  Biasing the photon transport step is a
straightforward application of the CADIS method *need source*.  The detector
response function  can be set equal to the photon flux tally fitted with
flux-to-dose-rate conversion factors.  The adjoint flux can then be used as an
importance function to determine source and transport biasing parameters *need
source*.  Biasing the neutron transport step, however, is not as
straightforward.  The response function depends on the subsequent
computational steps of activation and decay photon transport \cite{Ibrahim_2015}.
The Multi-Step Consistent Adjoint Driven Importance Sampling (MS-CADIS) method
was developed to increase the efficiency of the neutron transport step of the
SDR calculation process \cite{Ibrahim_2015}.  It uses an importance function
to bias the neutron transport that represents the importance of neutrons to
the final SDR.
As shown before in Eq. *need eqn number*, the detector response can be
expressed as the integral of the importance function, I, times the source
distribution, q
\begin{equation} \label{eq:1}
R = \int_{V}\int_{E} 
    I(\overrightarrow{r}, E)
    q(\overrightarrow{r}, E)
    dV dE
\end{equation}
 
