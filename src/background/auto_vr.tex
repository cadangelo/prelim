\subsection{Automated VR Techniques}

%Historically, variance reduction methods require a vast amount of user expertise
%and time.
VR techniques reqire the user to have a priori knowledge of the problem physics
in order to assign importance parameters.
Many techniques have been developed over the years to automate the selection and
assignment of these parameters to reduce the time and expertise required by
the user.

One class of VR techniques, known as hybrid deterministic/MC methods,
is based upon
the solution to the adjoint Boltzmann transport equation having physical
significance as the measure of importance of a particle to some specified
objective function.  
Because determisitic solutions to the transport equation are less precise but
require much less computation time, they are useful as an estimate of
particle flux throughout the domain which can then be used to determine
importance of specific regions.
To demonstrate the use of the adjoint solution as an importance function,
we will first start with the linear, time-independent Boltzmann transport
equation shown below
\begin{equation} \label{eq:1a}
  H\Psi = q
\end{equation}
where $\Psi$ is the angular flux, q is the source of particles, and the operator H
is given by
\begin{equation} \label{eq:1b}
		H = \widehat{\Omega} \cdot \nabla +
		    \sigma_{t}(\overrightarrow{r},E) - 
			\int_{0}^{\infty} dE'
			\int_{4\pi} d\Omega'
			\sigma_{s}( \overrightarrow{r}, E' 
			\rightarrow E, \widehat{\Omega}' 
			\rightarrow \widehat{\Omega} )
\end{equation}
where $\sigma_{t}$ is the total cross-section and $\sigma_{s}$ is the
differential scattering cross-section.
The adjoint identity states that
\begin{equation} \label{eq:2}
		\langle \Psi^{+} , H\Psi \rangle =
		\langle \Psi, H^{+}\Psi^{+} \rangle
\end{equation}
where $ \langle \cdot \rangle$ refers to the integration over space,
energy, and angle and the adjoint operator $H^{+}$ is given by
\begin{equation} \label{eq:2b}
		H^{+} = -\widehat{\Omega} \cdot \nabla +
		    \sigma_{t}(\overrightarrow{r},E) - 
			\int_{0}^{\infty} dE'
			\int_{4\pi} d\Omega'
			\sigma_{s}( \overrightarrow{r}, E 
			\rightarrow E', \widehat{\Omega} 
			\rightarrow \widehat{\Omega}' )
\end{equation}
The identity can also be written as
\begin{equation} \label{eq:2b}
		\langle \Psi^{+} , q \rangle =
		\langle \Psi, q^{+} \rangle
\end{equation}
As mentioned, the adjoint solution to the transport
equation will be used as an importance function therefore we need to solve
\begin{equation} \label{eq:3}
		H^{+}\Psi{+} = q^{+}
\end{equation}
which requires the thoughtful selection of an adjoint source $q^{+}$.
To demonstrate the physical significance of the adjoint, we will consider the
detector response, R
\begin{equation} \label{eq:4}
		R = \langle \Psi, \sigma_{d}
\end{equation}
where $\sigma_{d}$ is a detector response function.
If we choose the adjoint source to be equivalent to the detector response
function,
\begin{equation} \label{eq:5}
		q^{+} = \sigma_{d}
\end{equation}
and substitute this into Eq. \ref{eq:4} and Eq. \ref{eq:2b} 
\begin{equation}
		R = \langle \Psi, q^{+} \rangle = \langle \Psi^{+}, q \rangle
\end{equation}
the adjoint flux $\Psi^{+}$ represents the importance of a region to
$\sigma_{d}$.
This final relation allows us to know the response R for any source q once the
adjoint flux $\Psi^{+}$ for a detector of interest is known.

The Consistent Adjoint Driven Importance Sampling (CADIS) method is one of the
hybrid deterministic  VR
techniques that uses the adjoint solution to 
formulate source and transport biasing parameters for MC transport.
More specifically, CADIS determines the parameters for source biasing and the
weight window lower bounds in a consistent manner.
The response, or tally, of interest in a transport calculation can be represented by
the following equation
\begin{equation} \label{eq:6}
	R = \int_{V}d\overrightarrow{r} \int_{E} dE
	\int_{4\pi} d\widehat{\Omega}
	\sigma_{d}(\overrightarrow{r}, E, \widehat{\Omega})
	\Psi(\overrightarrow{r}, E, \widehat{\Omega})
\end{equation}
and in terms of the adjoint flux, 
\begin{equation} \label{eq:7}
	R = \int_{V}d\overrightarrow{r} \int_{E} dE
	\int_{4\pi} d\widehat{\Omega}
	q(\overrightarrow{r}, E, \widehat{\Omega})
	\Psi^{+}(\overrightarrow{r}, E, \widehat{\Omega})
\end{equation}
The MC solution of the response relies upon the sampling of the particle source
distribution, $q(\overrightarrow{r}, E, \widehat{\Omega})$, represented by a PDF.
In an effort to decrease the variance, it is possible to sample from a biased
PDF which is given by
\begin{equation} \label{eq:8}
	\widehat{q}(\overrightarrow{r}, E, \widehat{\Omega}) =
	\frac{\Psi^{+}(\overrightarrow{r}, E,\widehat{\Omega})
	q(\overrightarrow{r}, E, \widehat{\Omega})}{R}
\end{equation}
This biased PDF represents the contribution of particles from phase-space
$(\overrightarrow{r}, E, \widehat{\Omega})$ to the total detector response, R.
As previously mentioned, when sampling from a biased PDF, the particle weight
needs to be adjusted to eliminate systematic bias.  
\begin{equation} \label{eq:9}
	w(\overrightarrow{r}, E,
	\widehat{\Omega})\widehat{q}(\overrightarrow{r}, E, \widehat{\Omega})=
	w_{0}q(\overrightarrow{r}, E, \widehat{\Omega})
\end{equation}
Substituting in Eq. \ref{eq:8}, the corrected paticle weight is shown to have
an inverse relation to the adjoint flux, or importance function.
\begin{equation} \label{eq:10}
	w(\overrightarrow{r}, E, \widehat{\Omega})=
	\frac{R}{\Psi^{+}(\overrightarrow{r}, E, \widehat{\Omega})}
\end{equation}
%The biased transport operator is derived consistently from the biased source
%parameter.
In the weight window technique, particles are either split or
rouletted as they move from region to region based on the ratio of their
importances and their weight is updated accordingly.  
%This can be extended to
%the transport biasing parameter which uses the adjoint flux as an importance
%function as follows
The weights are used for both the source and transport biasing parameters and
are derived from importance sampling in a consistent manner.  
The transport is biased according to the following relationship
\begin{equation} \label{eq:11}
	w(\overrightarrow{r}, E, \widehat{\Omega})=
	w(\overrightarrow{r}', E', \widehat{\Omega}')
	\left [ \frac{\Psi^{+}(\overrightarrow{r}', E', \widehat{\Omega}')}
	{\Psi^{+}(\overrightarrow{r}, E, \widehat{\Omega})} \right ]
\end{equation}
The width of the weight windows is determined by a parameter defined to be the
ratio between upper and lower bounds $\alpha =
w_{u}/w_{l}$.  MCNP uses a default value for $\alpha$ and the weight window lower
bounds are given by 
\begin{equation} \label{eq:12}
	w_{l}(\overrightarrow{r}, E, \widehat{\Omega}) = 
	\frac{R}{\Psi^{+}(\overrightarrow{r}, E, \widehat{\Omega})
	(\frac{\alpha + 1}{2})}
\end{equation}




