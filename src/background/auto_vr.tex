\subsection{Automated VR Techniques}

%Historically, variance reduction methods require a vast amount of user expertise
%and time.
VR techniques reqire the user to have a priori knowledge of the problem physics
in order to assign importance parameters.
Many techniques have been developed over the years to automate the selection and
assignment of these parameters to reduce the time and expertise required by
the user.

One class of VR techniques, known as hybrid deterministic/MC methods,
is based upon
the solution to the adjoint Boltzmann transport equation having physical
significance as the measure of importance of a particle to some specified
objective function.  
Because determisitic solutions to the transport equation are less precise but
require much less computation time, they are useful as an estimate of
particle flux throughout the domain which can then be used to determine
importance of specific regions.
To demonstrate the use of the adjoint solution as an importance function,
we will first start with the linear, time-independent Boltzmann transport
equation shown below
\begin{equation} \label{eq:1a}
  H\Psi = q
\end{equation}
where $\Psi$ is the angular flux, q is the source of particles, and the operator H
is given by
\begin{equation} \label{eq:1b}
		H = \widehat{\Omega} \cdot \nabla +
		    \sigma_{t}(\overrightarrow{r},E) - 
			\int_{0}^{\infty} dE'
			\int_{4\pi} d\Omega'
			\sigma_{s}( \overrightarrow{r}, E' 
			\rightarrow E, \widehat{\Omega}' 
			\rightarrow \widehat{\Omega} )
\end{equation}
where $\sigma_{t}$ is the total cross-section and $\sigma_{s}$ is the
differential scattering cross-section.
The adjoint identity states that
\begin{equation} \label{eq:2}
		\langle \Psi^{+} , H\Psi \rangle =
		\langle \Psi, H^{+}\Psi^{+} \rangle
\end{equation}
where $ \langle \cdot \rangle$ refers to the integration over space,
energy, and angle and the adjoint operator $H^{+}$ is given by
\begin{equation} \label{eq:2b}
		H^{+} = -\widehat{\Omega} \cdot \nabla +
		    \sigma_{t}(\overrightarrow{r},E) - 
			\int_{0}^{\infty} dE'
			\int_{4\pi} d\Omega'
			\sigma_{s}( \overrightarrow{r}, E 
			\rightarrow E', \widehat{\Omega}' 
			\rightarrow \widehat{\Omega}' )
\end{equation}
The identity can also be written as
\begin{equation} \label{eq:2b}
		\langle \Psi^{+} , q \rangle =
		\langle \Psi, q^{+} \rangle
\end{equation}
As mentioned, the adjoint solution to the transport
equation will be used as an importance function therefore we need to solve
\begin{equation} \label{eq:3}
		H^{+}\Psi{+} = q^{+}
\end{equation}
which requires the thoughtful selection of an adjoint source $q^{+}$.
To demonstrate the physical significance of the adjoint, we will consider the
detector response, R
\begin{equation} \label{eq:4}
		R = \langle \Psi, \sigma_{d}
\end{equation}
where $\sigma_{d}$ is a detector response function.
If we choose the adjoint source to be equivalent to the detector response
function,
\begin{equation} \label{eq:5}
		q^{+} = \sigma_{d}
\end{equation}
and substitute this into \ref{eq:4} and \ref{eq:2b} 
\begin{equation}
		R = \langle \Psi, q^{+} \rangle = \langle \Psi^{+}, q \rangle
\end{equation}
the adjoint flux $\Psi^{+}$ represents the importance of a region to
$\sigma_{d}$.
This final relation allows us to know the response R for any source q once the
adjoint flux $\Psi^{+}$ for a detector of interest is known.

The Consistent Adjoint Driven Importance Sampling (CADIS) method is one of the
hybrid deterministic  VR
techniques that uses the adjoint solution to 
formulate source and transport biasing parameters for MC transport.
More specifically, CADIS determines the parameters for the weight window
technique.


