% Section about analog MC and the challenges faced when trying to calculate SDR in
% FES with no VR.

\subsection{Analog Monte Carlo Calculations}

To quantify the dose rate, we need to know th flux of photons
produced by the decaying radionuclides in the activated components. The
activation is causedd by neutron irradiation.
We need to know what the neutron flux looks like in every region of phase
space.

We need a detailed distribution of the neutron flux throughout all regions of phase
space.
The neutron distribution along with and irradiation and decay scheme of interest 
can then be input for a nuclear inventory code to generate the photon
emission density.  This photon emission density is used as a source for the
photon transport step which ultimately produces the photon flux in every region.
The photon flux coupled with a flux-to-dose-rate conversion factor will give us
the dose rate.


There are two ways to solve the Boltzmann transport equation: deterministically
and stochastically.
The most optimal way to perform accurate, full-scale analysis of FES is
through Monte Carlo radiation transport *need source*.
In general, Monte Carlo (MC) calculations rely on repeated, random sampling to solve
mathematical problems.  The MC method can be applied to radiation transport by
solving the Boltzmann transport equation through the simulation of random particle
walks through phase space.  In analog operation mode (i.e. no variance reduction), 
the source particle's position, energy, direction
and subsequent collisions are sampled from probability
distribution functions (PDFs).  Quantities of interest such as flux can be
scored, or tallied, by averaging particle behavior
in discrete regions of phase space.

Radiation transport simulations in geometries that have thick shielding, such as
FES, become challenging for MC codes.  The particles have lots of interactions
(absorption and scattering) in the shielding regions.  This results in low
particle fluxes in these discrete regions of phase space.  
These scores have a statistical error given by, *add eqn*

The statistical error is a function of the relative error, R, which is defined as
*add eqn* 
where x is the average of the tally scores, S is the standard
deviation of the tally scores, and N is the number of tally scores.  
In order for
accurate, low error, results to be predicted in these regions, many particle
histories need to be simulated, requiring large amounts of computer time
\cite{haghighat_wagner_2003}.
MC calculation efficiency is measured by a quantity known as the figure of merit
(FOM).  The FOM is a function of statistical error and computer processing time.
*add eqn*
The statistical error is a function of the relative error which is defined as
*add eqn* where x is the average of the tally scores, S is the standard
deviation of the tally scores, and N is the number of tally scores.  
To increase
the figure of merit, we can increase the number of tally scores, N, or decrease
the standard deviation, S. VR methods aim to do these two things.  VR methods
aim to increase N and decrease S by sampling from biased PDFs; this effectively
forces more collisions in regions of phase space that are
important to the tally of interest. 
