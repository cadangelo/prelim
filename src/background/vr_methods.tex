\subsection{VR Methods}

Certain MC calculations require the use of variance reduction methods in order
to complete or improve the efficiency of the calculation. 
This is accomplished
by preferentially sampling trajectories that are likely to contribute to the tallies of
interest.
In order to compensate for this biased sampling, the particle statistical weight
is adjusted accordingly.
The relationship between the particle statistical weight, w, and the PDF that
governs particle behavior is as follows
\begin{equation} \label{eq:2.1}
		w_{biased} pdf_{biased} = w_{unbiased} pdf_{unbiased}
\end{equation}

One of the earliest and still commonly used methods of VR is particle splitting
and rouletting.  This is particularly useful in deep penetration simulations,
like FES.  Generally speaking, to increase the number of particle
histories that can contribute to tallies of interest, it is desirable to split
particles as the enter more important regions and roulette particles as the
enter less important regions. 
This requires assigning an importance, I, to every
region in the geometry and adjusting the weight, w, of the new particles. 
When a particle moves from a region A to a region B,
the ratio of importances is calculated.  If region B is more important than
region A such that 
$I_{B}/I_{A} \geqslant 1$, 
the particle with original weight $w_{0}$ is split into 
$n = I_{B}/I_{A}$
particles, each with weight $w_{0}/n$.  If instead region B is less important
than
region A such that
$I_{B}/I_{A} < 1$, 
the particle will undergo roulette. 
The particle will survive with a probability n and weight $w_{0}/n$
\cite{Carter_Cashwell_1975}.
The weight window method in MCNP utilizes both splitting and
rouletting.  A weight window is a region of phase-space that is assigned an upper and
lower bound.  The windows can be assigned to cells in the geometry, on a
superimposed mesh, and to energy bins.  When a particle enters a weight window, its weight is assessed; if
its weight is above the upper bound or below the lower bound, it is either split
or rouletted, respectively.
