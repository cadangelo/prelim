\documentclass{beamer}
\usepackage{tikz}
\usepackage{multicol}
\usetikzlibrary{shapes,arrows}
\usetikzlibrary{shapes.multipart}
\usetikzlibrary{automata,positioning}
\usepackage{animate}
\usepackage{xmpmulti}

\usetheme[white]{Wisconsin}
\begin{document}
\newcommand*{\alphabet}{ABCDEFGHIJKLMNOPQRSTUVWXYZabcdefghijklmnopqrstuvwxyz}
\newlength{\highlightheight}
\newlength{\highlightdepth}
\newlength{\highlightmargin}
\setlength{\highlightmargin}{2pt}
\settoheight{\highlightheight}{\alphabet}
\settodepth{\highlightdepth}{\alphabet}
\addtolength{\highlightheight}{\highlightmargin}
\addtolength{\highlightdepth}{\highlightmargin}
\addtolength{\highlightheight}{\highlightdepth}
\newcommand*{\Highlight}{\rlap{\textcolor{HighlightBackground}{\rule[-\highlightdepth]{\linewidth}{\highlightheight}}}}
%\newcommand{\outlineslide}{\begin{frame}\tableofcontents[currentsection]\end{frame}}
%\newcommand{\outlineslide}{
%\begin{frame}<beamer>
%  \frametitle{Outline}
%  \tableofcontents[currentsection,currentsubsection,hideothersubsections,subsectionstyle=show/shaded]
%\end{frame}
%}

\AtBeginSection[]{
  \begin{frame}
  \vfill
  \centering
  \begin{beamercolorbox}[sep=8pt,center,shadow=true,rounded=true]{title}
    \usebeamerfont{title}\insertsectionhead\par%
  \end{beamercolorbox}
  \vfill
  \end{frame}
}

%\graphicspath{../../src/figs}
%% Slides

%% Title
\title{Generating Variance Reduction Parameters \\ for Shutdown Dose Rate Analysis \\ of Moving Systems}
\date{ Nov. 13, 2018}
\author{Chelsea D'Angelo}
%\institute{Preliminary Exam}
\institute{ANS Winter Meeting}
\begin{frame}
%	\titlepage
	\maketitle
\end{frame}

%\tableofcontents
%\AtBeginSection[]{specialframe}
%\AtBeginSection[]{
	\begin{frame}
                \frametitle{Outline}
		%\begin{multicols}{2}
                \begin{minipage}{\textwidth}
		\tableofcontents%[currentsection]
                \end{minipage}
		%\end{multicols}
	\end{frame}
%}

\section{Motivation}
\begin{frame}{Shutdown Dose Rate (SDR) Analysis}
\begin{columns}
\column{0.5\textwidth}
\begin{itemize}
	\item{Fusion Energy Systems (FES)}
		\begin{itemize}
			\item{Burning plasma, D-T fusion}
			\item{$^{2}_{1}H + ^{3}_{1}H \rightarrow ^{4}_{2}He + ^{1}_{0} n$}
		\end{itemize}
	\item{Neutrons penetrate deeply into system components,
				causing activation}
	\item{Radioisotopes persist long after shutdown}
	\item{Important to quantify the dose caused by decay photons}
\end{itemize}
\column{0.5\textwidth}
\begin{figure}
	\centering
	\includegraphics[scale=0.22]{iter.jpg}
	\caption{Cutaway view of ITER drawing.}
\end{figure}

\end{columns}
		
\end{frame}

%\begin{frame}
%\frametitle{SDR Analysis: Movement After Shutdown}
%	\begin{itemize}
%		\item{FES are designed with modular components}
%	\begin{itemize}
%		\item{Can move during maintenance procedure}
%	\end{itemize}
%		\item{Interested in SDR at a particular
%					location}
%		\item{SDR will change as a function of the
%					activated component's position over
%					time}
%	\end{itemize}
%        \begin{figure}
%        	\centering
%        	\includegraphics[scale=0.3]{config-0.jpg}
%        \end{figure}
%
%\end{frame}

\begin{frame}
\frametitle{SDR Analysis: Movement After Shutdown}
	\begin{itemize}
		\item{FES are designed with modular components}
	\begin{itemize}
		\item{Can move during maintenance procedure}
	\end{itemize}
		\item{Interested in SDR at a particular
					location}
		\item{SDR will change as a function of the
					activated component's position over
					time}
	\end{itemize}
	\begin{center}
	%\animategraphics[loop,controls,width=\textwidth]{5}{config-}{0}{5}
%	\transduration<0-5>{0}
	\multiinclude[<+->][format=jpg, graphics={ scale=0.3}]{config}
	\end{center}
\end{frame}

\begin{frame}{Goal}
	\textbf{Optimize} the %\textbf{radiation transport} simulation used to
	calculation of the \textbf{shutdown dose rate}
	%at a particular location 
        when activated components are
	\textbf{moving} around the facility.
\end{frame}

%\begin{frame}{Computational Radiation Transport}
%Deterministic vs. MC
%\end{frame}
\section{Background}
\subsection{Shutdown Dose Rate Analysis}
\begin{frame}
\frametitle{SDR Solution Method}

%\begin{minipage}{0.47\textwidth}
%\begin{columns}
%		\column{0.45\textwidth}
%	\begin{block}{Direct 1-Step Method (D1S) \cite{d1s}}
%	\vspace{1cm}
%	\begin{figure}
%        \centering
%
%        \tikzstyle{script} = [ellipse, fill=UWGold!30, draw, text centered,
%		node distance=3.0cm]
%        \tikzstyle{code} = [ellipse, fill=UWRed!30, draw, text centered, node
%		distance=3.0cm]
%        \tikzstyle{file} = [rectangle, draw, text centered, node distance=3.0cm]
%        \tikzstyle{line} = [draw, -latex]
%        \tikzstyle{arrow} = [thick, ->, >=stealth]
%        
%	\resizebox{4.9cm}{3.6cm}{
%        \begin{tikzpicture}[every text node part/.style={align=center}]
%		% Optimized neutron transport
%		\node [code] (mc) {Monte Carlo \\ neutron/photon transport};
%                \node [file, above of = mc, yshift=-0.6cm] (mod) {Modified\\ cross sections};
%                \node [file, below of = mc, yshift=0.6cm] (sdr) {Shutdown dose rate};
%		\path [line] (mod) -- (mc);
%		\path [line] (mc) -- (sdr);
%
%	\end{tikzpicture}
%		}
%        \end{figure}
%\end{block}
%%\end{minipage}
%%\hfill
%%\begin{minipage}{0.47\textwidth}
%		\column{0.49\textwidth}
%	\begin{block}{Rigorous 2-Step Method (R2S) \cite{r2s}}
\begin{itemize}
	\item{Rigorous 2-Step Method (R2S) \cite{r2s}}
\end{itemize}
	\vspace{0.3cm}

	\begin{figure}
        \centering
        \tikzstyle{script} = [ellipse, fill=UWGold!30, draw, text centered,
		node distance=3.0cm]
        \tikzstyle{code} = [ellipse, fill=UWRed!30, draw, text centered, node
		distance=3.0cm]
        \tikzstyle{file} = [rectangle, draw, text centered, node distance=3.0cm]
        \tikzstyle{line} = [draw, -latex]
        \tikzstyle{arrow} = [thick, ->, >=stealth]
        
	\resizebox{7.8cm}{5.3cm}{
        \begin{tikzpicture}[every text node part/.style={align=center}]
		% Optimized neutron transport
		\node [code] (nmc) {Monte Carlo \\ neutron transport};
                \node [file, left of = nmc, yshift=-0.9 cm, xshift=-1.0cm] (nflux) {Neutron flux};
		\node [file, right of = nmc, yshift=-0.9cm, xshift=1.0cm]
		(irrdec) {Irradiation and \\ decay scenario};
		\node [script, below of = nmc, yshift=1.2cm] (act) {Activation analysis};
                \node [file, left of = act, yshift=-0.9cm, xshift=-1.0cm] (psrc) {Photon source};
		\node [code, below of = act, yshift=1.2cm] (pmc) {Monte Carlo\\ photon transport};
                \node [file, below of = pmc, yshift=1.2cm] (sdr) {Shutdown dose rate};
		\path [line] (nmc) -- (nflux);
                \path [line] (nflux) -- (act);
                \path [line] (irrdec) -- (act);
                \path [line] (act) -- (psrc);
                \path [line] (psrc) -- (pmc);
		\path [line] (pmc) -- (sdr);

	\end{tikzpicture}
        }
%	\caption{R2S workflow \label{fig:r2s_flow}}
\end{figure}
%\end{block}
%\end{minipage}
%\end{columns}
%Add pros/cons of D1S, R2S
\end{frame}


\begin{frame}
\frametitle{Monte Carlo Radiation Transport}
\begin{columns}
\column{0.7\textwidth}
\begin{itemize}
	\item{Monte Carlo (MC) method \cite{mcnp_manual}:}%analysis of FES is:}
	\begin{itemize}
		\item{Accurate for large, complex models}
		%\item{Stochastic method}
		%\item{Simulate random particle walks through phase space}
	        %\item{Score quantities of interest 
		%in discrete regions of phase space}
		\item{Challenged in highly attenuated regions}
                  \begin{itemize}
                  \item{Results scored in regions that have low particle flux, have higher statistical uncertainty}
                     %\begin{itemize}
                     %  \item{\textbf{Need variance reduction}}
                     %\end{itemize}
                  \end{itemize}
	\end{itemize}
\end{itemize}

\column{0.3\textwidth}
\begin{figure}
	\centering
	\includegraphics[scale=0.5]{iter_pflux.jpg}
	\caption{Photon source in ITER tokamak building.}
\end{figure}

\end{columns}
\end{frame}

%\begin{frame}{Error in MC Calculations}
%\begin{itemize}
%\item{Uncertainty in MC calculations:}
%\begin{equation} \label{eq:1.2}
%		\Re = \frac{\sigma_{\overline{x}}}{{\overline{x}}}
%\end{equation}
%		\begin{center}
%		{$\sigma_{\overline{x}} \propto \frac{1}{\sqrt{N}}$}%, N is number of histories} %\cite{mcnp_manual}.
%		\end{center}
%\item{Efficiency in MC calculations:}
%\begin{equation} \label{eq:1.3}
%	FOM = \frac{1}{{{{\Re}^2}t_{proc}}}
%\end{equation}
%
%\item{To decrease uncertainty:}
%  \begin{itemize}
%  \item{Increase number of histories, N}
%  \item{Use variance reduction (VR) techniques}
%  \end{itemize}
%\end{itemize}
%\end{frame}

\subsection{Variance Reduction}
\begin{frame}{MC Variance Reduction Techniques}
\begin{itemize}
\item{Techniques to modify particle behavior}
   \begin{itemize}
	\item{\textbf{Goal:} preferentially sample events that will contribute to results of interest}
   \end{itemize}
\item{Adjust statistical weight of particles to keep playing a fair game}
\item{Types}
	\begin{itemize}
		\item{\textbf{Modified Sampling:} source biasing}
		\item{\textbf{Population Control:} splitting/rouletting}
	\end{itemize}
%                \begin{figure}
%		%\centering
%			\vspace{0.5cm}
%			\hspace{-1cm}
%		\includegraphics[scale=0.44]{vr_types.jpg}
%			\caption{a) analog b) source biasing c)
%			splitting/rouletting \cite{mcnp_primer}}
%		\end{figure}
\end{itemize}
\end{frame}

%\begin{frame}{Hybrid Deterministic/MC VR Methods: CADIS}
\begin{frame}{Hybrid Deterministic/MC VR Methods}
%    \begin{itemize}
%\end{itemize}
\begin{itemize}
\item{Use \textbf{deterministic} estimate of the adjoint flux, $\Psi^+$, to
		generate \textbf{Monte Carlo} VR parameters} %in a \textbf{consistent} manner}
\item{Adjoint flux can define the importance of regions of phase space to the detector response}
\item{Define adjoint source to be detector response function}
\end{itemize}
\vspace{0.5cm}
%Forward particle transport
\textbf{Forward:}
\begin{equation} \label{eq:3.1a}
	 H\Psi(\overrightarrow{r}, E,\widehat{\Omega})  = q(\overrightarrow{r}, E,\widehat{\Omega})
\end{equation}
%\begin{equation} \label{eq:3.1b}
	\begin{center}
		$H = \widehat{\Omega} \cdot \nabla +
		    \sigma_{t}(\overrightarrow{r},E) - 
			\int_{0}^{\infty} dE'
			\int_{4\pi} d\Omega'
			\sigma_{s}( \overrightarrow{r}, E' 
			\rightarrow E, \widehat{\Omega}' 
			\rightarrow \widehat{\Omega} )$
	\end{center}
%\end{equation}
	\vspace{0.5cm}
%Adjoint particle transport
\textbf{Adjoint:}
\begin{equation} \label{eq:3.2b}
		\langle \Psi^{+} , H\Psi \rangle =
		\langle \Psi, H^+\Psi^{+} \rangle
\end{equation}
       \begin{equation}
	 %       H^+\Psi^+ = q^+
	 H^+ \Psi^+(\overrightarrow{r}, E,\widehat{\Omega})  = q^+(\overrightarrow{r}, E,\widehat{\Omega})
       \end{equation} 
       %\begin{equation}
	\begin{center}
	$H^{+} = -\widehat{\Omega} \cdot \nabla +
	    \sigma_{t}(\overrightarrow{r},E) - 
		\int_{0}^{\infty} dE'
		\int_{4\pi} d\Omega'
		\sigma_{s}( \overrightarrow{r}, E 
		\rightarrow E', \widehat{\Omega} 
		\rightarrow \widehat{\Omega}' )$
       \end{center}
%       \end{equation}
%    \begin{itemize}
    %\item{Adjoint transport $\sim$ "backwards" transport}
%    \end{itemize}


\end{frame}

%\begin{frame}{Hybrid Deterministic/MC VR Methods}
%  \begin{itemize}
%  \item{Use the adjoint flux, $\phi^+$, to generate MC source and transport biasing
%	  parameters}
%  \end{itemize}
%	\begin{columns}
%        \column{0.45\textwidth}
%			\begin{block}%{Uniform vs. biased source distribution}
%                                     {Source: biased PDF}
%               \begin{figure}
%		\centering
%		\includegraphics[scale=0.5]{uni_bias_src.jpg}
%		\end{figure}
%			\end{block}
%        \column{0.45\textwidth}
%                \begin{block}{Transport: weight windows}
%			\vspace{0.8cm}
%                \begin{figure}
%		\centering
%		\includegraphics[scale=0.51]{ww_ex2.png}
%		\end{figure}
%			\vspace{0.56cm}
%                \end{block}
%	\end{columns}
%\end{frame}

\begin{frame}
	\frametitle{Hybrid Deterministic/MC VR Methods: CADIS \cite{cadis}}
	\begin{block}{\textbf{C}onsistent \textbf{A}djoint \textbf{D}riven
	\textbf{I}mportance \textbf{S}ampling (CADIS)}
  \begin{itemize}
  \item{Use the adjoint flux, $\Psi^+$, to generate MC VR parameters}
%    \item{Particles are born within weight windows}
  \item{Biased source:}
\begin{equation} \label{eq:3.8}
	\widehat{q}(\overrightarrow{r}, E, \widehat{\Omega}) =
	\frac{\Psi^{+}(\overrightarrow{r}, E,\widehat{\Omega})
	q(\overrightarrow{r}, E, \widehat{\Omega})}{R}
\end{equation}

  \item{Weight window lower bounds:}
\begin{equation} \label{eq:3.12}
	w_{l}(\overrightarrow{r}, E, \widehat{\Omega}) = 
	\frac{R}{\Psi^{+}(\overrightarrow{r}, E, \widehat{\Omega})
	(\frac{\alpha + 1}{2})}
\end{equation}
  \end{itemize}
\end{block}
\end{frame}

\begin{frame}
\frametitle{Variance Reduction for SDR Analysis}
%\begin{itemize}
		\begin{block}{VR for \textbf{photon} transport}
  \begin{itemize}
  \item{\textbf{Straightforward}}
  \item{Can use CADIS method to direct photons towards detector}
    \begin{itemize}
    \item{Flux-to-dose-rate conversion factors define adjoint source}
    \end{itemize}
  \end{itemize}
	\end{block}
		\begin{block}{VR for \textbf{neutron} transport}
    \begin{itemize}
    \item{More \textbf{complicated}}
    \item{Biasing function needs to capture}
      \begin{enumerate}
        \item{Potential of regions to become activated}
        \item{Potential to produce photons that will contribute to the SDR}
      \end{enumerate}
    \item{Can use CADIS if we can construct adjoint source that will fulfill
	    these criteria}
    \end{itemize}
\end{block}
%\end{itemize}
\end{frame}


\begin{frame}{Variance Reduction for SDR Analysis: MS-CADIS}
	\begin{block}{\textbf{M}ulti-\textbf{S}tep (MS)-CADIS \cite{mscadis}}
  \begin{itemize}
  \item {VR method to optimize the initial radiation transport step of a
	  coupled, multi-step process}
		 % \begin{itemize}
		 %         \item{Relies upon biasing function that represents importance
		 %       	  of particles to final response of
		 %       	  interest}
		 % \end{itemize}
	  
	  \item{When applied to SDR analysis, MS-CADIS optimizes the
		  neutron transport step of R2S}
%    \begin{itemize}
%    \item{Use function that represents the   importance of the neutrons to
%	    the final shutdown dose rate}
%    \end{itemize}
%  \end{itemize}
\end{itemize}
\end{block}
% \centering
%	\begin{equation}
%	\int_{\overrightarrow{r}} \int_{E_n} \phi_n(\overrightarrow{r}, E_n)
%	q_n^+(\overrightarrow{r}, E_n) d\overrightarrow{r} dE_n = SDR
%	\end{equation}
%% \vspace{0.03 cm}
%
% \centering
%	\begin{equation}
%		SDR = \int_{\overrightarrow{r}} \int_{E_{\gamma}} 
%		\phi_{\gamma}^+(\overrightarrow{r}, E_{\gamma})
%		q_{\gamma}(\overrightarrow{r}, E_{\gamma})
%		dr dE_{\gamma} 
%	\end{equation}
\end{frame}

%\begin{frame}{Generalized MS-CADIS}
%%Back up a bit and look at MS-CADIS in generic fashion
%\begin{itemize}
%  \item{System of coupled, multi-physics:}
%
%	\begin{center}
%          \begin{equation}
%		Primary:	H\phi(u) = q(u)
%          \end{equation}
%		\vspace{-0.4cm}          
%          \begin{equation}
%		 Secondary:	L\psi(v) = b(v)
%          \end{equation}
%	\end{center}
%
%	\begin{center}
%		\vspace{0.4cm}
%		$b(v) = f(\phi(u))$
%	\end{center}
%  \pause
%  \item{Adjoint identities:}
%	\begin{center}
%          \begin{equation}
%          	\langle \phi^{+}, q \rangle =
%          	\langle \phi, q^{+} \rangle 
%          \end{equation}
%		\vspace{-0.4cm}          
%          \begin{equation}
%          	\langle \psi^{+}, b \rangle =
%          	\langle \psi, b^{+} \rangle 
%          \end{equation}
%	\end{center}
%  %\item{MS-CADIS requires a representation of the relationship between primary
%%	  and secondary physics:}
%%	\begin{center}
%%          \begin{equation}\label{eq:coupling}
%%            b(v) = \langle \sigma_b(u,v), \phi(u) \rangle
%%          \end{equation}
%%        \end{center}
%\end{itemize}
%\end{frame}
%
%\begin{frame}{Generalized MS-CADIS}
%\begin{itemize}
%\item{Response to secondary physics:}
%\begin{equation} \label{eq:response}
%  R_{final} = \langle \omega_R(v), \psi(v) \rangle
%\end{equation}
%\pause
%\item{Define  $b^+ \equiv \omega_R$ and apply adjoint
%	identity:}
%\begin{equation}
%  R_{final} = \langle \omega_R, \psi \rangle 
%	= \langle b^+, \psi \rangle
%	= \langle b, \psi_R^{+}\rangle
%\end{equation}
%		\vspace{-0.4cm}          
%		\begin{itemize}
%		\item{$\psi_R^+$ represents importance function for $R_{final}$}
%		\end{itemize}
%		\vspace{0.4cm}         
%\pause
%\item{Set primary response to final response and apply adjoint
%	identity:}
%	\begin{equation}
%		R_{final} = \langle q^+, \phi \rangle
%		          = \langle q, \phi_R^+ \rangle
%	\end{equation}
%		\vspace{-0.4cm}          
%		\begin{itemize}
%		\item{$\phi_R^+$ represents importance function for $R_{final}$}
%		\end{itemize}
%		\vspace{0.4cm}         
%\pause
%\item{Solving for $q^+$ requires this unique relationship:}
%          \begin{equation}\label{eq:coupling}
%            b(v) = \langle \sigma_b(u,v), \phi(u) \rangle
%          \end{equation}
%
%
%\end{itemize}
%\end{frame}
%
%\begin{frame}{Generalized MS-CADIS}
%\begin{itemize}
%\item{Substitute Eq. \ref{eq:coupling} and set primary response equal to secondary :}
%\begin{equation}
%	R_{final} = \langle q^+(u), \phi(u) \rangle
%	=\left \langle\  \langle \sigma_b(u,v) , \phi(u) \rangle,\
%	\psi_R^{+}(v) \ \right\rangle
%\end{equation}
%\pause
%	\item{Switch the order of integration}
%\begin{equation}\label{eq:pseudo-response}
%	R_{final} = \langle q^+(u), \phi(u) \rangle
%	=\left \langle \ \langle \sigma_b(u,v) , \psi_R^{+}(v) \rangle,\
%	\phi(u) \ \right\rangle
%\end{equation}
%		\vspace{0.2cm}
%\pause
%\item{MS-CADIS adjoint primary source:}
%\begin{equation}
%  q^{+}(u) \equiv \langle \sigma_b(u,v) , \psi_R^{+}(v) \rangle
%\end{equation}
%\end{itemize}
%\end{frame}
%
%\begin{frame}{Generalized MS-CADIS}
%\begin{itemize}
%\item{MS-CADIS adjoint primary source:}
%\begin{equation}
%  q^{+}(u) \equiv \langle \sigma_b(u,v) , \psi_R^{+}(v) \rangle
%\end{equation}
%\item{Apply to coupled neutron-photon physics:}
%	\begin{itemize}
%		\item{$q^+(u) \equiv q_n^+(E_n)$}
%			\vspace{0.1cm}
%		\item{$\psi^+(v) \equiv \phi_{\gamma}^+(E_{\gamma})$}
%			\vspace{0.1cm}
%	\item{Prompt photon production: $\sigma_b(u,v) \equiv \sigma_{n,\gamma}(E_n,
%	E_{\gamma})$}
%			\vspace{0.1cm}
%	\item{Delayed photon production: $\sigma_b(u,v) \equiv 
%	T_{n,\gamma}(E_n,E_{\gamma})$}
%%		\begin{itemize}
%%			\item{T_{n,\gamma}(E_n,E_{\gamma}) }
%%		\end{itemize}
%	\end{itemize}
%
%\end{itemize}
%\end{frame}

%\begin{frame}
%\frametitle{Variance Reduction for SDR Analysis: MS-CADIS}
%	\begin{itemize}
%		\item{Combining these equations:}
%	\end{itemize}
%	\begin{centering}
%	\begin{equation}
%	\int_{\overrightarrow{r}} \int_{E_n} \phi_n(\overrightarrow{r}, E_n)
%	q_n^+(\overrightarrow{r}, E_n) d\overrightarrow{r} dE_n =
%		\int_{\overrightarrow{r}} \int_{E_{\gamma}} 
%		\phi_{\gamma}^+(\overrightarrow{r}, E_{\gamma})
%		q_{\gamma}(\overrightarrow{r}, E_{\gamma})
%		d\overrightarrow{r} dE_{\gamma} 
%	\end{equation}
%	\end{centering}
%	\begin{itemize}
%		\item{To solve for the adjoint neutron source, $q_n^+$, a
%			relationship between $q_{\gamma}$
%			and $\phi_n$ is required}
%	\end{itemize}
%\centering
%\begin{equation}
%	q_{\gamma}(E_{\gamma}) = \int_{E_n} T(E_n, E_{\gamma}) \phi_n(E_n) dE_n
%\end{equation}
%
%\end{frame}
		
\begin{frame}{Variance Reduction for SDR Analysis: GT-CADIS}
\begin{block}{\textbf{G}roupwise \textbf{T}ransmutation (GT)-CADIS \cite{gtcadis}}
  \begin{itemize}
  \item{Implementation of MS-CADIS specifically for SDR analysis}
  \item{Provides method to calculate optimal adjoint neutron source, $q_n^+$:}
\begin{equation}
	q^{+}_n(E_n) = \langle T(E_n, E_{\gamma}) ,
	\phi_{\gamma}^{+}(E_{\gamma}) \rangle
\end{equation}
\item{$T(E_n, E_{\gamma})$ }%is a term that relates the neutron flux to		  photon source}
		  \begin{itemize}
			  \item{Approximation of the transmutation
				  process}
			  %\item{Solution exits when SNILB criteria
				%  are met}
			  \item{Defined by this relationship:}
		  \end{itemize}
\begin{equation} \label{eq:Tdef}
	q_{\gamma}(E_{\gamma}) = 
	\langle T(E_{n}, E_{\gamma}),
	\phi_{n}(E_{n}) \rangle
\end{equation}

  \end{itemize}
\end{block}
%
%\end{itemize}
%\centering
%
\end{frame}

%\begin{frame}{Variance Reduction for SDR Analysis: GT-CADIS}
%\begin{block}{\textbf{G}roupwise \textbf{T}ransmutation (GT)-CADIS \cite{gtcadis}}
%  \begin{itemize}
%	  \item{Calculate T:}
%	  \begin{enumerate}
%		  \item{Irradiate each material with neutrons from a single
%			  energy group, g}
%		  \item{Record resulting photon emission in each energy group,
%			  h}
%	  \end{enumerate}
%  \end{itemize}
%	\begin{equation}
%		T_{g,h} = \frac{q_{\gamma,h}(\phi_{n,g})}{\phi_{n,g}}
%	\end{equation}
%\end{block}
%\end{frame}


%\begin{frame}
%\frametitle{Variance Reduction for SDR Analysis: GT-CADIS}
%
%\begin{itemize}
%	\item{MOVE:Use T to solve for adjoint neutron source:}
%\end{itemize}
%\centering
%\begin{equation}
%	q_n^+(E_n)=\int_{E_{\gamma}} T(E_n, E_{\gamma}) \phi_{\gamma}^+(E_{\gamma})
%	dE_{\gamma}
%\end{equation}

%\begin{itemize}
%\item{This is used as the source in the deterministic calculation of the    adjoint neutron flux, $\phi_{n}^{+}$}
%\end{itemize}
%\end{frame}

%\subsection{Demonstration}
\begin{frame}{GT-CADIS Demonstration: Problem Description}
	\begin{columns}
        \column{0.48\textwidth}
           \begin{itemize}
		   \item{Geometry}
			   \begin{itemize}
				   \item{Steel chamber}
				   \item{2m x 2m x 2m central cavity}
			   \end{itemize}
		   \item{Source}
			   \begin{itemize}
				   \item{Uniform volume source in central cavity}
				   \item{13.8-14.2 MeV neutrons}
			   \end{itemize}
		   \item{Detector}
			   \begin{itemize}
			   \item{Calculate SDR}
			   \item{2m away from chamber}
			   \end{itemize}
	   \end{itemize}
        \column{0.48\textwidth}
		\begin{figure}
		\centering
		\includegraphics[scale=0.40]{orig_geom_pres.png}
                \end{figure}
	\end{columns}
\end{frame}

\begin{frame}{GT-CADIS Demonstration: Adjoint Photon Transport}
	\begin{columns}
		\column{0.4\textwidth}
		\hspace{-1.5cm}
\centering
GT-CADIS workflow
\begin{figure}

        \tikzstyle{script} = [ellipse, fill=UWGold!30, draw, text centered,
	node distance=1.3cm]
        \tikzstyle{code} = [ellipse, fill=UWRed!30, draw, text centered, node
	distance=1.3cm]
        \tikzstyle{file} = [rectangle, draw, text centered, node distance=1.3cm]
        \tikzstyle{line} = [draw, -latex]
        \tikzstyle{arrow} = [thick, ->, >=stealth]
       
	\resizebox{!}{4.5cm}{
        \begin{tikzpicture}[every text node part/.style={align=center}]
		% Optimized neutron transport
		\node [script] (adjnsrc) {Calculate $q_n^+$};
		\node [code, above of = adjnsrc, xshift=-0.9cm, yshift=1.1cm] (adjp) {Adjoint photon transport};
		%\node [file] (adjpflux) {Adjoint photon flux};
		\node [script, above of = adjnsrc, xshift=1.5cm] (tcalc) {Calculate T};
		\node [code, below of = adjnsrc ] (adjn) {Adjoint neutron transport};
		%\node [file] (adjnflux) {Adjoint neutron flux};
		\node [script, below of = adjn] (vr) {Calculate MC VR};
		\node [file, below of = vr, xshift =-2cm] (ww) {Weight windows};
		\node [file, below of = vr, xshift=2cm] (bsrc) {Biased source};

		\path [line] (adjp) -- (adjnsrc);
		\path [line] (tcalc) -- (adjnsrc);
		\path [line] (adjnsrc) -- (adjn);
		\path [line] (adjn) -- (vr);
		\path [line] (vr) -- (ww);
		\path [line] (vr) -- (bsrc);

	\end{tikzpicture}
	}%resize
\end{figure}

		\column{0.5\textwidth}
\begin{figure}
	\includegraphics[scale=0.185]{gtcadis_adj_p_g41.png}
	\caption [GT-CADIS adjoint photon flux] 
	{Adjoint photon flux\label{fig:ex.adj_p_flux}}
\end{figure}

	\end{columns}

\end{frame}

\begin{frame}{GT-CADIS Demonstration: Adjoint Neutron Transport}
	\begin{columns}
		\column{0.4\textwidth}
		\hspace{-1.5cm}
\centering
GT-CADIS workflow
\begin{figure}

        \tikzstyle{script} = [ellipse, fill=UWGold!30, draw, text centered,
	node distance=1.3cm]
        \tikzstyle{code} = [ellipse, fill=UWRed!30, draw, text centered, node
	distance=1.3cm]
        \tikzstyle{file} = [rectangle, draw, text centered, node distance=1.3cm]
        \tikzstyle{line} = [draw, -latex]
        \tikzstyle{arrow} = [thick, ->, >=stealth]
       
	\resizebox{!}{4.5cm}{
        \begin{tikzpicture}[every text node part/.style={align=center}]
		% Optimized neutron transport
		\node [script] (adjnsrc) {Calculate $q_n^+$};
		\node [code, above of = adjnsrc, xshift=-0.9cm, yshift=1.1cm] (adjp) {Adjoint photon transport};
		%\node [file] (adjpflux) {Adjoint photon flux};
		\node [script, above of = adjnsrc, xshift=1.5cm] (tcalc) {Calculate T};
		\node [code, below of = adjnsrc ] (adjn) {Adjoint neutron transport};
		%\node [file] (adjnflux) {Adjoint neutron flux};
		\node [script, below of = adjn] (vr) {Calculate MC VR};
		\node [file, below of = vr, xshift =-2cm] (ww) {Weight windows};
		\node [file, below of = vr, xshift=2cm] (bsrc) {Biased source};

		\path [line] (adjp) -- (adjnsrc);
		\path [line] (tcalc) -- (adjnsrc);
		\path [line] (adjnsrc) -- (adjn);
		\path [line] (adjn) -- (vr);
		\path [line] (vr) -- (ww);
		\path [line] (vr) -- (bsrc);

	\end{tikzpicture}
	}%resize
\end{figure}

		\column{0.5\textwidth}
        \begin{figure}
	\centering
	%\includegraphics[scale=0.20]{gtcadis_adjn.jpg}
	\includegraphics[scale=0.22]{gtcadis_adjn.jpg}
		\caption{GT-CADIS adjoint neutron flux. Functions as importance
		map.}
	\end{figure}

	\end{columns}

\end{frame}

%\begin{frame}{GT-CADIS Demonstration: VR Parameters}
%	\begin{columns}
%		\column{0.47\textwidth}
%\begin{figure} 
%	\includegraphics[scale=0.17]{biased_n_src.png}
%	\caption [GT-CADIS biased neutron source] 
%	{Biased neutron source generated with GT-CADIS method.\label{fig:ex.biased_src}}
%\end{figure}
%
%
%		\column{0.47\textwidth}
%\begin{figure} 
%	\includegraphics[scale=0.15]{gtcadis_wwn.png}
%	\caption [GT-CADIS weight window mesh]
%	{Weight window mesh generated with GT-CADIS method.\label{fig:ex.wwinp}}
%\end{figure}
%
%	\end{columns}
%\end{frame}

%\begin{frame}{GT-CADIS Demonstration: Forward Neutron Flux}
%	\begin{columns}
%		\column{0.47\textwidth}
%\begin{figure} 
%	\includegraphics[width=6.1cm] {analog_tot_n_f.png}
%%	\includegraphics[scale=0.4] {figs/analog_tot_n_err.png}
%	\caption [Analog neutron flux.] 
%	{Neutron flux resulting from analog MC simulation.\label{fig:ex.ana_nflux}}
%\end{figure}
%		\column{0.47\textwidth}
%		\vspace{0.6cm}
%\begin{figure} 
%	\includegraphics[width=6.1cm]{gtcadis_tot_n_f.png}
%	\caption [GT-CADIS neutron flux] 
%	{Neutron flux resulting from MC simulation using
%	 GT-CADIS biased source and weight window mesh.\label{fig:ex.gt_nflux}}
%\end{figure}
%	\end{columns}
%\end{frame}
%
%\begin{frame}{GT-CADIS Demonstration: Forward Photon Source}
%	\begin{columns}
%		\column{0.47\textwidth}
%\begin{figure} 
%	\includegraphics[scale=0.15]{analog_p_src_g1.png}
%	\caption [Analog photon source]
%	{Photon source generated by ALARA activation calculation using the
%	 analog MC neutron transport result.\label{fig:ex.analog_psrc}}
%\end{figure}
%		\column{0.47\textwidth}
%
%\begin{figure} 
%	\includegraphics[scale=0.15]{gtcadis_photon_src_g1.png}
%	\caption [GT-CADIS photon source]
%	{Photon source generated after ALARA activation calculation using the
%	GT-CADIS optimized neutron transport result.\label{fig:ex.gt_psrc}}
%\end{figure}
%	\end{columns}
%\end{frame}

%\begin{frame}{GT-CADIS Efficiency}
%	\begin{itemize}
%		\item{VR parameters produced by GT-CADIS method result in much
%			faster convergence of the neutron transport flux in
%			comparison to  analog and FW-CADIS methods}
%	\end{itemize}
%\begin{figure} 
%	\includegraphics[scale=0.35]{gt_fom.jpg}
%	\caption {FOM as function of neutron transport processor time.
%	\cite{eb_thesis}}
%\end{figure}
%
%\end{frame}

\begin{frame}
\frametitle{GT-CADIS Demonstration}
	GT-CADIS importance map is \textbf{insufficient} for moving systems.

	\begin{columns}
        \column{0.5\textwidth}
        \begin{figure}
%	\centering
	\vspace{-0.9cm}
		\hspace{-1cm}
%	\includegraphics[scale=0.27, xshift=-3cm]{config-0.jpg}
%	\transduration<0-5>{0}
	\multiinclude[<+->][format=jpg, graphics={ scale=0.22}]{config}
	%	\vspace{0.3cm}
		\caption{Demo model.}
        \end{figure}

        \column{0.5\textwidth}
        \begin{figure}
	\centering
	\includegraphics[scale=0.20]{gtcadis_adjn_hi.jpg}
		\caption{GT-CADIS adjoint neutron flux. Functions as importance
		map.}
	\end{figure}
	\end{columns}

\end{frame}



%\subsection{Moving Geometries and Sources}
%\begin{frame}{Moving Geometries and Sources}
%	\begin{block}{MCNP6 Moving Objects \cite{mcnp_moving_1},
%		\cite{mcnp_moving_2}}
%	\begin{itemize}
%		\item{Update in future version of MCNP6}
%		\item{Allows movement of objects, sources, delayed particles
%			during single simulation}
%		\item{Available for native MCNP geometry descriptions (not
%			mesh)}
%	\end{itemize}
%	\end{block}
%\end{frame}
%
%\begin{frame}{Moving Geometries and Sources}
%	\begin{block}{Mesh Coupled R2S (MCR2S)\cite{mcr2s}}
%	\begin{itemize}
%		\item{Capability that allows components to move before photon
%			transport step}
%		\item{Transformations are applied to copies of moving
%			components}
%		\item{Original component still in original location, set to void material}
%	\end{itemize}
%		\end{block}
%\end{frame}

%\begin{frame}{Review}
%	\begin{itemize}
%		\item{MC method is most accurate way to obtain detailed
%			particle flux distributions}
%			\begin{itemize}
%          	  	  \item{Use MC codes for both neutron and photon transport steps
%          	  	  	of R2S}
%          	  	  \item{Need to use VR methods to optimize the transport
%          	  	  	calculations}
%          		\end{itemize}
%		\item{GT-CADIS has proven to optimize the neutron transport
%			step of R2S}
%%		\item{MCNP6 and MCR2S have developed some capabilities for
%%			performing transport on moving geometries}
%		\item{No automated VR for optimizing neutron transport
%			in systems that move after shutdown}
%	\end{itemize}
%
%%	\begin{itemize}
%%		\item{This work aims to combine and advance these efforts
%%			by...}
%%		\item{MCNP simulation of moving CAD-based geometries }
%%		\item{VR technique to optimize initial radiation transport of
%%			coupled, multi-physics processes in moving systems}
%%	\end{itemize}
%\end{frame}



%\section{Proposal}
\section{Methodology}

%\begin{frame}{VR for Multi-physics Analysis of Moving Systems}
%	\begin{itemize}
%		\item{MS-CADIS optimizes initial radiation transport step in a coupled, multi-step
%			process}
%		\item{Movement during secondary step changes the construction
%			of the adjoint primary source}
%	\end{itemize}
%	\vspace{1cm}
%	\centering
%		{\textbf{Need to derive new adjoint source}}
%
%\end{frame}
\subsection{Analysis of Moving Systems}
\begin{frame}{VR for SDR Analysis of Moving Systems}
	\begin{itemize}
		\item{GT-CADIS optimizes neutron transport step in static
			systems}
			\begin{center}
	$q^{+}_n(E_n) = \langle T(E_n, E_{\gamma}) ,
	\phi_{\gamma}^{+}(E_{\gamma}) \rangle$
			\end{center}
			\vspace{0.5cm}
		\item{Movement after shutdown, during photon transport:}
			%, changes the construction of the 
			%adjoint primary source}
	\end{itemize}
	\vspace{0.5cm}
	\centering
		{\textbf{Need time-integrated adjoint neutron source}}

\end{frame}


\begin{frame}{Time-integrated Adjoint Neutron Source}
	\begin{itemize}
		\item{Score adjoint photon flux in discrete volume elements at
			each time step}
	   \begin{itemize}
	   \item	{Adjoint flux in volume element v at time t: 
		   $\phi_{\gamma}^{+}(\overrightarrow{r}_{v}(t), t)$ }
	   \item  {Position of volume element v at time t:
	   	$\overrightarrow{r}_{v}(t)$ }
	   \end{itemize}
   \item{ Combine with T and integrate over time}
        \begin{equation}\label{eq:adj_src_1_avg}
		q_{n}^{+} = \frac
		{\int_{t}  \phi_{\gamma}^{+}(\overrightarrow{r}_{v}(t), t)
		 T_v(t)\, dt}
		 {\int_t dt}
        \end{equation}
	\end{itemize}
%\end{frame}

%\begin{frame}

%	\includegraphics[scale=0.27, xshift=-3cm]{mesh_config-5.jpg}
%	\vspace{-1.9cm}
%	\begin{center}
	%\animategraphics[loop,controls,width=\textwidth]{5}{config-}{0}{5}
	%\transduration<0-5>{0}
%	\multiinclude[<+->][format=jpg, graphics={ scale=0.3}]{mesh_config}
%	\end{center}

	\begin{columns}
		\begin{column}{0.3\textwidth}
        \begin{figure}
	\centering
		\vspace{-0.4cm}
	\includegraphics[scale=0.20]{mesh-0.jpg}
	\end{figure}
\end{column}
\vrule{}
		\begin{column}{0.3\textwidth}
		\vspace{0.6cm}
        \begin{figure}
	\centering
	\includegraphics[scale=0.20]{mesh-3.jpg}
	\end{figure}
\end{column}
\vrule{}
		\begin{column}{0.3\textwidth}
		\vspace{-0.4cm}
        \begin{figure}
	\centering
	\includegraphics[scale=0.20]{mesh-5.jpg}
	\end{figure}
\end{column}
	\end{columns}
\end{frame}


%\begin{frame}{TGT-CADIS: Adjoint Neutron Source}
%%	\begin{itemize}
%%		\item{Average the adjoint photon flux calculated at each time step}
%%	\end{itemize}
%%\begin{equation}\label{eq:sum}
%%	\phi_{\gamma,v, h}^{+} =
%%	\frac{\sum_{t_{mov}}{\phi_{\gamma,v,h,t_{mov}}^{+}}\Delta{t_{mov}}}
%%	{t_{tot}}
%%\end{equation}
%	\begin{itemize}
%         % \item{Calculate coupling term, T, for each volume element}
%
%	 %   \begin{equation}\label{eq:T}
%	 % 	T_{v,g,h} = \dfrac{q_{\gamma,v,h}(\phi_{n,v,g})}{\phi_{n,v,g}}
%	 %   \end{equation}
%
%	  %\item{Combine with adjoint photon flux in each volume element}
%	  \item{Discrete form:}
%		  \begin{itemize}
%			  \item{$t_{mov}$ is the time step}
%			  \item{$\Delta t_{mov}$ is the duration of the time step}
%			  \item{$t_{tot}$ is the total number of time
%				  steps}
%		  \end{itemize}
%
%	    \begin{equation}\label{eq:tgt_n_src}
%	    	q_{n,v,g}^{+} = \frac
%	    	{\sum_{t_{mov}}\big(\sum_{h} T_{v,g,h} \
%	    	\phi_{\gamma,v,h,t_{mov}}^{+}\big) \Delta t_{mov}}
%	    	{t_{tot}}
%	    \end{equation}
%	\end{itemize}
%%        \begin{figure}
%%		\vspace{0.3cm}
%%	\includegraphics[scale=0.280]{mesh_total.jpg}
%%	\end{figure}
%
%	\begin{columns}
%		\begin{column}{0.3\textwidth}
%        \begin{figure}
%	\centering
%		\vspace{-0.4cm}
%	\includegraphics[scale=0.20]{mesh-0.jpg}
%	\end{figure}
%\end{column}
%\vrule{}
%		\begin{column}{0.3\textwidth}
%		\vspace{0.6cm}
%        \begin{figure}
%	\centering
%	\includegraphics[scale=0.20]{mesh-3.jpg}
%	\end{figure}
%\end{column}
%\vrule{}
%		\begin{column}{0.3\textwidth}
%		\vspace{-0.4cm}
%        \begin{figure}
%	\centering
%	\includegraphics[scale=0.20]{mesh-5.jpg}
%	\end{figure}
%\end{column}
%	\end{columns}
%
%\end{frame}

%\begin{frame}{Time-integrated Adjoint Neutron Source}
%\end{frame}

%\begin{frame}{Time-integrated (T)GT-CADIS}
%	\begin{itemize}
%		\item{Perform deterministic adjoint neutron transport using the
%			time-integrated source}
%		\item{Resultant adjoint neutron flux should look something like
%			this:}
%	\end{itemize}
%        \begin{figure}
%	\centering
%	\includegraphics[scale=0.20]{tgt_adj_n.jpg}
%%		\caption{GT-CADIS adjoint neutron flux. Functions as importance
%%		map.}
%	\end{figure}
%%	\begin{itemize}
%%		\item{Use this adjoint neutron flux to generate biasing
%%			parameters that will optimize the MC neutron transport
%%			step of R2S}
%%	\end{itemize}
%\end{frame}

%\begin{frame}{Time-integrated (T)GT-CADIS: Workflow}
%%	\begin{itemize}
%%		\item{To apply time-integration to GT-CADIS:}
%%			\begin{enumerate}
%%				\item{Perform adjoint photon transport at each
%%					time step of geometry movement}
%%				\item{Score adjoint photon flux in each volume
%%					element, v}
%%				\item{ over time}
%%			\end{enumerate}
%%	\end{itemize}
%%\begin{equation}\label{eq:adj_src_1_avg}
%%	 q_{n,v}^{+}(E_{n}) =\frac
%%	 {\int_{t}  \int_{E_{\gamma}}
%%	 T_{v}(E_n, E_{\gamma}, t) 
%%	 \phi_{\gamma}^{+}(\overrightarrow{r}_{v}(t), E_{\gamma},t)
%%	 \, dE_{\gamma} \, dt}
%%	 {\int_t dt}
%% \end{equation}
%
%%\begin{itemize}
%%\item{$\phi_{\gamma}^{+}(\overrightarrow{r}_{v}(t), E_{\gamma},t) $ is the 
%%adjoint flux of photons of energy $E_{\gamma}$, in volume element v, at time t}
%%\item{$T_{v}(E_n, E_{\gamma}, t) $ is the T value of the material in volume
%%		element v, at decay time t}
%%\end{itemize}
%\begin{figure}
%\centering
%
%        \tikzstyle{script} = [ellipse, fill=UWGold!30, draw, text centered,
%	node distance=1.3cm]
%        \tikzstyle{code} = [ellipse, fill=UWRed!30, draw, text centered, node
%	distance=1.3cm]
%        \tikzstyle{file} = [rectangle, draw, text centered, node distance=1.3cm]
%        \tikzstyle{line} = [draw, -latex]
%        \tikzstyle{arrow} = [thick, ->, >=stealth]
%        %gray
%        \tikzstyle{gscript} = [ellipse, fill=gray!30, draw, text centered,
%	node distance=1.3cm]
%        \tikzstyle{gcode} = [ellipse, fill=gray!30, draw, text centered, node
%	distance=1.3cm]
%        \tikzstyle{gfile} = [rectangle, fill=gray!30, draw, text centered, node distance=1.3cm]
%       
%	\resizebox{!}{5.7cm}{
%        \begin{tikzpicture}[every text node part/.style={align=center}]
%		% Optimized neutron transport
%		\node [script] (adjnsrc) { Calculate $q_n^+$};
%		\node [code, above of = adjnsrc, xshift=-3cm, yshift=0.4cm]
%		(adjp) {Adjoint photon transport \\ at each $t_{mov}$};
%		%\node [file] (adjpflux) {Adjoint photon flux};
%		\node [script, above of = adjnsrc, xshift=3cm, yshift=0.1cm] (tcalc)
%		{Calculate $T$};
%		\node [code, below of = adjnsrc, yshift=-0.5cm ] (adjn) {Adjoint neutron transport};
%		%\node [file] (adjnflux) {Adjoint neutron flux};
%		\node [script, below of = adjn] (vr) {Calculate MC VR};
%		\node [file, below of = vr, xshift =-2cm] (ww) {Weight windows};
%		\node [file, below of = vr, xshift=2cm] (bsrc) {Biased source};
%
%		\path [line] (adjp) -- (adjnsrc);
%		\path [line] (tcalc) -- (adjnsrc);
%		\path [line] (adjnsrc) -- (adjn);
%		\path [line] (adjn) -- (vr);
%		\path [line] (vr) -- (ww);
%		\path [line] (vr) -- (bsrc);
%
%	\end{tikzpicture}
%	}%resize
%\end{figure}
%
%
%\end{frame}


%\section{Implementation Progress}

%\subsection{Software}
%\begin{frame}{Software}
%	\begin{itemize}
%		\item{\textbf{PARTISN}: \textbf{PAR}allel, \textbf{TI}me-Dependent \textbf{SN}
%			 \cite{partisn}}
%		   \begin{itemize}
%		   \item{Deterministic adjoint transport}
%		   \end{itemize}
%	   \item{\textbf{DAGMC}: \textbf{D}irect \textbf{A}ccelerated
%		   \textbf{G}eometry \cite{dagmc},
%		   \textbf{M}onte \textbf{C}arlo, \textbf{MCNP}: \textbf{M}onte
%			\textbf{C}arlo \textbf{N} \textbf{P}article
%			\cite{mcnp_manual}}
%		   \begin{itemize}
%		   \item{Forward MC transport on CAD geometry}
%		   \end{itemize}
%	   \item{\textbf{ALARA}: \textbf{A}nalytic and \textbf{L}aplacian
%		   \textbf{A}daptive \textbf{R}adioactivity \textbf{A}nalysis
%			\cite{alara}}
%			\vspace{-0.4cm}
%		   \begin{itemize}
%		   \item{Activation analysis}
%		   \end{itemize}
%	   \item{\textbf{PyNE}: \textbf{Py}thon for \textbf{N}uclear \textbf{E}ngineering
%			 \cite{pyne}}
%		   \begin{itemize}
%		   \item{Tools to support transport}
%		   \end{itemize}
%	   \item{\textbf{MOAB}: \textbf{M}esh-\textbf{O}riented 
%		   dat\textbf{AB}ase \cite{moab}}
%		   \begin{itemize}
%		   \item{Moving geometries}
%		   \end{itemize}
%	\end{itemize}
%\end{frame}
%
%\subsection{Workflow}
%\begin{frame}{Time-integrated (T)GT-CADIS: Workflow}
%\begin{figure}
%\centering
%
%        \tikzstyle{script} = [ellipse, fill=UWGold!30, draw, text centered,
%	node distance=1.3cm]
%        \tikzstyle{code} = [ellipse, fill=UWRed!30, draw, text centered, node
%	distance=1.3cm]
%        \tikzstyle{file} = [rectangle, draw, text centered, node distance=1.3cm]
%        \tikzstyle{line} = [draw, -latex]
%        \tikzstyle{arrow} = [thick, ->, >=stealth]
%        %gray
%        \tikzstyle{gscript} = [ellipse, fill=gray!30, draw, text centered,
%	node distance=1.3cm]
%        \tikzstyle{gcode} = [ellipse, fill=gray!30, draw, text centered, node
%	distance=1.3cm]
%        \tikzstyle{gfile} = [rectangle, fill=gray!30, draw, text centered, node distance=1.3cm]
%       
%	\resizebox{!}{5.7cm}{
%        \begin{tikzpicture}[every text node part/.style={align=center}]
%		% Optimized neutron transport
%		\node [script] (adjnsrc) { Calculate $q_n^+$};
%		\node [code, above of = adjnsrc, xshift=-3cm, yshift=0.4cm]
%		(adjp) {Adjoint photon transport \\ at each $t_{mov}$};
%		%\node [file] (adjpflux) {Adjoint photon flux};
%		\node [script, above of = adjnsrc, xshift=3cm, yshift=0.1cm] (tcalc)
%		{Calculate $T$};
%		\node [code, below of = adjnsrc, yshift=-0.5cm ] (adjn) {Adjoint neutron transport};
%		%\node [file] (adjnflux) {Adjoint neutron flux};
%		\node [script, below of = adjn] (vr) {Calculate MC VR};
%		\node [file, below of = vr, xshift =-2cm] (ww) {Weight windows};
%		\node [file, below of = vr, xshift=2cm] (bsrc) {Biased source};
%
%		\path [line] (adjp) -- (adjnsrc);
%		\path [line] (tcalc) -- (adjnsrc);
%		\path [line] (adjnsrc) -- (adjn);
%		\path [line] (adjn) -- (vr);
%		\path [line] (vr) -- (ww);
%		\path [line] (vr) -- (bsrc);
%
%	\end{tikzpicture}
%	}%resize
%\end{figure}
%\end{frame}

\subsection{Demonstration}

\begin{frame}{Time-integrated (T)GT-CADIS: Workflow}
\begin{figure}
\centering

        \tikzstyle{script} = [ellipse, fill=UWGold!30, draw, text centered,
	node distance=1.3cm]
        \tikzstyle{code} = [ellipse, fill=UWRed!30, draw, text centered, node
	distance=1.3cm]
        \tikzstyle{file} = [rectangle, draw, text centered, node distance=1.3cm]
        \tikzstyle{line} = [draw, -latex]
        \tikzstyle{arrow} = [thick, ->, >=stealth]
        %gray
        \tikzstyle{gscript} = [ellipse, fill=gray!30, draw, text centered,
	node distance=1.3cm]
        \tikzstyle{gcode} = [ellipse, fill=gray!30, draw, text centered, node
	distance=1.3cm]
        \tikzstyle{gfile} = [rectangle, fill=gray!30, draw, text centered, node distance=1.3cm]
       
	\resizebox{!}{5.7cm}{
        \begin{tikzpicture}[every text node part/.style={align=center}]
		% Optimized neutron transport
		\node [gscript] (adjnsrc) { Calculate $q_n^+$};
		\node [code, above of = adjnsrc, xshift=-3cm, yshift=0.4cm]
		(adjp) {Adjoint photon transport \\ at each $t_{mov}$};
		%\node [file] (adjpflux) {Adjoint photon flux};
		\node [gscript, above of = adjnsrc, xshift=3cm, yshift=0.1cm] (tcalc)
		{Calculate $T$};
		\node [gcode, below of = adjnsrc, yshift=-0.5cm ] (adjn) {Adjoint neutron transport};
		%\node [file] (adjnflux) {Adjoint neutron flux};
		\node [gscript, below of = adjn] (vr) {Calculate MC VR};
		\node [gfile, below of = vr, xshift =-2cm] (ww) {Weight windows};
		\node [gfile, below of = vr, xshift=2cm] (bsrc) {Biased source};

		\path [line] (adjp) -- (adjnsrc);
		\path [line] (tcalc) -- (adjnsrc);
		\path [line] (adjnsrc) -- (adjn);
		\path [line] (adjn) -- (vr);
		\path [line] (vr) -- (ww);
		\path [line] (vr) -- (bsrc);

	\end{tikzpicture}
	}%resize
\end{figure}
\end{frame}

%%%%%% 1. Prepare geometries
\begin{frame}{TGT-CADIS: Generate Stepwise Geometries}
\begin{columns}
\column{0.45\textwidth}
	\vspace{-6cm}
	\begin{itemize}
		\item{Generate geometry files at each time step of movement
			after shutdown}
%	\multiinclude[<+->][format=jpg, graphics={ scale=0.25}]{gray_config}
        \begin{figure}
	  \includegraphics[scale=0.23]{orig_pos.jpg}
        \end{figure}
        \begin{figure}
	  \includegraphics[scale=0.23]{final_pos.jpg}
        \end{figure}
	\end{itemize}
		
\column{0.55\textwidth}
\begin{figure}
\centering
        \vspace{1.3cm}
	\resizebox{!}{13.5cm}{
        \tikzstyle{script} = [ellipse, fill=UWGold!30, draw, text centered, node distance=3cm]
        \tikzstyle{code} = [ellipse, fill=UWRed!30, draw, text centered, node distance=3cm]
        \tikzstyle{file} = [rectangle, draw, text centered, node distance=3cm]
        \tikzstyle{line} = [draw, -latex']
        
        %\begin{tikzpicture}[node distance = 2 cm, auto]
        \begin{tikzpicture}[every text node part/.style={align=center}]
        	%Generate adj gamma Partisn input
                \node [file, xshift=-1cm] (pgeom0) {DAGMC geom at $t_{mov}=0$};
		\node [script, below = 0.4 of pgeom0] (dagtrans) {Transform tool};
		\phantom{ \node [script, below = 0.4 of dagtrans] (partgen) {Prepare
		PARTISN};}
                %\node [file, left = 0.9 of partgen] (pgeomi) {DAGMC geom\\
		%at $t_{mov}$};
                \node [file, below = 0.4 of dagtrans] (pgeomi) {DAGMC geom\\
		at $t_{mov}$};
		\phantom{                \node [file, right=0.9 of partgen] (f2d) {Flux-to-dose conversion factors};
                \node [file, below =0.4 of partgen] (ppartinp) {PARTISN input
		file};
                \node [code, below =0.4 of ppartinp] (partadjp) {PARTISN adjoint $\gamma$ transport};}
                \path [line] (pgeom0) -- (dagtrans);
		\path [line] (dagtrans) -- (pgeomi);
%                \path [line] (pgeomi) -- (partadjp);
		\phantom{                
                \path [line] (f2d) -- (partgen);
		\path [line] (partgen) -- (ppartinp);}
\phantom{        	% Partisn adj gamma transport
        	\node [file, below =0.4 of partadjp] (padjf)
		{PARTISN$\phi_{\gamma, t_{mov}}^{+}$ file};
        	\node [script, below =0.4 of padjf] (pf2mesh) {PyNE.R2S \\
		convert PARTISN to voxel mesh};
		\node [file, below =0.4 of pf2mesh] (padjfvmesh)
		{$\phi_{\gamma,t_{mov}}^{+}$ voxel mesh};
        	\node [script, below =0.4 of padjfvmesh] (conv_v_t){Convert voxel to tet};
        	\node [file, below =0.4 of
		conv_v_t](padjftmesh){$\phi_{\gamma,t_{mov}}^{+}$ tet mesh};
		\node [script, below =0.4 of padjftmesh](sum){Sum
		$\phi_{\gamma,t_{mov}}^{+}$over all time steps};
		\node [file, below =0.4 of sum](avgadjft){$\phi_{\gamma}^{+}$ tet mesh};
        	\node [script, below =0.4 of avgadjft] (conv_t_v){Convert tet to voxel};
		\node [file, below =0.4 of conv_t_v](avgadjfv){${\phi_{\gamma}^{+}}$ voxel mesh};
        	\path [line] (ppartinp) -- (partadjp);
        	\path [line] (partadjp) -- (padjf);
        	\path [line] (padjf) -- (pf2mesh);
        	\path [line] (pf2mesh) -- (padjfvmesh);
		\path [line] (padjfvmesh) -- (conv_v_t);
		\path [line] (conv_v_t) -- (padjftmesh);
		\path [line] (padjftmesh) -- (sum);
		\path [line] (sum) -- (avgadjft);
		\path [line] (avgadjft) -- (conv_t_v);
		\path [line] (conv_t_v) -- (avgadjfv);
	    	% Generate adj n source
        	\node [script, below =0.4 of avgadjfv] (gtcadis) {gtcadis.py}; 
		\node [file, left =0.9 of gtcadis] (ngeom) {DAGMC geom at
		$t_{mov}$};
        	\node [file, right =0.9 of gtcadis] (matlib) {ALARA material library};
        	\node [file, below =0.4 of gtcadis] (nadjs) {$q_{n}^{+}$ voxel mesh};
        	\path [line] (ngeom) -- (gtcadis);
        	\path [line] (avgadjfv) -- (gtcadis);
        	\path [line] (matlib) -- (gtcadis);
        	\path [line] (gtcadis) -- (nadjs);
}%phantom		
        \end{tikzpicture}
	}%resizebox
	\end{figure}
\end{columns}
\end{frame}

%
%\section{Progress}
%\begin{frame}{MC Moving Geometry Simulations}
%	\begin{itemize}
%		\item{Capabilities to update position of geometry based on
%			user-defined motion data}
%%			\begin{enumerate}
%%				\item{Production of step-wise geometry files}
%%				\item{DAGMCNP update}
%%			\end{enumerate}
%		%\item{Motion data:}
%		%	\begin{itemize}
%		%		\item{Time-dependent translation or rotation
%		%			vector, total length of time, number or desired time steps}
%		%		\item{Relocation transform}
%		%	\end{itemize}
%		\item{Basic functionality:}
%			\begin{itemize}
%				%\item{Use MOAB to manipulate mesh}
%				\item{%Parse geometry tagged with transformation
%					%data}
%					Read tag data that specifies 
%					type of transformation}
%				\item{Identify starting position of each
%					component}
%				\item{Update position according to
%					transformation}
%			\end{itemize}
%	\end{itemize}
%\end{frame}
%\begin{frame}{Geometry transformation tool}
%\begin{itemize}
%\item{Tool to generate new geometry file at each time step}
%\item{CAD geometry tagged with type of transformation}
%%\item{Motion data:}
%%	\begin{itemize}
%		\item{Corresponding time-dependent motion data given in text file:}
%			\begin{itemize}
%				\item{Translation or rotation vector}
%				\item{Total duration of geometry movement}
%				\item{Number of time steps}
%			\end{itemize}
%\item{Limitations:}
%   \begin{itemize}
%     \item{Does not treat object kinetics}
% %  \item{User must be careful to not cause overlap in components}
%     \item{Can only move components that do not share surfaces with other components}
%  \end{itemize}
%%		\item{Relocation:}
%%			\begin{itemize}
%%		        	\item{Translation or rotation distance}
%%			\end{itemize}
%
%%	\end{itemize}
%\end{itemize}
%	\centering
%	\transduration<0-7>{0}
%	\multiinclude[<+->][format=png, graphics={ scale=0.17}]{Photon_flux}
%
%%\end{columns}
%\end{frame}

%\begin{frame}{MC Moving Geometry: Stepwise Geometry Tool}
%\begin{itemize}
%\item{Tool to generate new geometry file at each time step during movement}
%\item{Motion data:}
%	\begin{itemize}
%		\item{Time-dependent:}
%			\begin{itemize}
%				\item{Translation and/or rotation vector}
%				\item{Duration of time}
%				\item{Number of time steps}
%			\end{itemize}
%		\item{Relocation:}
%			\begin{itemize}
%		        	\item{Translation and/or rotation distance}
%			\end{itemize}
%
%	\end{itemize}
%\end{itemize}
%	\begin{center}
%	\transduration<0-8>{0}
%	\multiinclude[<+->][format=png, graphics={ scale=0.15}]{snapshot}
%	\end{center}
%\end{frame}
%\subsection{DAGMCNP update}
%\begin{frame}{Moving Geometries: DAGMCNP Update}
%	\begin{itemize}
%		\item{Functionality to apply MCNP TR(n) card data to DAGMC
%			geometry}
%		\item{Motion data:}
%			\begin{itemize}
%				\item{Transformation distance}
%			\end{itemize}
%		\item{Separate input file containing transformations for each
%			time step of geometry movement}
%	\end{itemize}
%\end{frame}
%%%% 2. Adjoint Photon Transport
\begin{frame}{TGT-CADIS: Generate VR Parameters}
\begin{columns}
\column{0.45\textwidth}
        \vspace{-3cm}
	\begin{itemize}
		\item{Perform adjoint photon transport at each time step}
	\end{itemize}
%	\multiinclude[<+->][format=jpg, graphics={ scale=0.25}]{newconfig}
	        \includegraphics[scale=0.20]{adj_pflux_0.jpg}
                \vspace{0.3cm}
	        \includegraphics[scale=0.20]{adj_pflux_1.jpg}
		
\column{0.55\textwidth}
	\vspace{-1cm}
\begin{figure}
\centering
 %       \vspace{1cm}
	\resizebox{!}{11cm}{
        \tikzstyle{script} = [ellipse, fill=UWGold!30, draw, text centered, node distance=3cm]
        \tikzstyle{code} = [ellipse, fill=UWRed!30, draw, text centered, node distance=3cm]
        \tikzstyle{file} = [rectangle, draw, text centered, node distance=3cm]
        \tikzstyle{line} = [draw, -latex']
        
        %\begin{tikzpicture}[node distance = 2 cm, auto]
        \begin{tikzpicture}[every text node part/.style={align=center}]
        	%Generate adj gamma Partisn input
\phantom{       \node [file] (pgeom0) {DAGMC geom at $t_{mov}=0$};
		\node [script, below = 0.4 of pgeom0] (dagtrans) {Transform tool};
		\node [script, below = 0.4 of dagtrans] (partgen) {Prepare
		PARTISN};
}
                \node [file, below = 0.4 of partgen] (pgeomi) {DAGMC geom\\
		at $t_{mov}$};

\phantom{       \node [file, right=0.9 of partgen] (f2d) {Flux-to-dose \\ 
		conversion factors};
                \node [file, below =0.4 of partgen] (ppartinp) {PARTISN input
		file};
                \path [line] (pgeom0) -- (dagtrans);
		\path [line] (dagtrans) -- (pgeomi);
		\path [line] (pgeomi) -- (partgen);
                \path [line] (f2d) -- (partgen);
		\path [line] (partgen) -- (ppartinp);
}
        	% Partisn adj gamma transport
                \node [code, below =0.4 of pgeomi] (partadjp) {PARTISN adjoint $\gamma$ transport \\ at each $t_{mov}$};
\phantom{
        	\node [file, below =0.4 of partadjp] (padjf)
		{PARTISN $\phi_{\gamma, t_{mov}}^{+}$ file};
        	\node [script, below =0.4 of padjf] (pf2mesh) {PyNE \\
		convert PARTISN to voxel mesh};
}
		\node [file, below =0.4 of partadjp] (padjfvmesh)
		{$\phi_{\gamma,t_{mov}}^{+}$ voxel mesh};
\phantom{
		\path [line] (partgen) -- (partadjp);
%        	\path [line] (ppartinp) -- (partadjp);
        	\path [line] (partadjp) -- (padjf);
        	\path [line] (padjf) -- (pf2mesh);
        	\path [line] (pf2mesh) -- (padjfvmesh);
}
                \path [line] (pgeomi) -- (partadjp);
                \path [line] (partadjp) -- (padjfvmesh);
%                \path [line] (padjfvmesh) -- (padjftmesh);
\phantom{	\node [script, below =0.4 of padjfvmesh] (conv_v_t){Convert voxel to tet};
        	\node [file, below =0.4 of
		conv_v_t](padjftmesh){$\phi_{\gamma,t_{mov}}^{+}$ tet mesh};
		\node [script, below =0.4 of padjftmesh](sum){Sum
		$\phi_{\gamma,t_{mov}}^{+}$over all time steps};
		\node [file, below =0.4 of sum](avgadjft){$\phi_{\gamma}^{+}$ tet mesh};
        	\node [script, below =0.4 of avgadjft] (conv_t_v){Convert tet to voxel};
		\node [file, below =0.4 of conv_t_v](avgadjfv){${\phi_{\gamma}^{+}}$ voxel mesh};
		\path [line] (padjfvmesh) -- (conv_v_t);
		\path [line] (conv_v_t) -- (padjftmesh);
		\path [line] (padjftmesh) -- (sum);
		\path [line] (sum) -- (avgadjft);
		\path [line] (avgadjft) -- (conv_t_v);
		\path [line] (conv_t_v) -- (avgadjfv);
	    	% Generate adj n source
        	\node [script, below =0.4 of avgadjfv] (gtcadis) {gtcadis.py}; 
		\node [file, left =0.9 of gtcadis] (ngeom) {DAGMC geom at
		$t_{mov}$};
        	\node [file, right =0.9 of gtcadis] (matlib) {ALARA material library};
        	\node [file, below =0.4 of gtcadis] (nadjs) {$q_{n}^{+}$ voxel mesh};
        	\path [line] (ngeom) -- (gtcadis);
        	\path [line] (avgadjfv) -- (gtcadis);
        	\path [line] (matlib) -- (gtcadis);
        	\path [line] (gtcadis) -- (nadjs);
}%phantom		
        \end{tikzpicture}
	}%resizebox
	\end{figure}
\end{columns}
\end{frame}

\begin{frame}{Time-integrated (T)GT-CADIS: Workflow}
\begin{figure}
\centering

        \tikzstyle{script} = [ellipse, fill=UWGold!30, draw, text centered,
	node distance=1.3cm]
        \tikzstyle{code} = [ellipse, fill=UWRed!30, draw, text centered, node
	distance=1.3cm]
        \tikzstyle{file} = [rectangle, draw, text centered, node distance=1.3cm]
        \tikzstyle{line} = [draw, -latex]
        \tikzstyle{arrow} = [thick, ->, >=stealth]
        %gray
        \tikzstyle{gscript} = [ellipse, fill=gray!30, draw, text centered,
	node distance=1.3cm]
        \tikzstyle{gcode} = [ellipse, fill=gray!30, draw, text centered, node
	distance=1.3cm]
        \tikzstyle{gfile} = [rectangle, fill=gray!30, draw, text centered, node distance=1.3cm]
       
	\resizebox{!}{5.7cm}{
        \begin{tikzpicture}[every text node part/.style={align=center}]
		% Optimized neutron transport
		\node [gscript] (adjnsrc) {Calculate $q_n^+$};
		\node [gcode, above of = adjnsrc, xshift=-3cm, yshift=0.4cm]
		(adjp) {Adjoint photon transport \\ at each $t_{mov}$};
		%\node [file] (adjpflux) {Adjoint photon flux};
		\node [script, above of = adjnsrc, xshift=3cm, yshift=0.1cm] (tcalc)
		{Calculate $T$};
		\node [gcode, below of = adjnsrc, yshift=-0.5cm ] (adjn) {Adjoint neutron transport};
		%\node [file] (adjnflux) {Adjoint neutron flux};
		\node [gscript, below of = adjn] (vr) {Calculate MC VR};
		\node [gfile, below of = vr, xshift =-2cm] (ww) {Weight windows};
		\node [gfile, below of = vr, xshift=2cm] (bsrc) {Biased source};

		\path [line] (adjp) -- (adjnsrc);
		\path [line] (tcalc) -- (adjnsrc);
		\path [line] (adjnsrc) -- (adjn);
		\path [line] (adjn) -- (vr);
		\path [line] (vr) -- (ww);
		\path [line] (vr) -- (bsrc);

	\end{tikzpicture}
	}%resize
\end{figure}
\end{frame}

\begin{frame}{Time-integrated (T)GT-CADIS: Workflow}
\begin{figure}
\centering

        \tikzstyle{script} = [ellipse, fill=UWGold!30, draw, text centered,
	node distance=1.3cm]
        \tikzstyle{code} = [ellipse, fill=UWRed!30, draw, text centered, node
	distance=1.3cm]
        \tikzstyle{file} = [rectangle, draw, text centered, node distance=1.3cm]
        \tikzstyle{line} = [draw, -latex]
        \tikzstyle{arrow} = [thick, ->, >=stealth]
        %gray
        \tikzstyle{gscript} = [ellipse, fill=gray!30, draw, text centered,
	node distance=1.3cm]
        \tikzstyle{gcode} = [ellipse, fill=gray!30, draw, text centered, node
	distance=1.3cm]
        \tikzstyle{gfile} = [rectangle, fill=gray!30, draw, text centered, node distance=1.3cm]
       
	\resizebox{!}{5.7cm}{
        \begin{tikzpicture}[every text node part/.style={align=center}]
		% Optimized neutron transport
		\node [script] (adjnsrc) { Calculate time-integrated $q_n^+$};
		\node [gcode, above of = adjnsrc, xshift=-3cm, yshift=0.4cm]
		(adjp) {Adjoint photon transport \\ at each $t_{mov}$};
		%\node [file] (adjpflux) {Adjoint photon flux};
		\node [gscript, above of = adjnsrc, xshift=3cm, yshift=0.1cm] (tcalc)
		{Calculate $T$};
		\node [gcode, below of = adjnsrc, yshift=-0.5cm ] (adjn) {Adjoint neutron transport};
		%\node [file] (adjnflux) {Adjoint neutron flux};
		\node [gscript, below of = adjn] (vr) {Calculate MC VR};
		\node [gfile, below of = vr, xshift =-2cm] (ww) {Weight windows};
		\node [gfile, below of = vr, xshift=2cm] (bsrc) {Biased source};

		\path [line] (adjp) -- (adjnsrc);
		\path [line] (tcalc) -- (adjnsrc);
		\path [line] (adjnsrc) -- (adjn);
		\path [line] (adjn) -- (vr);
		\path [line] (vr) -- (ww);
		\path [line] (vr) -- (bsrc);

	\end{tikzpicture}
	}%resize
\end{figure}
\end{frame}

%%%% 3. Map voxel to tet mesh 
\begin{frame}{TGT-CADIS: Calculate Adjoint Neutron Source}
\begin{columns}
\column{0.45\textwidth}
	\begin{itemize}
		\item{Combine $\phi_{\gamma, t_{mov}}^+$ and $T_{t_{mov}}$} 
		\item{Map the structured (voxel) mesh to a tetrahedral mesh}
	\end{itemize}
\vspace{0.2cm}
%	\includegraphics[scale=0.29]{mesh_total.jpg}
	\includegraphics[scale=0.14]{adj_pflux_0_tet.jpg}
\vspace{0.1cm}
	\includegraphics[scale=0.14]{adj_pflux_1_tet.jpg}
		
\column{0.55\textwidth}
\begin{figure}
	\vspace{-6cm}
\centering

	\resizebox{!}{11cm}{
        \tikzstyle{script} = [ellipse, fill=UWGold!30, draw, text centered, node distance=3cm]
        \tikzstyle{code} = [ellipse, fill=UWRed!30, draw, text centered, node distance=3cm]
        \tikzstyle{file} = [rectangle, draw, text centered, node distance=3cm]
        \tikzstyle{line} = [draw, -latex']
        
        %\begin{tikzpicture}[node distance = 2 cm, auto]
        \begin{tikzpicture}[every text node part/.style={align=center}]
\phantom{        	%Generate adj gamma Partisn input
                \node [file] (pgeom0) {DAGMC geom at $t_{mov}=0$};
		\node [script, below = 0.4 of pgeom0] (dagtrans) {Transform tool};
		\node [script, below = 0.4 of dagtrans] (partgen) {Prepare
		PARTISN};
                \node [file, left = 0.9 of partgen] (pgeomi) {DAGMC geom
		at $t_{mov}$};
		\node [file, right=0.9 of partgen] (f2d) {Flux-to-dose \\ 
		conversion factors};
                \node [file, below =0.4 of partgen] (ppartinp) {PARTISN input
		file};
                \path [line] (pgeom0) -- (dagtrans);
		\path [line] (dagtrans) -- (pgeomi);
		\path [line] (pgeomi) -- (partgen);
                \path [line] (f2d) -- (partgen);
		\path [line] (partgen) -- (ppartinp);
        	% Partisn adj gamma transport
                \node [code, below =0.4 of ppartinp] (partadjp) {PARTISN adjoint $\gamma$ transport};
        	\node [file, below =0.4 of partadjp] (padjf)
		{PARTISN$\phi_{\gamma, t_{mov}}^{+}$ file};
        	\node [script, below =0.4 of padjf] (pf2mesh) {PyNE.R2S \\
		convert PARTISN to voxel mesh};
        	\path [line] (ppartinp) -- (partadjp);
        	\path [line] (partadjp) -- (padjf);
        	\path [line] (padjf) -- (pf2mesh);
        	\path [line] (pf2mesh) -- (padjfvmesh);
}%phantom		
		\node [file, below =0.4 of pf2mesh, xshift=1.8cm] (tmat) {$T_{t_{mov}}$};
		\node [file, below =0.4 of pf2mesh, xshift=-1cm] (padjfvmesh)
		{$\phi_{\gamma,t_{mov}}^{+}$ voxel mesh};
	\node [script, below =0.4 of padjfvmesh] (conv_v_t){Map voxel to tet};
\phantom{        	\node [file, below =0.4 of
		conv_v_t](padjftmesh){$\phi_{\gamma,t_{mov}}^{+}$ tet mesh};
}
		\node [script, below =0.4 of conv_v_t](sum){Avg.
		%$\phi_{\gamma,t_{mov}}^{+}$ \\
                over \\all time steps};
\phantom{
		\node [file, below =0.4 of sum](avgadjft){$\phi_{\gamma}^{+}$ tet mesh};
}
        	\node [script, below =0.4 of sum] (conv_t_v){Map tet to voxel};
		\node [file, below =0.4 of conv_t_v](avgadjfv){${q{n}^{+}}$ voxel mesh};
		\path [line] (padjfvmesh) -- (conv_v_t);
		\path [line] (tmat) -- (conv_v_t);
		\path [line] (conv_v_t) -- (sum);
%		\path [line] (padjftmesh) -- (sum);
		\path [line] (sum) -- (conv_t_v);
%		\path [line] (avgadjft) -- (conv_t_v);
		\path [line] (conv_t_v) -- (avgadjfv);
\phantom{	    	% Generate adj n source
        	\node [script, below =0.4 of avgadjfv] (gtcadis) {gtcadis.py}; 
		\node [file, left =0.9 of gtcadis] (ngeom) {DAGMC geom at
		$t_{mov}$};
        	\node [file, right =0.9 of gtcadis] (matlib) {ALARA material library};
        	\node [file, below =0.4 of gtcadis] (nadjs) {$q_{n}^{+}$ voxel mesh};
        	\path [line] (ngeom) -- (gtcadis);
        	\path [line] (avgadjfv) -- (gtcadis);
        	\path [line] (matlib) -- (gtcadis);
        	\path [line] (gtcadis) -- (nadjs);
}%phantom
        \end{tikzpicture}
	}%resizebox
	\end{figure}
\end{columns}
\end{frame}

%%%%% 3. Avg the adj photon flux 
%\begin{frame}{TGT-CADIS: Calculate Adjoint Neutron Source}
%\begin{columns}
%\column{0.45\textwidth}
%	\begin{itemize}
%		\item{Average over all time steps}
%	%	\item{Map tetrahedral mesh onto voxel mesh at original configuration}
%	
%\vspace{0.2cm}
%	\includegraphics[scale=0.16]{avg_adj_pflux_vox.jpg}
%		
%\column{0.55\textwidth}
%\begin{figure}
%	\vspace{-6cm}
%\centering
%
%	\resizebox{!}{11cm}{
%        \tikzstyle{script} = [ellipse, fill=UWGold!30, draw, text centered, node distance=3cm]
%        \tikzstyle{code} = [ellipse, fill=UWRed!30, draw, text centered, node distance=3cm]
%        \tikzstyle{file} = [rectangle, draw, text centered, node distance=3cm]
%        \tikzstyle{line} = [draw, -latex']
%        
%        %\begin{tikzpicture}[node distance = 2 cm, auto]
%        \begin{tikzpicture}[every text node part/.style={align=center}]
%\phantom{        	%Generate adj gamma Partisn input
%                \node [file] (pgeom0) {DAGMC geom at $t_{mov}=0$};
%		\node [script, below = 0.4 of pgeom0] (dagtrans) {Transform tool};
%		\node [script, below = 0.4 of dagtrans] (partgen) {Prepare PARTISN};
%                \node [file, left = 0.9 of partgen] (pgeomi) {DAGMC geom at $t_{mov}$};
%		\node [file, right=0.9 of partgen] (f2d) {Flux-to-dose \\ conversion factors};
%                \node [file, below =0.4 of partgen] (ppartinp) {PARTISN input file};
%                \path [line] (pgeom0) -- (dagtrans);
%		\path [line] (dagtrans) -- (pgeomi);
%		\path [line] (pgeomi) -- (partgen);
%                \path [line] (f2d) -- (partgen);
%		\path [line] (partgen) -- (ppartinp);
%        	% Partisn adj gamma transport
%                \node [code, below =0.4 of ppartinp] (partadjp) {PARTISN adjoint $\gamma$ transport};
%        	\node [file, below =0.4 of partadjp] (padjf)
%		{PARTISN$\phi_{\gamma, t_{mov}}^{+}$ file};
%        	\node [script, below =0.4 of padjf] (pf2mesh) {PyNE.R2S \\convert PARTISN to voxel mesh};
%        	\path [line] (ppartinp) -- (partadjp);
%        	\path [line] (partadjp) -- (padjf);
%        	\path [line] (padjf) -- (pf2mesh);
%        	\path [line] (pf2mesh) -- (padjfvmesh);
%}%phantom		
%		\node [file, below =0.4 of pf2mesh, xshift=1.8cm] (tmat) {$T_{t_{mov}}$};
%		\node [file, below =0.4 of pf2mesh, xshift=-1cm] (padjfvmesh)
%		{$\phi_{\gamma,t_{mov}}^{+}$ voxel mesh};
%	\node [script, below =0.4 of padjfvmesh] (conv_v_t){Map voxel to tet};
%\phantom{        	\node [file, below =0.4 of
%		conv_v_t](padjftmesh){$\phi_{\gamma,t_{mov}}^{+}$ tet mesh};
%}
%		\node [script, below =0.4 of conv_v_t](sum){Avg.
%		%$\phi_{\gamma,t_{mov}}^{+}$ \\
%                over \\all time steps};
%\phantom{
%		\node [file, below =0.4 of sum](avgadjft){$\phi_{\gamma}^{+}$ tet mesh};
%}
%        	\node [script, below =0.4 of sum] (conv_t_v){Map tet to voxel};
%		\node [file, below =0.4 of conv_t_v](avgadjfv){${q{n}^{+}}$ voxel mesh};
%		\path [line] (padjfvmesh) -- (conv_v_t);
%		\path [line] (tmat) -- (conv_v_t);
%		\path [line] (conv_v_t) -- (sum);
%%		\path [line] (padjftmesh) -- (sum);
%		\path [line] (sum) -- (conv_t_v);
%%		\path [line] (avgadjft) -- (conv_t_v);
%		\path [line] (conv_t_v) -- (avgadjfv);
%\phantom{	    	% Generate adj n source
%        	\node [script, below =0.4 of avgadjfv] (gtcadis) {gtcadis.py}; 
%		\node [file, left =0.9 of gtcadis] (ngeom) {DAGMC geom at
%		$t_{mov}$};
%        	\node [file, right =0.9 of gtcadis] (matlib) {ALARA material library};
%        	\node [file, below =0.4 of gtcadis] (nadjs) {$q_{n}^{+}$ voxel mesh};
%        	\path [line] (ngeom) -- (gtcadis);
%        	\path [line] (avgadjfv) -- (gtcadis);
%        	\path [line] (matlib) -- (gtcadis);
%        	\path [line] (gtcadis) -- (nadjs);
%}%phantom
%        \end{tikzpicture}
%	}%resizebox
%	\end{figure}
%\end{columns}
%\end{frame}

%\begin{frame}{Time-integrated (T)GT-CADIS: Workflow}
%\begin{figure}
%\centering
%
%        \tikzstyle{script} = [ellipse, fill=UWGold!30, draw, text centered,
%	node distance=1.3cm]
%        \tikzstyle{code} = [ellipse, fill=UWRed!30, draw, text centered, node
%	distance=1.3cm]
%        \tikzstyle{file} = [rectangle, draw, text centered, node distance=1.3cm]
%        \tikzstyle{line} = [draw, -latex]
%        \tikzstyle{arrow} = [thick, ->, >=stealth]
%        %gray
%        \tikzstyle{gscript} = [ellipse, fill=gray!30, draw, text centered,
%	node distance=1.3cm]
%        \tikzstyle{gcode} = [ellipse, fill=gray!30, draw, text centered, node
%	distance=1.3cm]
%        \tikzstyle{gfile} = [rectangle, fill=gray!30, draw, text centered, node distance=1.3cm]
%       
%	\resizebox{!}{5.7cm}{
%        \begin{tikzpicture}[every text node part/.style={align=center}]
%		% Optimized neutron transport
%		\node [gscript] (adjnsrc) { Calculate $q_n^+$};
%		\node [gcode, above of = adjnsrc, xshift=-3cm, yshift=0.4cm]
%		(adjp) {Adjoint photon transport \\ at each $t_{mov}$};
%		%\node [file] (adjpflux) {Adjoint photon flux};
%		\node [gscript, above of = adjnsrc, xshift=3cm, yshift=0.1cm] (tcalc)
%		{Calculate $T$};
%		\node [code, below of = adjnsrc, yshift=-0.5cm ] (adjn) {Adjoint neutron transport};
%		%\node [file] (adjnflux) {Adjoint neutron flux};
%		\node [script, below of = adjn] (vr) {Calculate MC VR};
%		\node [file, below of = vr, xshift =-2cm] (ww) {Weight windows};
%		\node [file, below of = vr, xshift=2cm] (bsrc) {Biased source};
%
%		\path [line] (adjp) -- (adjnsrc);
%		\path [line] (tcalc) -- (adjnsrc);
%		\path [line] (adjnsrc) -- (adjn);
%		\path [line] (adjn) -- (vr);
%		\path [line] (vr) -- (ww);
%		\path [line] (vr) -- (bsrc);
%
%	\end{tikzpicture}
%	}%resize
%\end{figure}
%\end{frame}
%%%% 3. Avg the adj photon flux 
\begin{frame}{TGT-CADIS: Calculate Adjoint Neutron Source}
\begin{columns}
\column{0.45\textwidth}
	\begin{itemize}
		\item{Average over all time steps}
		\item{Map tetrahedral mesh onto voxel mesh at original configuration}
	\end{itemize}
\vspace{0.2cm}
	\includegraphics[scale=0.16]{avg_adj_pflux_vox.jpg}
		
\column{0.55\textwidth}
\begin{figure}
	\vspace{-6cm}
\centering

	\resizebox{!}{11cm}{
        \tikzstyle{script} = [ellipse, fill=UWGold!30, draw, text centered, node distance=3cm]
        \tikzstyle{code} = [ellipse, fill=UWRed!30, draw, text centered, node distance=3cm]
        \tikzstyle{file} = [rectangle, draw, text centered, node distance=3cm]
        \tikzstyle{line} = [draw, -latex']
        
        %\begin{tikzpicture}[node distance = 2 cm, auto]
        \begin{tikzpicture}[every text node part/.style={align=center}]
\phantom{        	%Generate adj gamma Partisn input
                \node [file] (pgeom0) {DAGMC geom at $t_{mov}=0$};
		\node [script, below = 0.4 of pgeom0] (dagtrans) {Transform tool};
		\node [script, below = 0.4 of dagtrans] (partgen) {Prepare
		PARTISN};
                \node [file, left = 0.9 of partgen] (pgeomi) {DAGMC geom
		at $t_{mov}$};
		\node [file, right=0.9 of partgen] (f2d) {Flux-to-dose \\ 
		conversion factors};
                \node [file, below =0.4 of partgen] (ppartinp) {PARTISN input
		file};
                \path [line] (pgeom0) -- (dagtrans);
		\path [line] (dagtrans) -- (pgeomi);
		\path [line] (pgeomi) -- (partgen);
                \path [line] (f2d) -- (partgen);
		\path [line] (partgen) -- (ppartinp);
        	% Partisn adj gamma transport
                \node [code, below =0.4 of ppartinp] (partadjp) {PARTISN adjoint $\gamma$ transport};
        	\node [file, below =0.4 of partadjp] (padjf)
		{PARTISN$\phi_{\gamma, t_{mov}}^{+}$ file};
        	\node [script, below =0.4 of padjf] (pf2mesh) {PyNE.R2S \\
		convert PARTISN to voxel mesh};
        	\path [line] (ppartinp) -- (partadjp);
        	\path [line] (partadjp) -- (padjf);
        	\path [line] (padjf) -- (pf2mesh);
        	\path [line] (pf2mesh) -- (padjfvmesh);
}%phantom		
		\node [file, below =0.4 of pf2mesh, xshift=1.8cm] (tmat) {$T_{t_{mov}}$};
		\node [file, below =0.4 of pf2mesh, xshift=-1cm] (padjfvmesh)
		{$\phi_{\gamma,t_{mov}}^{+}$ voxel mesh};
	\node [script, below =0.4 of padjfvmesh] (conv_v_t){Map voxel to tet};
\phantom{        	\node [file, below =0.4 of
		conv_v_t](padjftmesh){$\phi_{\gamma,t_{mov}}^{+}$ tet mesh};
}
		\node [script, below =0.4 of conv_v_t](sum){Avg.
		%$\phi_{\gamma,t_{mov}}^{+}$ \\
                over \\all time steps};
\phantom{
		\node [file, below =0.4 of sum](avgadjft){$\phi_{\gamma}^{+}$ tet mesh};
}
        	\node [script, below =0.4 of sum] (conv_t_v){Map tet to voxel};
		\node [file, below =0.4 of conv_t_v](avgadjfv){${q{n}^{+}}$ voxel mesh};
		\path [line] (padjfvmesh) -- (conv_v_t);
		\path [line] (tmat) -- (conv_v_t);
		\path [line] (conv_v_t) -- (sum);
%		\path [line] (padjftmesh) -- (sum);
		\path [line] (sum) -- (conv_t_v);
%		\path [line] (avgadjft) -- (conv_t_v);
		\path [line] (conv_t_v) -- (avgadjfv);
\phantom{	    	% Generate adj n source
        	\node [script, below =0.4 of avgadjfv] (gtcadis) {gtcadis.py}; 
		\node [file, left =0.9 of gtcadis] (ngeom) {DAGMC geom at
		$t_{mov}$};
        	\node [file, right =0.9 of gtcadis] (matlib) {ALARA material library};
        	\node [file, below =0.4 of gtcadis] (nadjs) {$q_{n}^{+}$ voxel mesh};
        	\path [line] (ngeom) -- (gtcadis);
        	\path [line] (avgadjfv) -- (gtcadis);
        	\path [line] (matlib) -- (gtcadis);
        	\path [line] (gtcadis) -- (nadjs);
}%phantom
        \end{tikzpicture}
	}%resizebox
	\end{figure}
\end{columns}
\end{frame}

\begin{frame}{Time-integrated (T)GT-CADIS: Workflow}
\begin{figure}
\centering

        \tikzstyle{script} = [ellipse, fill=UWGold!30, draw, text centered,
	node distance=1.3cm]
        \tikzstyle{code} = [ellipse, fill=UWRed!30, draw, text centered, node
	distance=1.3cm]
        \tikzstyle{file} = [rectangle, draw, text centered, node distance=1.3cm]
        \tikzstyle{line} = [draw, -latex]
        \tikzstyle{arrow} = [thick, ->, >=stealth]
        %gray
        \tikzstyle{gscript} = [ellipse, fill=gray!30, draw, text centered,
	node distance=1.3cm]
        \tikzstyle{gcode} = [ellipse, fill=gray!30, draw, text centered, node
	distance=1.3cm]
        \tikzstyle{gfile} = [rectangle, fill=gray!30, draw, text centered, node distance=1.3cm]
       
	\resizebox{!}{5.7cm}{
        \begin{tikzpicture}[every text node part/.style={align=center}]
		% Optimized neutron transport
		\node [gscript] (adjnsrc) { Calculate $q_n^+$};
		\node [gcode, above of = adjnsrc, xshift=-3cm, yshift=0.4cm]
		(adjp) {Adjoint photon transport \\ at each $t_{mov}$};
		%\node [file] (adjpflux) {Adjoint photon flux};
		\node [gscript, above of = adjnsrc, xshift=3cm, yshift=0.1cm] (tcalc)
		{Calculate $T$};
		\node [code, below of = adjnsrc, yshift=-0.5cm ] (adjn) {Adjoint neutron transport};
		%\node [file] (adjnflux) {Adjoint neutron flux};
		\node [script, below of = adjn] (vr) {Calculate MC VR};
		\node [file, below of = vr, xshift =-2cm] (ww) {Weight windows};
		\node [file, below of = vr, xshift=2cm] (bsrc) {Biased source};

		\path [line] (adjp) -- (adjnsrc);
		\path [line] (tcalc) -- (adjnsrc);
		\path [line] (adjnsrc) -- (adjn);
		\path [line] (adjn) -- (vr);
		\path [line] (vr) -- (ww);
		\path [line] (vr) -- (bsrc);

	\end{tikzpicture}
	}%resize
\end{figure}
\end{frame}

%%%% 4. Adjoint Neutron Transport
\begin{frame}{TGT-CADIS: Generate VR Parameters}
\begin{columns}
\column{0.45\textwidth}
	\begin{itemize}
%		\item{Calculate T of each voxel}
%		\item{Calculate adjoint neutron source}
		\item{Perform adjoint neutron transport}
		%\item{Generate biased source and weight window mesh via CADIS methodology}
	\end{itemize}
	\includegraphics[scale=0.22]{adj_n_flux.png}
		
\column{0.55\textwidth}
\begin{figure}
	\vspace{-12cm}
\centering

	\resizebox{!}{15cm}{
        \tikzstyle{script} = [ellipse, fill=UWGold!30, draw, text centered, node distance=3cm]
        \tikzstyle{code} = [ellipse, fill=UWRed!30, draw, text centered, node distance=3cm]
        \tikzstyle{file} = [rectangle, draw, text centered, node distance=3cm]
        \tikzstyle{line} = [draw, -latex']
        
        %\begin{tikzpicture}[node distance = 2 cm, auto]
        \begin{tikzpicture}[every text node part/.style={align=center}]
\phantom{        	%Generate adj gamma Partisn input
                \node [file] (pgeom0) {DAGMC geom at $t_{mov}=0$};
		\node [script, below = 0.4 of pgeom0] (dagtrans) {Transform tool};
		\node [script, below = 0.4 of dagtrans] (partgen) {Prepare
		PARTISN};
                \node [file, left = 0.9 of partgen] (pgeomi) {DAGMC geom
		at $t_{mov}$};
		\node [file, right=0.9 of partgen] (f2d) {Flux-to-dose \\ 
		conversion factors};
                \node [file, below =0.4 of partgen] (ppartinp) {PARTISN input
		file};
                \path [line] (pgeom0) -- (dagtrans);
		\path [line] (dagtrans) -- (pgeomi);
		\path [line] (pgeomi) -- (partgen);
                \path [line] (f2d) -- (partgen);
		\path [line] (partgen) -- (ppartinp);
        	% Partisn adj gamma transport
                \node [code, below =0.4 of ppartinp] (partadjp) {PARTISN adjoint $\gamma$ transport};
        	\node [file, below =0.4 of partadjp] (padjf)
		{PARTISN$\phi_{\gamma, t_{mov}}^{+}$ file};
        	\node [script, below =0.4 of padjf] (pf2mesh) {PyNE.R2S \\
		convert PARTISN to voxel mesh};
        	\path [line] (ppartinp) -- (partadjp);
        	\path [line] (partadjp) -- (padjf);
        	\path [line] (padjf) -- (pf2mesh);
        	\path [line] (pf2mesh) -- (padjfvmesh);
		\node [file, below =0.4 of pf2mesh] (padjfvmesh)
		{$\phi_{\gamma,t_{mov}}^{+}$ voxel mesh};
	\node [script, below =0.4 of padjfvmesh] (conv_v_t){Convert voxel to tet};
        	\node [file, below =0.4 of
		conv_v_t](padjftmesh){$\phi_{\gamma,t_{mov}}^{+}$ tet mesh};
		\node [script, below =0.4 of padjftmesh](sum){Sum
		$\phi_{\gamma,t_{mov}}^{+}$over all time steps};
		\node [file, below =0.4 of sum](avgadjft){$\phi_{\gamma}^{+}$ tet mesh};
        	\node [script, below =0.4 of avgadjft] (conv_t_v){Convert tet to voxel};
		\path [line] (padjfvmesh) -- (conv_v_t);
		\path [line] (conv_v_t) -- (padjftmesh);
		\path [line] (padjftmesh) -- (sum);
		\path [line] (sum) -- (avgadjft);
		\path [line] (avgadjft) -- (conv_t_v);
		\path [line] (conv_t_v) -- (avgadjfv);
%}%phantom		
%\phantom{
        	\node [file, below =0.4 of conv_t_v](avgadjfv){${\phi_{\gamma}^{+}}$ voxel mesh};
	    	% Generate adj n source
        	\node [script, below =0.4 of avgadjfv] (gtcadis) {gtcadis.py}; 
		\node [file, left =0.9 of gtcadis] (ngeom) {DAGMC geom \\ at
		$t_{mov}$};
        	%\node [file, right =0.9 of gtcadis] (matlib) {ALARA \\
		%material library};
        	\node [file, right =0.9 of gtcadis] (calcT) {T matrix};
}
        	\node [file, below =0.4 of gtcadis, xshift=-1cm] (nadjs) {$q_{n}^{+}$ voxel mesh};
\phantom{      	\path [line] (ngeom) -- (gtcadis);
        	\path [line] (avgadjfv) -- (gtcadis);
        	\path [line] (calcT) -- (gtcadis);
        	\path [line] (gtcadis) -- (nadjs);
		% Generate VR parameters
		\node [script, below =0.4 of nadjs] (partgen) {Prepare PARTISN};
		\node [file, left =0.9 of partgen] (ngeom) {DAGMC geom \\at
		$t_{mov}=0$};
%                \node [file, right=0.9 of partgen] (nadjs) {$q_{n}^{+}$ voxel mesh};
                \node [file, below =0.4 of partgen] (npartinp) {PARTISN input file};
                \path [line] (ngeom) -- (partgen);
                \path [line] (nadjs) -- (partgen);
        	\path [line] (partgen) -- (npartinp);
}
        	% Partisn adj neutron transport
                \node [code, below =0.4 of nadjs] (partadjn) {PARTISN \\adjoint neutron transport};
\phantom{      	\node [file, below =0.4 of partadjn] (nadjf) {PARTISN $\phi_{n}^{+}$ file};
        	\node [script, below =0.4 of nadjf] (pf2mesh) {PyNE.R2S \\
		convert PARTISN to voxel mesh};
}
        	\node [file, below =0.4 of partadjn] (nadjfmesh) {$\phi_{n}^{+}$
		voxel mesh};
        	\path [line] (nadjs) -- (partadjn);
        	\path [line] (partadjn) -- (nadjf);
%        	\path [line] (nadjf) -- (pf2mesh);
%        	\path [line] (pf2mesh) -- (nadjfmesh);
		%Generate biased source and WW
		\node [script, below =0.4 of nadjfmesh] (cadis) {CADIS};
		\node [file, left =0.9 of cadis ] (unbiased) {Unbiased	$q_n$ mesh};
		\node [file, below =0.4 of cadis, xshift=-0.6cm] (biased) {Biased $q_n$ mesh};
		\node [file, right = 0.6 of biased] (wwinp) {Weight window\\
		mesh};
		\path [line] (unbiased) -- (cadis);
		\path [line] (nadjfmesh) -- (cadis);
		\path [line] (cadis) -- (biased);
		\path [line] (cadis) -- (wwinp);
        \end{tikzpicture}
	}%resizebox
	\end{figure}
\end{columns}
\end{frame}

%%%% 4.Gen VR params 
\begin{frame}{TGT-CADIS: Generate VR Parameters}
\begin{columns}
\column{0.45\textwidth}
	\begin{itemize}
%		\item{Calculate T of each voxel}
%		\item{Calculate adjoint neutron source}
		%\item{Perform adjoint neutron transport}
		\item{Generate biased source and weight window mesh via CADIS methodology}
	\end{itemize}
	\includegraphics[scale=0.22]{biased_src.png}
        \vspace{0.3cm}
	\includegraphics[scale=0.22]{ww_mesh.png}
		
\column{0.55\textwidth}
\begin{figure}
	\vspace{-12cm}
\centering

	\resizebox{!}{15cm}{
        \tikzstyle{script} = [ellipse, fill=UWGold!30, draw, text centered, node distance=3cm]
        \tikzstyle{code} = [ellipse, fill=UWRed!30, draw, text centered, node distance=3cm]
        \tikzstyle{file} = [rectangle, draw, text centered, node distance=3cm]
        \tikzstyle{line} = [draw, -latex']
        
        %\begin{tikzpicture}[node distance = 2 cm, auto]
        \begin{tikzpicture}[every text node part/.style={align=center}]
\phantom{        	%Generate adj gamma Partisn input
                \node [file] (pgeom0) {DAGMC geom at $t_{mov}=0$};
		\node [script, below = 0.4 of pgeom0] (dagtrans) {Transform tool};
		\node [script, below = 0.4 of dagtrans] (partgen) {Prepare
		PARTISN};
                \node [file, left = 0.9 of partgen] (pgeomi) {DAGMC geom
		at $t_{mov}$};
		\node [file, right=0.9 of partgen] (f2d) {Flux-to-dose \\ 
		conversion factors};
                \node [file, below =0.4 of partgen] (ppartinp) {PARTISN input
		file};
                \path [line] (pgeom0) -- (dagtrans);
		\path [line] (dagtrans) -- (pgeomi);
		\path [line] (pgeomi) -- (partgen);
                \path [line] (f2d) -- (partgen);
		\path [line] (partgen) -- (ppartinp);
        	% Partisn adj gamma transport
                \node [code, below =0.4 of ppartinp] (partadjp) {PARTISN adjoint $\gamma$ transport};
        	\node [file, below =0.4 of partadjp] (padjf)
		{PARTISN$\phi_{\gamma, t_{mov}}^{+}$ file};
        	\node [script, below =0.4 of padjf] (pf2mesh) {PyNE.R2S \\
		convert PARTISN to voxel mesh};
        	\path [line] (ppartinp) -- (partadjp);
        	\path [line] (partadjp) -- (padjf);
        	\path [line] (padjf) -- (pf2mesh);
        	\path [line] (pf2mesh) -- (padjfvmesh);
		\node [file, below =0.4 of pf2mesh] (padjfvmesh)
		{$\phi_{\gamma,t_{mov}}^{+}$ voxel mesh};
	\node [script, below =0.4 of padjfvmesh] (conv_v_t){Convert voxel to tet};
        	\node [file, below =0.4 of
		conv_v_t](padjftmesh){$\phi_{\gamma,t_{mov}}^{+}$ tet mesh};
		\node [script, below =0.4 of padjftmesh](sum){Sum
		$\phi_{\gamma,t_{mov}}^{+}$over all time steps};
		\node [file, below =0.4 of sum](avgadjft){$\phi_{\gamma}^{+}$ tet mesh};
        	\node [script, below =0.4 of avgadjft] (conv_t_v){Convert tet to voxel};
		\path [line] (padjfvmesh) -- (conv_v_t);
		\path [line] (conv_v_t) -- (padjftmesh);
		\path [line] (padjftmesh) -- (sum);
		\path [line] (sum) -- (avgadjft);
		\path [line] (avgadjft) -- (conv_t_v);
		\path [line] (conv_t_v) -- (avgadjfv);
%}%phantom		
%\phantom{
        	\node [file, below =0.4 of conv_t_v](avgadjfv){${\phi_{\gamma}^{+}}$ voxel mesh};
	    	% Generate adj n source
        	\node [script, below =0.4 of avgadjfv] (gtcadis) {gtcadis.py}; 
		\node [file, left =0.9 of gtcadis] (ngeom) {DAGMC geom \\ at
		$t_{mov}$};
        	%\node [file, right =0.9 of gtcadis] (matlib) {ALARA \\
		%material library};
        	\node [file, right =0.9 of gtcadis] (calcT) {T matrix};
}
        	\node [file, below =0.4 of gtcadis, xshift=-1cm] (nadjs) {$q_{n}^{+}$ voxel mesh};
\phantom{      	\path [line] (ngeom) -- (gtcadis);
        	\path [line] (avgadjfv) -- (gtcadis);
        	\path [line] (calcT) -- (gtcadis);
        	\path [line] (gtcadis) -- (nadjs);
		% Generate VR parameters
		\node [script, below =0.4 of nadjs] (partgen) {Prepare PARTISN};
		\node [file, left =0.9 of partgen] (ngeom) {DAGMC geom \\at
		$t_{mov}=0$};
%                \node [file, right=0.9 of partgen] (nadjs) {$q_{n}^{+}$ voxel mesh};
                \node [file, below =0.4 of partgen] (npartinp) {PARTISN input file};
                \path [line] (ngeom) -- (partgen);
                \path [line] (nadjs) -- (partgen);
        	\path [line] (partgen) -- (npartinp);
}
        	% Partisn adj neutron transport
                \node [code, below =0.4 of nadjs] (partadjn) {PARTISN \\adjoint neutron transport};
\phantom{      	\node [file, below =0.4 of partadjn] (nadjf) {PARTISN $\phi_{n}^{+}$ file};
        	\node [script, below =0.4 of nadjf] (pf2mesh) {PyNE.R2S \\
		convert PARTISN to voxel mesh};
}
        	\node [file, below =0.4 of partadjn] (nadjfmesh) {$\phi_{n}^{+}$
		voxel mesh};
        	\path [line] (nadjs) -- (partadjn);
        	\path [line] (partadjn) -- (nadjf);
%        	\path [line] (nadjf) -- (pf2mesh);
%        	\path [line] (pf2mesh) -- (nadjfmesh);
		%Generate biased source and WW
		\node [script, below =0.4 of nadjfmesh] (cadis) {CADIS};
		\node [file, left =0.9 of cadis ] (unbiased) {Unbiased	$q_n$ mesh};
		\node [file, below =0.4 of cadis, xshift=-0.6cm] (biased) {Biased $q_n$ mesh};
		\node [file, right = 0.6 of biased] (wwinp) {Weight window\\
		mesh};
		\path [line] (unbiased) -- (cadis);
		\path [line] (nadjfmesh) -- (cadis);
		\path [line] (cadis) -- (biased);
		\path [line] (cadis) -- (wwinp);
        \end{tikzpicture}
	}%resizebox
	\end{figure}
\end{columns}
\end{frame}


%\begin{frame}{TGT-CADIS: Generate VR Parameters}
%	\begin{itemize}
%		\item{}
%	\end{itemize}
%\begin{figure}
%\centering
%
%	\resizebox{!}{9cm}{
%        \tikzstyle{script} = [ellipse, fill=UWGold!30, draw, text centered, node distance=3cm]
%        \tikzstyle{code} = [ellipse, fill=UWRed!30, draw, text centered, node distance=3cm]
%        \tikzstyle{file} = [rectangle, draw, text centered, node distance=3cm]
%        \tikzstyle{line} = [draw, -latex']
%        
%        %\begin{tikzpicture}[node distance = 2 cm, auto]
%        \begin{tikzpicture}[every text node part/.style={align=center}]
%        	%Generate adj gamma Partisn input
%                \node [file] (pgeom0) {DAGMC geom at $t_{mov}=0$};
%		\node [script, below = 0.4 of pgeom0] (dagtrans) {DAGMC
%		transform tool};
%                \node [script, below = 0.4 of dagtrans] (partgen) {Prepare
%		PARTISN};
%                \node [file, left = 0.9 of partgen] (pgeomi) {DAGMC geom
%		at $t_{mov}$};
%                \node [file, right=0.9 of partgen] (f2d) {Flux-to-dose conversion factors};
%                \node [file, below =0.4 of partgen] (ppartinp) {PARTISN input file};
%                \path [line] (pgeom0) -- (dagtrans);
%		\path [line] (dagtrans) -- (pgeomi);
%                \path [line] (pgeomi) -- (partgen);
%                \path [line] (f2d) -- (partgen);
%        	\path [line] (partgen) -- (ppartinp);
%        	% Partisn adj gamma transport
%                \node [code, below =0.4 of ppartinp] (partadjp) {PARTISN adjoint $\gamma$ transport};
%        	\node [file, below =0.4 of partadjp] (padjf)
%		{PARTISN$\phi_{\gamma, t_{mov}}^{+}$ file};
%        	\node [script, below =0.4 of padjf] (pf2mesh) {PyNE.R2S \\
%		convert PARTISN to voxel mesh};
%		\node [file, below =0.4 of pf2mesh] (padjfvmesh)
%		{$\phi_{\gamma,t_{mov}}^{+}$ voxel mesh};
%        	\node [script, below =0.4 of padjfvmesh] (conv_v_t){Convert voxel to tet};
%        	\node [file, below =0.4 of
%		conv_v_t](padjftmesh){$\phi_{\gamma,t_{mov}}^{+}$ tet mesh};
%		\node [script, below =0.4 of padjftmesh](sum){Sum
%		$\phi_{\gamma,t_{mov}}^{+}$over all time steps};
%		\node [file, below =0.4 of sum](avgadjft){$\phi_{\gamma}^{+}$ tet mesh};
%        	\node [script, below =0.4 of avgadjft] (conv_t_v){Convert tet to voxel};
%		\node [file, below =0.4 of conv_t_v](avgadjfv){${\phi_{\gamma}^{+}}$ voxel mesh};
%        	\path [line] (ppartinp) -- (partadjp);
%        	\path [line] (partadjp) -- (padjf);
%        	\path [line] (padjf) -- (pf2mesh);
%        	\path [line] (pf2mesh) -- (padjfvmesh);
%		\path [line] (padjfvmesh) -- (conv_v_t);
%		\path [line] (conv_v_t) -- (padjftmesh);
%		\path [line] (padjftmesh) -- (sum);
%		\path [line] (sum) -- (avgadjft);
%		\path [line] (avgadjft) -- (conv_t_v);
%		\path [line] (conv_t_v) -- (avgadjfv);
%	    	% Generate adj n source
%        	\node [script, below =0.4 of avgadjfv] (gtcadis) {gtcadis.py}; 
%		\node [file, left =0.9 of gtcadis] (ngeom) {DAGMC geom at
%		$t_{mov}$};
%        	%\node [file, right =0.9 of gtcadis] (matlib) {ALARA material library};
%        	\node [file, right =0.9 of gtcadis] (calcT) {T matrix};
%        	\node [file, below =0.4 of gtcadis] (nadjs) {$q_{n}^{+}$ voxel mesh};
%        	\path [line] (ngeom) -- (gtcadis);
%        	\path [line] (avgadjfv) -- (gtcadis);
%        	\path [line] (calcT) -- (gtcadis);
%        	\path [line] (gtcadis) -- (nadjs);
%		
%        \end{tikzpicture}
%	}
%	\end{figure}
%\end{frame}

%\section{Experiment}
%\begin{frame}{Experiment: Toy Problem}
%\begin{columns}
%\column{0.48\textwidth}
%	\begin{itemize}
%		\item{Assess efficiency of TGT-CADIS}
%	\begin{itemize}
%		\item{Steel chamber with moving component}
%		\item{Incrementally add optimization}
%		\item{Calculate figure of merit (FOM) }
%	\end{itemize}
%%\item{Full-scale FES demonstration}
%	\end{itemize}
%\column{0.48\textwidth}
%	\multiinclude[<+->][format=jpg, graphics={ scale=0.25}]{newconfig}
%\end{columns}
%\end{frame}
%%\begin{frame}{Time-integrated R2S: Analog}
%%%% 1. No VR 
%\begin{frame}{Experiment: Toy Problem}
%	\begin{columns}
%		\column{0.3\textwidth}
%			Experimental Steps:
%	\begin{enumerate}
%	\item{No VR}
%%\phantom{
%%	\item{Photon VR}
%%	\item{Neutron and Photon VR}
%%}
%\end{enumerate}
%%\end{frame}
%\column{0.55\textwidth}
%	\begin{figure}
%        \centering
%
%        \tikzstyle{script} = [ellipse, fill=UWGold!30, draw, text centered,
%		node distance=2.8cm]
%        \tikzstyle{code} = [ellipse, fill=UWRed!30, draw, text centered, node
%		distance=2.8cm]
%        \tikzstyle{file} = [rectangle, draw, text centered, node distance=2.8cm]
%        \tikzstyle{line} = [draw, -latex]
%        \tikzstyle{arrow} = [thick, ->, >=stealth]
%        
%	\resizebox{!}{3.9cm}{
%        \begin{tikzpicture}[every text node part/.style={align=center}]
%		% Optimized neutron transport
%		\node [code] (nmc) {Monte Carlo \\neutron transport};
%                \node [file, left of = nmc, yshift=-0.9 cm, xshift=-1.91cm] (nflux) {Neutron flux};
%		\node [file, right of = nmc, yshift=-0.9cm, xshift=1.91cm]
%		(irrdec) {Irradiation and\\ decay scenario};
%		\node [script, below of = nmc, yshift=0.91cm] (act) {Activation analysis};
%                \node [file, left of = act, yshift=-0.8cm, xshift=-1.91cm] (psrc) {Photon source};
%		\node [code, below of = act, yshift=0.91cm] (pmc) {Monte Carlo\\
%		photon transport\\ at each time step};
%                \node [file, below of = pmc, yshift=0.5cm] (sdr) {Shutdown dose
%		rate\\ at each time step};
%		\path [line] (nmc) -- (nflux);
%                \path [line] (nflux) -- (act);
%                \path [line] (irrdec) -- (act);
%                \path [line] (act) -- (psrc);
%                \path [line] (psrc) -- (pmc);
%		\path [line] (pmc) -- (sdr);
%
%	\end{tikzpicture}
%        }
%%	\caption{R2S workflow \label{fig:r2s_flow}}
%\end{figure}
%\end{columns}
%\end{frame}
%
%%%%% 2. Photon VR 
%\begin{frame}{Experiment: Toy Problem}
%	\begin{columns}
%		\column{0.3\textwidth}
%			Experimental Steps:
%	\begin{enumerate}
%	\item{No VR}
%	\item{Photon VR: CADIS}
%%\phantom{
%%	\item{Neutron and Photon VR}
%%}
%\end{enumerate}
%%\end{frame}
%\column{0.55\textwidth}
%	\begin{figure}
%        \centering
%
%        \tikzstyle{script} = [ellipse, fill=UWGold!30, draw, text centered,
%		node distance=2.8cm]
%        \tikzstyle{code} = [ellipse, fill=UWRed!30, draw, text centered, node
%		distance=2.8cm]
%        \tikzstyle{file} = [rectangle, draw, text centered, node distance=2.8cm]
%        \tikzstyle{line} = [draw, -latex]
%        \tikzstyle{arrow} = [thick, ->, >=stealth]
%        
%	\resizebox{!}{4.5cm}{
%        \begin{tikzpicture}[every text node part/.style={align=center}]
%		% Optimized neutron transport
%		\node [code] (nmc) {Monte Carlo \\neutron transport};
%                \node [file, left of = nmc, yshift=-0.9 cm, xshift=-1.91cm] (nflux) {Neutron flux};
%		\node [file, right of = nmc, yshift=-0.9cm, xshift=1.91cm]
%		(irrdec) {Irradiation and\\ decay scenario};
%		\node [script, below of = nmc, yshift=0.91cm] (act) {Activation analysis};
%		\node [script, below of = act, yshift=0.91cm] (cadis) {CADIS};
%                \node [file, left of = cadis, yshift=-0.8cm, xshift=-1.91cm] (psrc) {Biased \\photon source};
%                \node [file, right of = cadis, yshift=-0.8cm, xshift=1.91cm] (wwinp) {Weight windows};
%		\node [code, below of = cadis, yshift=0.5cm] (pmc) {Monte Carlo\\
%		photon transport\\ at each time step};
%                \node [file, below of = pmc, yshift=0.5cm] (sdr) {Shutdown dose
%		rate\\ at each time step};
%		\path [line] (nmc) -- (nflux);
%                \path [line] (nflux) -- (act);
%                \path [line] (irrdec) -- (act);
%                \path [line] (act) -- (cadis);
%                \path [line] (cadis) -- (psrc);
%                \path [line] (cadis) -- (wwinp);
%                \path [line] (psrc) -- (pmc);
%                \path [line] (wwinp) -- (pmc);
%		\path [line] (pmc) -- (sdr);
%
%	\end{tikzpicture}
%        }
%%	\caption{R2S workflow \label{fig:r2s_flow}}
%\end{figure}
%\end{columns}
%\end{frame}
%
%%%% 3. Photon and Neutron VR
%\begin{frame}{Experiment: Toy Problem}
%	\begin{columns}
%		\column{0.3\textwidth}
%		Experimental Steps:
%	\begin{enumerate}
%	\item{No VR}
%	\item{Photon VR: CADIS}
%	\item{Neutron and Photon VR: TGT-CADIS, CADIS}
%\end{enumerate}
%%\end{frame}
%\column{0.55\textwidth}
%	\begin{figure}
%        \centering
%
%        \tikzstyle{script} = [ellipse, fill=UWGold!30, draw, text centered,
%		node distance=2.8cm]
%        \tikzstyle{code} = [ellipse, fill=UWRed!30, draw, text centered, node
%		distance=2.8cm]
%        \tikzstyle{file} = [rectangle, draw, text centered, node distance=2.8cm]
%        \tikzstyle{line} = [draw, -latex]
%        \tikzstyle{arrow} = [thick, ->, >=stealth]
%        
%	\resizebox{!}{6cm}{
%        \begin{tikzpicture}[every text node part/.style={align=center}]
%		% Optimized neutron transport
%		\node [script, above of = nmc, yshift=-0.91cm] (gtcadis)
%		{TGT-CADIS};
%                \node [file, left of = gtcadis, yshift=-0.8cm, xshift=-1.91cm]
%		(nsrc) {Biased \\neutron source};
%                \node [file, right of = gtcadis, yshift=-0.8cm, xshift=1.91cm]
%		(nwwinp) {Weight windows};
%		\node [code] (nmc) {Monte Carlo \\neutron transport};
%                \node [file, left of = nmc, yshift=-0.9 cm, xshift=-1.91cm] (nflux) {Neutron flux};
%		\node [file, right of = nmc, yshift=-0.9cm, xshift=1.91cm]
%		(irrdec) {Irradiation and\\ decay scenario};
%		\node [script, below of = nmc, yshift=0.91cm] (act) {Activation analysis};
%		\node [script, below of = act, yshift=0.91cm] (cadis) {CADIS};
%                \node [file, left of = cadis, yshift=-0.8cm, xshift=-1.91cm] (psrc) {Biased \\photon source};
%                \node [file, right of = cadis, yshift=-0.8cm, xshift=1.91cm] (wwinp) {Weight windows};
%		\node [code, below of = cadis, yshift=0.5cm] (pmc) {Monte Carlo\\
%		photon transport\\ at each time step};
%                \node [file, below of = pmc, yshift=0.5cm] (sdr) {Shutdown dose
%		rate\\ at each time step};
%		\path [line] (gtcadis) -- (nsrc);
%		\path [line] (gtcadis) -- (nwwinp);
%		\path [line] (nsrc) -- (nmc);
%		\path [line] (nwwinp) -- (nmc);
%		\path [line] (nmc) -- (nflux);
%                \path [line] (nflux) -- (act);
%                \path [line] (irrdec) -- (act);
%                \path [line] (act) -- (cadis);
%                \path [line] (cadis) -- (psrc);
%                \path [line] (cadis) -- (wwinp);
%                \path [line] (psrc) -- (pmc);
%                \path [line] (wwinp) -- (pmc);
%		\path [line] (pmc) -- (sdr);
%
%	\end{tikzpicture}
%        }
%%	\caption{R2S workflow \label{fig:r2s_flow}}
%\end{figure}
%\end{columns}
%\end{frame}
%
%\begin{frame}{Full-scale Demonstration}
%\begin{figure}
%	\centering
%	\includegraphics[scale=0.3]{iter.jpg}
%	\caption{Cutaway view of ITER drawing.}
%\end{figure}
%
%\end{frame}
%
%%\begin{frame}{Error Calculation}
%%\end{frame}


%\section{Final Thoughts}
%\subsection{Assumptions}
%\begin{frame}{Assumptions}
%			\begin{itemize}
%				\item{Photon transport occurs much faster than
%					geometry movement $\therefore$
%					reasonable to do quasi-static
%					simulation}
%				\item{Period of geometry movement is short
%					enough that the photon source will not
%					change appreciably $\therefore$ can use
%					same photon source for all time steps
%					of geometry movement}
%			\end{itemize}
%	\end{frame}
%%\subsection{Challenges}
%%\begin{frame}{Limitations}
%%\begin{itemize}
%%\item{Geometry movement tools do not treat object kinetics}
%%   \begin{itemize}
%%   \item{User must be careful to not cause overlap in components}
%%   \end{itemize}
%%\item{Can only move components that do not share a surface with any other components}
%%\end{itemize}
%%\end{frame}
%
%\begin{frame}{Challenges}
%			\begin{itemize}
%				\item{Depending on complexity of model and
%					fidelity of time resolution, can amass
%					large number of CAD geometry files,
%					volume mesh tally files}
%				\item{Need to optimize this workflow in order
%					to keep file storage at minimum}
%			\end{itemize}
%\end{frame}
\section{Summary}
\begin{frame}{Summary}
	\begin{itemize}
		\item{Accurate quantification of the SDR during maintenance
			procedures is crucial to the design and operation of
			FES}
%		\item{GT-CADIS has proven to optimize SDR analysis in
%			static FES}
%\vspace{0.5cm}
		\item{\textbf{TGT-CADIS} aims to provide the capabilities necessary to
			\textbf{optimize} the calculation of the \textbf{SDR}
			%at various time points 
			during operations that involve activated components
			\textbf{moving} around the facility}
                 \item{Next step: Compare efficiency of TGT-CADIS with analog and other VR methods}
	\end{itemize}
\end{frame}

\begin{frame}[c]
	\frametitle{\tiny{.}}
	\begin{center}
	{\Huge Questions?}
	\end{center}
\end{frame}

%\begin{frame}{TODO}
%	\begin{itemize}
%		\item{add error plots}
%		\item{labels on tet mesh time slices}
%		\item{Overall alignment/sizing/spacing}
%		\item{Add more on moving geom progress, movie}
%		\item{OBB tree optimization}
%	\end{itemize}
%\end{frame}
\bibliographystyle{unsrt}
\bibliography{thebibliography}
\end{document}

