\documentclass{beamer}
\usepackage{tikz}
\usetikzlibrary{shapes,arrows}

\usetheme[white]{Wisconsin}
\begin{document}
\newcommand*{\alphabet}{ABCDEFGHIJKLMNOPQRSTUVWXYZabcdefghijklmnopqrstuvwxyz}
\newlength{\highlightheight}
\newlength{\highlightdepth}
\newlength{\highlightmargin}
\setlength{\highlightmargin}{2pt}
\settoheight{\highlightheight}{\alphabet}
\settodepth{\highlightdepth}{\alphabet}
\addtolength{\highlightheight}{\highlightmargin}
\addtolength{\highlightdepth}{\highlightmargin}
\addtolength{\highlightheight}{\highlightdepth}
\newcommand*{\Highlight}{\rlap{\textcolor{HighlightBackground}{\rule[-\highlightdepth]{\linewidth}{\highlightheight}}}}

%% Slides

%% Title
\title{Variance Reduction for Multiphysics Analysis \\ of Moving Systems}
%\subtitle{Graduate Student Seminar}
\date{ Dec. 8, 2017}
\author{Chelsea D'Angelo}
%\institute{University of Wisconsin}
\begin{frame}
	\titlepage
\end{frame}

%%Background
\begin{frame}
\frametitle{Shutdown Dose Rate Analysis}

\begin{columns}

\column{0.5\textwidth}
\begin{itemize}
	\item{Fusion Energy Systems (FES)}
		\begin{itemize}
			\item{Burning plasma, D-T fusion}
			\item{$^{2}H + ^{3}H \rightarrow ^{4}_{2}He + n$}
		\end{itemize}
	\item{Neutrons penetrate deeply into system components,
				causing activation}
	\item{Radioisotopes persist long after shutdown}
	\item{Important to quantify the dose caused by decay photons}
\end{itemize}
\column{0.5\textwidth}
\begin{figure}
	\centering
	\includegraphics[scale=0.22]{iter.jpg}
	\caption{Cutaway view of ITER drawing.}
\end{figure}

\end{columns}
		
\end{frame}

\begin{frame}
\frametitle{SDR Analysis: Maintenance Operations}
	\begin{itemize}
		\item{During a maintenance procedure:}
			\begin{itemize}
				\item{Need to move component(s) around facility}
				\item{Interested in SDR at a particular
					location}
				\item{SDR will change as a function of the
					activated component's position over
					time}
			\end{itemize}
	\end{itemize}
\end{frame}

\begin{frame}
\frametitle{SDR Solution Method}
\begin{itemize}
\item{Rigorous 2-Step Method (R2S)}
%\begin{itemize}
%\item{Monte Carlo neutron transport}
%\item{Activation analysis}
%\item{Monte Carlo photon transport}
%\end{itemize}
\begin{figure}
\centering

        \tikzstyle{script} = [ellipse, fill=UWGold!30, draw, text centered, node distance=3cm]
        \tikzstyle{code} = [ellipse, fill=UWRed!30, draw, text centered, node distance=3cm]
        \tikzstyle{file} = [rectangle, draw, text centered, node distance=3cm]
        \tikzstyle{line} = [draw, -latex]
        \tikzstyle{arrow} = [thick, ->, >=stealth]
        
        \begin{tikzpicture}[node distance = 2 cm, auto]
		% Optimized neutron transport
		\node [code] (nmc) {Monte Carlo neutron transport};
                \node [file, left of = nmc, yshift=-0.9 cm, xshift=-1.8cm] (nflux) {Neutron flux};
		\node [script, below of = nmc, yshift=1.2cm] (act) {Activation analysis};
                \node [file, left of = act, yshift=-0.9cm, xshift=-1.8cm] (psrc) {Photon source};
		\node [code, below of = act, yshift=1.2cm] (pmc) {Monte Carlo photon transport};
                \node [file, below of = pmc, yshift=1.4cm] (sdr) {Shutdown dose rate};
		\path [line] (nmc) -- (nflux);
                \path [line] (nflux) -- (act);
                \path [line] (act) -- (psrc);
                \path [line] (psrc) -- (pmc);
		\path [line] (pmc) -- (sdr);

	\end{tikzpicture}

%	\caption{R2S workflow \label{fig:r2s_flow}}
\end{figure}

\end{itemize}
\end{frame}

%!!!!! Should add in pic of ITER flux
\begin{frame}
\frametitle{Monte Carlo Radiation Transport}
\begin{columns}

\column{0.7\textwidth}
\begin{itemize}
\item{Monte Carlo (MC) analysis of fusion energy systems is challenging due to the highly attenuating structural materials}
   \begin{itemize}
   \item{Results scored in regions that have low particle flux, have higher statistical uncertainty}
   \end{itemize}
\item{To decrease statistical uncertainty:}
\begin{itemize}
\item{Increase number of histories}
\item{Use variance reduction (VR) techniques}
\end{itemize}
\end{itemize}

\column{0.3\textwidth}
\begin{figure}
	\centering
	\includegraphics[scale=0.5]{iter_pflux.jpg}
	\caption{Photon flux in ITER tokamak building.}
\end{figure}

\end{columns}
\end{frame}

\begin{frame}
\frametitle{MC Variance Reduction Techniques}
\begin{itemize}
\item{Techniques to modify particle behavior}
\begin{itemize}
\item{Preferentially sample events that will contribute to results of interest}
\end{itemize}
\item{Statistical weight of particles is adjusted to keep playing a fair game}
\end{itemize}
\end{frame}

\begin{frame}
\frametitle{Hybrid Deterministic/MC VR Methods}
\begin{itemize}
\item{Use \textbf{deterministic} estimate of the adjoint flux, $\Psi^+$, to generate \textbf{Monte Carlo} biasing prarameters}
\end{itemize}
%\item{ 
%}
\begin{itemize}
\item{Adjoint flux can define the importance of regions of phase space to the detector response}
\end{itemize}

\vspace{0.2cm}
       \begin{equation}
        H^+\Psi^+ = q^+
       \end{equation} 
       \begin{equation}
	H^{+} = -\widehat{\Omega} \cdot \nabla +
	    \sigma_{t}(\overrightarrow{r},E) - 
		\int_{0}^{\infty} dE'
		\int_{4\pi} d\Omega'
		\sigma_{s}( \overrightarrow{r}, E 
		\rightarrow E', \widehat{\Omega} 
		\rightarrow \widehat{\Omega}' )
       \end{equation}
\end{frame}

\begin{frame}
\frametitle{Hybrid Deterministic/MC VR Methods}
\begin{itemize}
\item{Consistent Adjoint Driven Importance Sampling (CADIS)}
  \begin{itemize}
  \item{Use the adjoint flux to generate source and transport biasing parameters in a consistent manner}
    \begin{itemize}
    \item{Source: sample from biased PDF}
    \item{Transport: weight windows to control particle flow}
    \end{itemize}
  \end{itemize}
\end{itemize}
\end{frame}

\begin{frame}
\frametitle{Variance Reduction for SDR Analysis}
\begin{itemize}
\item{VR for photon transport step is straightforward}
  \begin{itemize}
  \item{Can use CADIS method}
    \begin{itemize}
    \item{Flux-to-dose conversion factors define adjoint source}
    \end{itemize}
  \end{itemize}
\item{VR for neutron transport step is more complicated}
    \begin{itemize}
    \item{Biasing function needs to capture}
      \begin{enumerate}
        \item{Potential of regions to become activated}
        \item{Potential to produce photons that will contribute to the SDR}
      \end{enumerate}
    \end{itemize}
\end{itemize}
\end{frame}


\begin{frame}
\frametitle{Variance Reduction for SDR Analysis}
\begin{itemize}
\item{Multi-Step (MS)-CADIS}
  \begin{itemize}
  \item {VR method to optimize the initial radiation transport step of a coupled, multi-step process}
  \item{When applied to SDR analysis, MS-CADIS will optimize the neutron transport}
    \begin{itemize}
    \item{Use a function that represents the   importance of the neutrons to the final dose rate}
    \end{itemize}
  \end{itemize}
\end{itemize}
\end{frame}

\begin{frame}
\frametitle{VR for SDR Analysis: GT-CADIS}
\begin{itemize}
\item{Groupwise Transmutation (GT)-CADIS}
  \begin{itemize}
  \item{Implementation of MS-CADIS specifically for SDR analysis}
  \item{Provides method to calculate optimal adjoint neutron source, $q_n^+$}
    \begin{itemize}
    \item{This is used as the source in the deterministic calculation of the    adjoint neutron flux, $\phi_{n}^{+}$, used as the importance function}
    
    \end{itemize}
  \end{itemize}
\end{itemize}
\centering
{$q_n^+(E_n)=\int_{E_p} T(E_n, E_p) \phi_p^+(E_p) dE_p$}
\end{frame}


\begin{frame}
\frametitle{VR for SDR Analysis: GT-CADIS}
\begin{itemize}
\item{Coupling term, T, defined by}
\end{itemize}

%\begin{equation}
\centering
{$q_p(E_p) = \int_{E_n} T(E_n, E_p) \phi_n(E_n) dE_n$}
%\end{equation}


\end{frame}


\end{document}
