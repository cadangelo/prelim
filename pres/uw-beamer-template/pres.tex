\documentclass{beamer}


\usetheme[white]{Wisconsin}
\begin{document}
\newcommand*{\alphabet}{ABCDEFGHIJKLMNOPQRSTUVWXYZabcdefghijklmnopqrstuvwxyz}
\newlength{\highlightheight}
\newlength{\highlightdepth}
\newlength{\highlightmargin}
\setlength{\highlightmargin}{2pt}
\settoheight{\highlightheight}{\alphabet}
\settodepth{\highlightdepth}{\alphabet}
\addtolength{\highlightheight}{\highlightmargin}
\addtolength{\highlightdepth}{\highlightmargin}
\addtolength{\highlightheight}{\highlightdepth}
\newcommand*{\Highlight}{\rlap{\textcolor{HighlightBackground}{\rule[-\highlightdepth]{\linewidth}{\highlightheight}}}}


%% Slides

%% Title
\title{Variance Reduction for Multiphysics Analysis \\ of Moving Systems}
%\subtitle{Graduate Student Seminar}
\date{ Dec. 8, 2017}
\author{Chelsea D'Angelo}
%\institute{University of Wisconsin}
\begin{frame}
	\titlepage
\end{frame}

%%Background
\begin{frame}
\frametitle{Shutdown Dose Rate Analysis}

\begin{columns}

\column{0.5\textwidth}
\begin{itemize}
	\item{Fusion Energy Systems (FES)}
		\begin{itemize}
			\item{Burning plasma, D-T fusion}
			\item{$^{2}H + ^{3}H \rightarrow ^{4}_{2}He + n$}
		\end{itemize}
	\item{Neutrons penetrate deeply into system components,
				causing activation}
	\item{Radioisotopes persist long after shutdown}
	\item{Important to quantify the dose caused by decay photons}
\end{itemize}
\column{0.5\textwidth}
\begin{figure}
	\centering
	\includegraphics[scale=0.22]{iter.jpg}
	\caption{Cutaway view of ITER drawing.}
\end{figure}

\end{columns}
		
\end{frame}

\begin{frame}
\frametitle{SDR Analysis: Maintenance Operations}
	\begin{itemize}
		\item{During a maintenance procedure:}
			\begin{itemize}
				\item{Need to move component(s) around facility}
				\item{Interested in SDR at a particular
					location}
				\item{SDR will change as a function of the
					activated component's position over
					time}
			\end{itemize}
	\end{itemize}
\end{frame}

\end{document}
