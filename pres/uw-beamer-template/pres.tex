\documentclass{beamer}
\usepackage{tikz}
\usetikzlibrary{shapes,arrows}
\usetikzlibrary{shapes.multipart}
\usepackage{animate}
\usepackage{xmpmulti}

\usetheme[white]{Wisconsin}
\begin{document}
\newcommand*{\alphabet}{ABCDEFGHIJKLMNOPQRSTUVWXYZabcdefghijklmnopqrstuvwxyz}
\newlength{\highlightheight}
\newlength{\highlightdepth}
\newlength{\highlightmargin}
\setlength{\highlightmargin}{2pt}
\settoheight{\highlightheight}{\alphabet}
\settodepth{\highlightdepth}{\alphabet}
\addtolength{\highlightheight}{\highlightmargin}
\addtolength{\highlightdepth}{\highlightmargin}
\addtolength{\highlightheight}{\highlightdepth}
\newcommand*{\Highlight}{\rlap{\textcolor{HighlightBackground}{\rule[-\highlightdepth]{\linewidth}{\highlightheight}}}}

%% Slides

%% Title
\title{Variance Reduction for Multi-physics Analysis \\ of Moving Systems}
%\subtitle{Graduate Student Seminar}
\date{ Dec. 8, 2017}
\author{Chelsea D'Angelo}
%\institute{University of Wisconsin}
\begin{frame}
	\titlepage
\end{frame}

%%Background
\begin{frame}
	\frametitle{Shutdown Dose Rate (SDR) Analysis}

\begin{columns}
\column{0.5\textwidth}
\begin{itemize}
	\item{Fusion Energy Systems (FES)}
		\begin{itemize}
			\item{Burning plasma, D-T fusion}
			\item{$^{2}_{1}H + ^{3}_{1}H \rightarrow ^{4}_{2}He + ^{1}_{0} n$}
		\end{itemize}
	\item{Neutrons penetrate deeply into system components,
				causing activation}
	\item{Radioisotopes persist long after shutdown}
	\item{Important to quantify the dose caused by decay photons}
\end{itemize}
\column{0.5\textwidth}
\begin{figure}
	\centering
	\includegraphics[scale=0.22]{iter.jpg}
	\caption{Cutaway view of ITER drawing.}
\end{figure}

\end{columns}
		
\end{frame}

\begin{frame}
\frametitle{SDR Analysis: Maintenance Operations}
	\begin{itemize}
		\item{During a maintenance procedure:}
			\begin{itemize}
				\item{Need to move component(s) around facility}
				\item{Interested in SDR at a particular
					location}
				\item{SDR will change as a function of the
					activated component's position over
					time}
			\end{itemize}
	\end{itemize}
	\begin{center}
	%\animategraphics[loop,controls,width=\textwidth]{5}{config-}{0}{5}
	\transduration<0-5>{0}
	\multiinclude[<+->][format=jpg, graphics={ scale=0.3}]{config}
	\end{center}
\end{frame}

\begin{frame}{Goal}
	\textbf{Optimize the MC radiation transport used to calculate the
	dose
	rate at a particular location as activated components are being
	transported around the facility.}
\end{frame}


\begin{frame}
\frametitle{SDR Solution Methods}

\begin{minipage}{0.49\textwidth}
\begin{itemize}
\item{Direct 1-Step Method (D1S)}
	\vspace{1cm}
	\begin{figure}
        \centering

        \tikzstyle{script} = [ellipse, fill=UWGold!30, draw, text centered,
		node distance=3.0cm]
        \tikzstyle{code} = [ellipse, fill=UWRed!30, draw, text centered, node
		distance=3.0cm]
        \tikzstyle{file} = [rectangle, draw, text centered, node distance=3.0cm]
        \tikzstyle{line} = [draw, -latex]
        \tikzstyle{arrow} = [thick, ->, >=stealth]
        
	\resizebox{5cm}{4cm}{
        \begin{tikzpicture}[every text node part/.style={align=center}]
		% Optimized neutron transport
		\node [code] (mc) {Monte Carlo \\ neutron/photon transport};
                \node [file, above of = mc, yshift=-0.3cm] (mod) {Modified\\ cross sections};
                \node [file, below of = mc, yshift=0.3cm] (sdr) {Shutdown dose rate};
		\path [line] (mod) -- (mc);
		\path [line] (mc) -- (sdr);

	\end{tikzpicture}
		}
        \end{figure}
\end{itemize}
\end{minipage}
\hfill
\begin{minipage}{0.49\textwidth}
\begin{itemize}
\item{Rigorous 2-Step Method (R2S)}
	\vspace{1cm}
	\begin{figure}
        \centering

        \tikzstyle{script} = [ellipse, fill=UWGold!30, draw, text centered,
		node distance=3.0cm]
        \tikzstyle{code} = [ellipse, fill=UWRed!30, draw, text centered, node
		distance=3.0cm]
        \tikzstyle{file} = [rectangle, draw, text centered, node distance=3.0cm]
        \tikzstyle{line} = [draw, -latex]
        \tikzstyle{arrow} = [thick, ->, >=stealth]
        
	\resizebox{5cm}{4cm}{
        \begin{tikzpicture}[every text node part/.style={align=center}]
		% Optimized neutron transport
		\node [code] (nmc) {Monte Carlo \\ neutron transport};
                \node [file, left of = nmc, yshift=-0.9 cm, xshift=-1.0cm] (nflux) {Neutron flux};
		\node [file, right of = nmc, yshift=-0.9cm, xshift=1.0cm]
		(irrdec) {Irradiation and \\ decay scenario};
		\node [script, below of = nmc, yshift=1.2cm] (act) {Activation analysis};
                \node [file, left of = act, yshift=-0.9cm, xshift=-1.0cm] (psrc) {Photon source};
		\node [code, below of = act, yshift=1.2cm] (pmc) {Monte Carlo\\ photon transport};
                \node [file, below of = pmc, yshift=1.2cm] (sdr) {Shutdown dose rate};
		\path [line] (nmc) -- (nflux);
                \path [line] (nflux) -- (act);
                \path [line] (irrdec) -- (act);
                \path [line] (act) -- (psrc);
                \path [line] (psrc) -- (pmc);
		\path [line] (pmc) -- (sdr);

	\end{tikzpicture}
        }
%	\caption{R2S workflow \label{fig:r2s_flow}}
\end{figure}
\end{itemize}
\end{minipage}

\end{frame}

\begin{frame}{Pros/cons of D1S and R2S}
\end{frame}

\begin{frame}{Why MC over Deterministic}
\end{frame}

\begin{frame}
\frametitle{Monte Carlo Radiation Transport}
\begin{columns}
\column{0.7\textwidth}
\begin{itemize}
	\item{Monte Carlo (MC) analysis of fusion energy systems is:}
	\begin{itemize}
		\item{Accurate for large, complex models}
		\item{Challenging due to the highly attenuating structural materials}
                  \begin{itemize}
                  \item{Results scored in regions that have low particle flux, have higher statistical uncertainty}
                  \end{itemize}
	\end{itemize}
\item{To decrease statistical uncertainty:}
\begin{itemize}
\item{Increase number of histories}
\item{Use variance reduction (VR) techniques}
\end{itemize}
\end{itemize}

\column{0.3\textwidth}
\begin{figure}
	\centering
	\includegraphics[scale=0.5]{iter_pflux.jpg}
	\caption{Photon flux in ITER tokamak building.}
\end{figure}

\end{columns}
\end{frame}

\begin{frame}{Stat error in MC and intro VR}
\end{frame}

\begin{frame}
\frametitle{MC Variance Reduction Techniques}
\begin{itemize}
\item{Techniques to modify particle behavior}
\begin{itemize}
	\item{\textbf{Goal}: preferentially sample events that will contribute to results of interest}
\end{itemize}
\item{Statistical weight of particles is adjusted to keep playing a fair game}
\end{itemize}
\end{frame}

\begin{frame}
\frametitle{Hybrid Deterministic/MC VR Methods: CADIS}
\begin{itemize}
\item{Consistent Adjoint Driven Importance Sampling (CADIS)}
\item{Use \textbf{deterministic} estimate of the adjoint flux, $\Psi^+$, to
	generate \textbf{Monte Carlo} VR parameters}
\item{Adjoint flux can define the importance of regions of phase space to the detector response}
\end{itemize}


\vspace{0.2cm}
       \begin{equation}
        H^+\Psi^+ = q^+
       \end{equation} 
       \begin{equation}
	H^{+} = -\widehat{\Omega} \cdot \nabla +
	    \sigma_{t}(\overrightarrow{r},E) - 
		\int_{0}^{\infty} dE'
		\int_{4\pi} d\Omega'
		\sigma_{s}( \overrightarrow{r}, E 
		\rightarrow E', \widehat{\Omega} 
		\rightarrow \widehat{\Omega}' )
       \end{equation}
\begin{itemize}
\item{Adjoint transport $\sim$ "backwards" transport}
\item{Detector response function is the adjoint source}
\end{itemize}
\end{frame}

\begin{frame}
\frametitle{Hybrid Deterministic/MC VR Methods: CADIS}
  \begin{itemize}
  \item{Use the adjoint flux to generate MC source and transport biasing parameters in a consistent manner}
    \begin{itemize}
    \item{Source: sample from biased PDF}
    \item{Transport: weight windows to control particle flow}
%    \item{Particles are born within weight windows}
    \end{itemize}
  \end{itemize}
	\begin{columns}
        \column{0.5\textwidth}
		\begin{itemize}
			\item{Uniform vs. biased source distribution}
		\end{itemize}
                \begin{figure}
		\centering
		\includegraphics[scale=0.32]{bsrc_ex.jpg}
		\end{figure}
        \column{0.5\textwidth}
		\begin{itemize}
			\item{Splitting/rouletting with weight windows}
		\end{itemize}
                \begin{figure}
		\centering
		\includegraphics[scale=0.51]{ww_ex2.png}
		\end{figure}
	\end{columns}
\end{frame}

\begin{frame}
\frametitle{Variance Reduction for SDR Analysis}
\begin{itemize}
	\item{VR for \textbf{photon} transport step is \textbf{straightforward}}
  \begin{itemize}
  \item{Can use CADIS method to direct photons towards detector}
    \begin{itemize}
    \item{Flux-to-dose-rate conversion factors define adjoint source}
    \end{itemize}
  \end{itemize}
\item{VR for \textbf{neutron} transport step is more \textbf{complicated}}
    \begin{itemize}
    \item{Biasing function needs to capture}
      \begin{enumerate}
        \item{Potential of regions to become activated}
        \item{Potential to produce photons that will contribute to the SDR}
      \end{enumerate}
    \item{Can use CADIS if we can construct adjoint source that will fulfill
	    these criteria}
    \end{itemize}
\end{itemize}
\end{frame}


\begin{frame}
\frametitle{Variance Reduction for SDR Analysis: MS-CADIS}
\begin{itemize}
\item{Multi-Step (MS)-CADIS}
  \begin{itemize}
  \item {VR method to optimize the initial radiation transport step of a coupled, multi-step process}
  \item{When applied to SDR analysis, MS-CADIS will optimize the neutron transport}
    \begin{itemize}
    \item{Use a function that represents the   importance of the neutrons to
	    the final dose rate}
    \end{itemize}
  \end{itemize}
\end{itemize}
 \centering
	\begin{equation}
	\int_{\overrightarrow{r}} \int_{E_n} \phi_n(\overrightarrow{r}, E_n)
	q_n^+(\overrightarrow{r}, E_n) d\overrightarrow{r} dE_n = SDR
	\end{equation}
% \vspace{0.03 cm}

 \centering
	\begin{equation}
		SDR = \int_{\overrightarrow{r}} \int_{E_{\gamma}} 
		\phi_{\gamma}^+(\overrightarrow{r}, E_{\gamma})
		q_{\gamma}(\overrightarrow{r}, E_{\gamma})
		dr dE_{\gamma} 
	\end{equation}
\end{frame}

\begin{frame}
\frametitle{Variance Reduction for SDR Analysis: MS-CADIS}
	\begin{itemize}
		\item{Combining these equations:}
	\end{itemize}
	\begin{centering}
	\begin{equation}
	\int_{\overrightarrow{r}} \int_{E_n} \phi_n(\overrightarrow{r}, E_n)
	q_n^+(\overrightarrow{r}, E_n) d\overrightarrow{r} dE_n =
		\int_{\overrightarrow{r}} \int_{E_{\gamma}} 
		\phi_{\gamma}^+(\overrightarrow{r}, E_{\gamma})
		q_{\gamma}(\overrightarrow{r}, E_{\gamma})
		d\overrightarrow{r} dE_{\gamma} 
	\end{equation}
	\end{centering}
	\begin{itemize}
		\item{To solve for the adjoint neutron source, $q_n^+$, a
			relationship between $q_{\gamma}$
			and $\phi_n$ is required}
	\end{itemize}
\centering
\begin{equation}
	q_{\gamma}(E_{\gamma}) = \int_{E_n} T(E_n, E_{\gamma}) \phi_n(E_n) dE_n
\end{equation}


\end{frame}
		
\begin{frame}
\frametitle{Variance Reduction for SDR Analysis: GT-CADIS}
\begin{itemize}
\item{Groupwise Transmutation (GT)-CADIS}
  \begin{itemize}
  \item{Implementation of MS-CADIS specifically for SDR analysis}
  \item{Provides method to calculate optimal adjoint neutron source, $q_n^+$,
	  by first calculating, T}
  \item{Calculate T:}
	  \begin{enumerate}
		  \item{Irradiate each material with neutrons from a single
			  energy group, g}
		  \item{Record resulting photon emission in each energy group,
			  h}
	  \end{enumerate}
  \end{itemize}

\end{itemize}
\centering
	\begin{equation}
		T_{g,h} = \frac{q_{\gamma,h}(\phi_{n,g})}{\phi_{n,g}}
	\end{equation}

\end{frame}


\begin{frame}
\frametitle{Variance Reduction for SDR Analysis: GT-CADIS}

\begin{itemize}
	\item{Use T to solve for adjoint neutron source:}
\end{itemize}
\centering
\begin{equation}
	q_n^+(E_n)=\int_{E_{\gamma}} T(E_n, E_{\gamma}) \phi_{\gamma}^+(E_{\gamma})
	dE_{\gamma}
\end{equation}

%\begin{itemize}
%\item{This is used as the source in the deterministic calculation of the    adjoint neutron flux, $\phi_{n}^{+}$}
%\end{itemize}


\end{frame}

\begin{frame}
\frametitle{Variance Reduction for SDR Analysis: GT-CADIS}

\begin{itemize}
\item{GT-CADIS workflow}
\end{itemize}
	\vspace{1cm}
\begin{figure}
\centering

        \tikzstyle{script} = [ellipse, fill=UWGold!30, draw, text centered,
	node distance=0.9cm]
        \tikzstyle{code} = [ellipse, fill=UWRed!30, draw, text centered, node
	distance=0.9cm]
        \tikzstyle{file} = [rectangle, draw, text centered, node distance=0.9cm]
        \tikzstyle{line} = [draw, -latex]
        \tikzstyle{arrow} = [thick, ->, >=stealth]
       
	\resizebox{!}{4cm}{
        \begin{tikzpicture}[node distance = 0.5 cm, auto]
		% Optimized neutron transport
		\node [script] (adjnsrc) {Calculate adjoint
		neutron source};
		\node [code, above of = adjnsrc, xshift=-3 cm] (adjp) {Adjoint photon transport};
		%\node [file] (adjpflux) {Adjoint photon flux};
		\node [script, right of = adjp, xshift=5 cm] (tcalc) {Calculate T};
		\node [code, below of = adjnsrc ] (adjn) {Adjoint neutron transport};
		%\node [file] (adjnflux) {Adjoint neutron flux};
		\node [script, below of = adjn] (vr) {Calculate MC VR parameters};
		\node [file, below of = vr, xshift =-3cm] (ww) {Weight windows};
		\node [file, below of = vr, xshift=3cm] (bsrc) {Biased source};

		\path [line] (adjp) -- (adjnsrc);
		\path [line] (tcalc) -- (adjnsrc);
		\path [line] (adjnsrc) -- (adjn);
		\path [line] (adjn) -- (vr);
		\path [line] (vr) -- (ww);
		\path [line] (vr) -- (bsrc);

	\end{tikzpicture}
	}
\end{figure}

\end{frame}

\begin{frame}
\frametitle{GT-CADIS Demonstration}

	\begin{columns}
        \column{0.5\textwidth}
        \begin{figure}
%	\centering
		%\vspace{1cm}
		\hspace{-1cm}
	\includegraphics[scale=0.27, xshift=-3cm]{config-0.jpg}
		\vspace{-0.9cm}
	\caption{Demo model. Steel chamber, walls are 2 m thick.  14 MeV
		neutron source in center.  Chamber surrounded by air.}
        \end{figure}

        \column{0.5\textwidth}
        \begin{figure}
	\centering
	\includegraphics[scale=0.20]{gtcadis_adjn.jpg}
		\caption{GT-CADIS adjoint neutron flux. Functions as importance
		map.}
	\end{figure}
	\end{columns}

\end{frame}

\begin{frame}
\frametitle{GT-CADIS Demonstration}
	GT-CADIS importance map is \textbf{insufficient} for moving systems.

	\begin{columns}
        \column{0.5\textwidth}
        \begin{figure}
%	\centering
	\vspace{0.8cm}
		\hspace{-1cm}
%	\includegraphics[scale=0.27, xshift=-3cm]{config-0.jpg}
	\transduration<0-5>{0}
	\multiinclude[<+->][format=jpg, graphics={ scale=0.22}]{config}
		\vspace{-0.3cm}
	\caption{Demo model. Steel chamber, walls are 2 m thick.  14 MeV
		neutron source in center.  Chamber surrounded by air.}
        \end{figure}

        \column{0.5\textwidth}
        \begin{figure}
	\centering
	\includegraphics[scale=0.20]{gtcadis_adjn_hi.jpg}
		\caption{GT-CADIS adjoint neutron flux. Functions as importance
		map.}
	\end{figure}
	\end{columns}

\end{frame}

\begin{frame}
\frametitle{Derive New Adjoint Neutron Source}
	\begin{itemize}
		\item{Geometry movement during photon transport effects the
			construction of the adjoint neutron source}
		\item{Need to:}
			\begin{enumerate}
				\item{Perform adjoint photon transport at each
					time step of geometry movement}
				\item{Integrate over time}
			\end{enumerate}
	\end{itemize}

  \begin{equation}\label{eq:adj_src_1_avg}
	 q_{n,v}^{+}(E_{n}) =
	 \int_{t}  \int_{E_{\gamma}}
	 T_{v}(E_n, E_{\gamma}, t) 
	 \phi_{\gamma}^{+}(\overrightarrow{r}_{v}(t), E_{\gamma},t)
	 \, dE_{\gamma} \, dt
  \end{equation}

\begin{itemize}
\item{$\phi_{\gamma}^{+}(\overrightarrow{r}_{v}(t), E_{\gamma},t) $ is the 
adjoint flux of photons of energy $E_{\gamma}$, in volume element v, at time t}
\item{$T_{v}(E_n, E_{\gamma}, t) $ is the T value of the material in volume
		element v, at decay time t}
\end{itemize}
\end{frame}

\begin{frame}
	\frametitle{Time-integrated GT-CADIS}
	\begin{itemize}
		\item{Perform deterministic adjoint neutron transport using the
			time-integrated source}
		\item{Resultant adjoint neutron flux should look something like
			this:}
	\end{itemize}
        \begin{figure}
	\centering
	\includegraphics[scale=0.20]{tgt_adj_n.jpg}
%		\caption{GT-CADIS adjoint neutron flux. Functions as importance
%		map.}
	\end{figure}
	\begin{itemize}
		\item{Use this adjoint neutron flux to generate biasing
			parameters that will optimize the MC neutron transport
			step of R2S.}
	\end{itemize}
\end{frame}

\begin{frame}
	\frametitle{Time-integrated R2S}
	\begin{itemize}
		\item{R2S workflow for geometry movement after shutdown}
	\end{itemize}
	\vspace{1cm}
	\begin{figure}
        \centering

        \tikzstyle{script} = [ellipse, fill=UWGold!30, draw, text centered, node distance=3cm]
        \tikzstyle{code} = [ellipse, fill=UWRed!30, draw, text centered, node distance=3cm]
        \tikzstyle{file} = [rectangle, draw, text centered, node distance=3cm]
        \tikzstyle{line} = [draw, -latex]
        \tikzstyle{arrow} = [thick, ->, >=stealth]
        
	\resizebox{!}{4cm}{
        \begin{tikzpicture}[node distance = 2 cm, auto]
		% Optimized neutron transport
		\node [code] (nmc) {Monte Carlo neutron transport};
                \node [file, left of = nmc, yshift=-0.9 cm, xshift=-1.8cm] (nflux) {Neutron flux};
		\node [file, right of = nmc, yshift=-0.9cm, xshift=1.9cm]
		(irrdec) {Irradiation and decay scenario};
		\node [script, below of = nmc, yshift=1.2cm] (act) {Activation analysis};
                \node [file, left of = act, yshift=-0.9cm, xshift=-1.8cm] (psrc) {Photon source};
		\node [code, below of = act, yshift=1.2cm] (pmc) {Monte Carlo
		photon transport at each time step};
                \node [file, below of = pmc, yshift=0.1cm] (sdr) {Shutdown dose
		rate at each time step};
		\path [line] (nmc) -- (nflux);
                \path [line] (nflux) -- (act);
                \path [line] (irrdec) -- (act);
                \path [line] (act) -- (psrc);
                \path [line] (psrc) -- (pmc);
		\path [line] (pmc) -- (sdr);

	\end{tikzpicture}
        }
%	\caption{R2S workflow \label{fig:r2s_flow}}
\end{figure}
\end{frame}
\begin{frame}
	\frametitle{Time-integrated GT-CADIS}
	\begin{itemize}
		\item{Assumptions}
			\begin{itemize}
				\item{Photon transport occurs much faster than
					geometry movement $\therefore$
					reasonable to do quasi-static
					simulation}
				\item{Period of geometry movement is short
					enough that the photon source will not
					change appreciably $\therefore$ can use
					same photon source for all MC
					calculations}
			\end{itemize}
		\item{Challenges}
			\begin{itemize}
				\item{Depending on complexity of model and
					fidelity of time resolution, can amass
					large number of CAD geometry files,
					volume mesh tally files}
				\item{Need to optimize this workflow in order
					to keep file storage at minimum}
			\end{itemize}
	\end{itemize}
	
\end{frame}

\begin{frame}[c]
	\frametitle{\tiny{.}}
	\begin{center}
	{\Huge Questions?}
	\end{center}
\end{frame}
\end{document}
